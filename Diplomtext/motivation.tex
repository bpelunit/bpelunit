Unternehmen sehen sich heute mit sehr dynamischen M�rkten konfrontiert und sind 
einem starken internationalen Wettbewerb ausgesetzt. Zu den Anforderungen solcher
M�rkte geh�ren oft flexible Gesch�ftsprozesse. Die notwendige Anpassungsf�higkeit der
Unternehmen und deren Gesch�ftsprozessen h�ngt meistens direkt von der vorhandenen
IT-Infrastruktur ab. Um diesen Anforderungen gerecht zu werden, muss die Softwarearchitektur
daf�r sorgen, dass IT-Systemen zum einen an den Gesch�ftsprozessen ausgerichtet
sind, zum anderen sehr einfach bei Ver�nderungen angepasst werden k�nnen. 
Eine m�gliche L�sung, die zur Zeit von vielen bevorzugt wird, sind Serviceorientierte Architekturen und Web Service Standards. 

Die Funktionen werden bei solchen Architekturen in Services gekapselt, die in Kompositionen zu vollautomatischen Gesch�ftsprozessen verkn�pft werden. F�r die Realisierung der Kompositionen steht mit WS-BPEL ein OASIS-Standard zu Verf�gung, der durch Unterst�tzung vieler wichtiger Softwarehersteller auf dem Weg zu einem Industriestandard ist. Damit �bernimmt BPEL  eine Schl�sselfunktion innerhalb der Konzeption einer serviceorientierten Architektur.

Die entscheidende Rolle der BPEL-Prozessen innerhalb eines Unternehmens erfordert hohe Qualit�tssicherung. Aufgrund der relativen Neuheit des ganzen Technologiebereichs gibt es auf diesem Gebiet noch wenige Erfahrungen.
Obwohl einige Konzepte und L�sungen f�r das systematische Testen von BPEL-Kompositionen bereits erarbeitet und umgesetzt wurden, wurden viele Methoden, die sich in der konventionellen Softwareentwicklung bew�hrt haben, noch nicht f�r BPEL angepasst. Die Testabdeckung, die im Zusammenhang mit Qualit�tssicherung der Software eine sehr wichtige Rolle spielt, z�hlt zu den Themen, die im BPEL-Kontext noch nicht behandelt wurde. 

Unter Testabdeckung versteht man Metriken f�r das Verh�ltnis zwischen zu testenden und tats�chlich getesteten Elementen des Pr�flings (in diesem Fall einer BPEL-Komposition). Diese Metriken werden  oft als Indikatoren f�r die Qualit�t und Fortschritt der Tests verwendet. Es fehlen f�r BPEL bislang sowohl Definitionen der Metriken als auch Verfahren zur Ermittlung der Testabdeckung. Diese Masterarbeit besch�ftigt sich mit genau diesen Themen.

Aus praktischer Sicht ist die Integration der Testabdeckungsmetriken in ein Test-Werkzeug besonders sinnvoll. Mit dem schnellen Zugriff auf die erreichten Testabdeckungswerte ist st�ndige und unmittelbare Kontrolle der Testqualit�t m�glich. Im praktischen Teil dieser Arbeit wird das BPELUnit-Framework, das das Testen von BPEL-Kompositionen unterst�tzt, erweitert. Als Ergebnis eines Tests soll das Framework neben der Aussage, ob der Test erfolgreich war oder nicht, auch die erreichte Testabdeckung liefern. 



