 Allerdings bietet uns das neue Paradigma der Service Orientierten Architektur (SOA) zusammen mit den aktuellen, technologischen Fortschritten aus dem Umfeld der Web Services mittlerweile schon als sehr gut zu bezeichnende Unterst�tzung auf dem Weg zum flexiblen Echtzeitunternehmen. 
 In der heute sehr konkurenzbetonten Weltwirtschaft unterliegen die Gesch�ftsprozesse einer stetigen Ver�nderung: Unernehmen m�ssen st�ndig beobachten, wie sich die Marktbedingungen �ndern, und ihre Strategie ebtsprechend anpassen, um diese Ver�nderungen wiederzuspiegeln. Eine Schl�sselanforderung f�r eine moderne Enterprise IT ist also, dass Ver�nderungen in den IT-Systemen des Unternehmens schnell und effizient ber�cksichtigt werden k�nnen. Wichtige Motivation f�r die Einrichtung der SOA ist das Bestreben, die Agilit�t der Systemen der IT zu steigern. BPEL bietet die entsprechenden M�glichkeiten, wenn es darum geht, dass verschiedene Web Services zusammenarbeiten sollen, um gemeinsam nicht nur eine einzelne Aktivit�t, sondern einen kompletten Gesch�ftsprozess technisch abzubilden und zu realisieren.