Es werden in dieser Arbeit die Testabdeckungsmetriken, die f�r BPEL-Prozesse geeignet sind , bestimmt und formal definiert. 
Einerseits werden dabei die aus der konventionellen Softwareentwicklung bekannten Metriken untersucht und f�r BPEL angepasst. Die Betrachtung beschr�nkt sich dabei auf Anweisungs- und Zweigabdeckung. Andererseits werden auch neue spezifischen Metriken, die die wichtigen Konzepte der Sprache ber�cksichtigen, bestimmt. 

Anschlie�end wird ein Konzept zur Erfassung der Metriken erarbeitet und Implementiert.

Die Realisierung wird dabei als Erweiterung des BPELUnit -Frameworks umgesetzt werden. BPELUnit ist ein Framework zum Testen von BPEL-Kompositionen, das am Fachgebiet Software Engineering im Rahmen einer Masterarbeit entwickelt wurde. 

Im Rahmen einer Masterarbeit wurde am Fachgebiet Software Engineering der Universit�t Hannover ein Framework BPELUnit entwickelt.  BPELUnit ist Unit Testing Framework f�r BPEL Kompositionen. Die Tests, die durch das Framework ausgef�hrt werden, k�nnen dauerhaft gespeichert und immer wieder ausgef�hrt werden. Durch die Ausf�hrung der Tests kann festgestellt werden, ob die durch BPEL formulierte Kombination der Dienste die funktionalen Anforderungen erf�llt. Die Aussage, dass alle Tests erfolgreich sind, reicht alleine nicht aus, um die Qualit�t der Software zu bewerten. Unter anderem muss die Qualit�t der Tests dabei ber�cksichtigt werden. Dar�ber liefert \nobreak{BPELUnit} aber keine Information. Die Aufgabe dieser Masterarbeit schlie�t sich an dieser Stelle an. Dabei steht die Testabdeckung, als ein wichtiges Merkmal der Softwarequalit�t, im Focus. 

Als Erstes sollen die geeigneten Metriken f�r die Abdeckung der Tests eines BPEL-Prozesses gefunden werden. Es soll �berpr�ft werden, inwiefern die in der Softwaretechnik bekannten Ma�e (z.B. Zweig�berdeckung) sich f�r BPEL-Prozesse eignen. Bei Bedarf  sollen entweder die bekannten Metriken angepasst oder neue definiert werden. Im zweiten Schritt soll ein Konzept erarbeitet werden, mit dem die Messung der Testabdeckung realisiert werden kann. Um eine einfache Integration in das BPELUnit zu erm�glichen, muss die Architektur des Frameworks ber�cksichtigt werden. Danach soll das BPELUnit-Framework entsprechend erweitert und angepasst werden. Die anschlie�ende Anpassung der Clients soll die neue Funktionalit�t  dem Benutzer  zur Verf�gung stellen.
