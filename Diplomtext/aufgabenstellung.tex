In dieser Arbeit werden die Testabdeckungsmetriken f�r BPEL-Prozesse formal definiert. Dabei werden die bekannten Metriken (Anweisungs- und Zweigabdeckung) auf BPEL-Kompositionen �bertragen und neue BPEL-spezifischen Metriken bestimmt, die die wichtigen Eigenschaften der Sprache abdecken. Die Metriken werden so gestaltet, dass die 100\% Abdeckung, was theoretisch optimal ist, bei jedem BPEL-Prozess erreichbar ist. 

Es werden Verfahren f�r die Ermittlung der Abdeckungsmetriken in BPEL-Kompositionen vorgestellt und bewertet. Unter Ber�cksichtigung des ausgew�hlten Verfahrens werden Konzepte zur Ermittlung der Metriken vorgestellt.

Als Realisierung wird im praktischen Teil dieser Arbeit das BPELUnit Framework erweitert und die zugeh�rigen Clients angepasst. Einfache Konfigurationsm�glichkeit, die Transparenz der Testabdeckungsmessung und detailierte und verst�ndliche Ergebnisdarstellung stehen dabei im Vordergrund. 
 