Unternehmen sind heutzutage an IT-Systemen interessiert, die die Gesch�ftsprozesse unterst�tzen und flexibel auf die �nderungen reagieren k�nnen. Bei der Gestaltung der IT-Architektur setzen dabei viele aus SOA.

Service Oriented Architecture verspricht diese F�ahigkeit zur effektiven und flexiblen Unterst
�utzung sich immer schneller �andernder Gesch�aftsprozes
Die Unternehmen sind heutzutage an IT-Systemen interessiert, die die Gesch�ftsprozesse unterst�tzen und flexibel auf Ver�nderungen reafieren k�nnen.  

  Unternehmen sehen sich heute mit sehr dynamischen M�rkten konfrontiert und sind oft einem starken internationalen Wettbewerb ausgesetzt. Zu den Anforderungen solcher M�rkte geh�ren oft flexible Gesch�ftsprozesse. Die notwendige Anpassungsf�higkeit der Unternehmen und deren Gesch�ftsprozessen h�ngt meistens direkt von der vorhandenen IT-Infrastruktur ab. Um diesen Anforderungen gerecht zu werden muss die Softwarearchitektur daf�r sorgen, dass IT-Systemen zum einen  an den Gesch�ftsprozessen ausgerichtet sind, zum anderen sehr einfach bei Ver�nderungen angepasst werden k�nnen. Service Oriented Architecture verspricht diese F�higkeit zur effektiven und flexiblen Unterst�tzung sich immer schneller �ndernder Gesch�ftsprozesse. 
  
  Service Oriented Architecture (SOA) ist ein Ansatz zum Entwurf verteilter Systeme. \textit{Die Kernidee der SOA, die Unternehmensfunktionalit�t
als Menge voneinander unabh�ngiger Dienste zur Verf�gung
zu stellen, bildet die Basis f�r Integration und Dynamik} \cite{THOMAS2005}. Eine m�gliche Definition f�r SOA:
\begin{Def}Eine serviceorientierte Architektur ist ein
Konzept f�r eine Systemarchitektur, in
welchem Funktionen in Form von wieder
verwendbaren, voneinander unabh�ngigen
und lose gekoppelten Services implementiert
werden. Services k�nnen unabh�ngig von
zugrunde liegenden Implementierungen �ber
Schnittstellen aufgerufen werden, deren
Spezifikationen �ffentlich sind. Serviceinteraktion
findet �ber eine daf�r vorgesehene
Kommunikationsinfrastruktur statt.
Quelle Wikipedia 23.01.2006
\end{Def} 
  Die Funktionen einer Anwendung sind als Service organisiert, die 
 beliebig verteilt sein k�nnen und lassen sich dynamisch zu Gesch�ftsprozessen verbinden. Die zugrunde liegenden technischen Plattformen der einzelnen Services spielen dabei keine Rolle. Zu betonen ist, dass SOA eine Systemarchitektur und keine Technologie beschreibt. 


 