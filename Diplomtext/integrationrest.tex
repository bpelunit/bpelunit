Ein BPEL-Prozess besteht aus der Gesch�ftslogik (definiert in BPEL), Servicebeschreibung (WSDL) und optional aus weiteren Datentypen (XML Schema). Diese Informationen werden zusammengefasst und in eine BPEL Engine deployt. Die Engine ist unter anderem daf�r verantwortlich, die \textit{Endpoints} der Partner des BPEL Prozesses zu spezifizieren. F�r jeden Link muss bekannt sein, welche WSDL service und port und welche Adresse f�r die Kommunikation benutzt werden sollen. Diese Information wird in so genannten Deployment descriptoren festgelegt.
   Das Deployment und die Syntax der Deploymentdescriptoren ist nicht spezifiziert.



F�r das Testen, wie bereits erl�utert, muss BPEL-Prozess in eine Engine deployt werden. Die BPEL-Dateien mit  den Deploymentinformation werden in einem Archive zusammengefasst und an die BPEL Engine �bergeben. Damit der Web Service f�r den BPEL-Prozess erreichbar ist, muss die WSDL Beschreibung des Web Services in dem Archive enthalten sein. Au�erdem muss f�r die Kommunikation mit diesem Web service ein PartnerLink definiert. F�r diesen Link muss in dem Deployment descriptor die WSDL service und port sowie die konkrete Adresse (in diesem Fall eine feste Adresse auf dem lokalen Host) festgelegt werden.
   Das Deployment und die Syntax der Deploymentdescriptoren ist nicht spezifiziert.

PARTNERLINKTYPE wird in der WSDL Platziert

 Hierzu m�ssen von der jeweiligen BPEL Engine abh�ngige Deployment-Informationen bereitgestellt werden. 

In disem Abschnitt werden zuerst zwei M�glichkeiten f�r die Integration der Messung der Testabdeckung in das Framework vorgestellt. Nach einem Vergleich der beiden Ans�tze wird einen als L�sung ausgew�hlt

Die Aufgabe 

die SOAP-Nachrichten, die an Coverage Receiver gerichtet sind m�ssen dekodiert werden und  an den Servece (Mock) weitergeleitet werden.  

Ansatz1
Es kann ein spezieller Partner Track f�r den Service definiert und zu jedem TestCase zugeordnet werden. Zu jedem TestCase kann dann ein Thread gestartet werden, der die Coverage Nachrichten empf�ngt. Es reicht aus jedem TestCase aus. Es muss die Logik f�r den Manager der Marken implementiert werden. Daf�r kann entweder die receive Aktivit�t mehrmals hintereinander geschaltet werden, oder eine neue Aktivit�t, die alle Nachrichten akzeptiert.



 Das Framework dekodiert und leitet weiter , man braucht sich keine Gedanken zu machen. 

nachteil gr��ere Eingriffe in das Framework und an vielen Stellen:
\begin{itemize}
	\item beim Start jedes Testcases muss ein zus�tzliches Thread gestartet werden
	\item beim shutdown unber�cksichtigt 
	\item beim Auswerten des Ergebnissen muss dieser Thread unber�cksichtigt bleiben 
\end{itemize}



Ansatz2

Direkt nachdem Empfangen der Nachricht wird diese an den Receiver weitergeleitet. Dieser muss sich um die Dekodierung der SOAP Nachrichten k�mmern


Dadurch dass beim ersten Ansatz die ganze Funktionalit�t  der Kommunikation (der �bermittlung und Dekodierung der Nachrichten) durch das Framework erledigt werden erscheint diese M�glichkeit als eleganter. Aber Velangsamen