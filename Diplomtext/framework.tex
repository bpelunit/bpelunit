  BPELUnit richtet sich prim�r an die Entwickler, die das Testen und Programmieren w�hrend des Entwicklungsprozesses abwechseln machen. Also bietet das  Framework schnelles Testen: Erzeugen und Ausf�hren sind einfach und schnell. 
   Testansatz von BPELUnit 
   
   BPELUnit f�r wiederholbare white-box automatisierbare BPEL Kompositionen Testen
   mit Konzentration auf:
   
\begin{itemize}
	\item Unterst�tzung von BPEL-spezifischen Eigenschaften: parallelen Aktivit�ten, Kompositionen und asynchronen Eigenschaften
	\item einfache Benutzung durch Entwickler
	\item realistische Sicht auf das Ausf�hren des BPEL Prozesses.
\end{itemize}
 
 Unit Testen ist das Testen von Komponenten in der Isolation. Das erreicht BPELUnit durch Mocks , die die Partner des Prozesse mit spezieller f�r den Test erzeugter Implementierung ersetzen. Testgeschirr ersetzt die Partner des BPEL Prozesses und den Client.
 
 BPELUnit Testing Architektur
 
\begin{itemize}
	\item eindeutig
	\item maschienenlesbar
	\item ausf�hrbar 
\end{itemize}
 
 BPELUnit   enth�lt alle relevanten Daten f�r das Testen eines bestimmten Pfades in BPEL prozess.
 
 Die Daten, die bei der Interaktion mit PUT verschickt werden, sind in XML spezifiziert. Bei der �berpr�fung der ankommenden Daten kann XPath eingesetzt werden um die relevanten Daten auszuw�hlen. 
 
 
Testen von BPEL bedeutet �berpr�fen die Korrektheit des implementierten Gesch�ftsprozesses. Daf�r m�ssen die Gesch�ftsprotokolle des Clienten und der Partner simuliert werden. Gesch�ftsprotokoll basiert auf Interaktion . Die Tester m�ssen die Interaktionssequenzen f�r die Partner und den Client in jedem Testcase festlegen. Sequenzen werden Tracks genannt und bestehen aus atomaren Interaktionen (Aktivit�ten)

Fault Handling

Bei synchronem Aufruf k�nnen Operationen Faults verschicken. BPELUnit ist f�hig die Faults zu empfangen und zu versenden. Behandlung von den Fehlern ist sehr wichtig f�r ein Testframework weil es erm�glicht dem Tester die Kompensation zu veranlassen. So kann die Korrektheit von Kompensationsaktivit�ten getestet werden.

PUT hat ein Client und eine beliebige Anzahl von Partner . Alle diese Web Services laufen Parallel und interagieren mit PUT.

Testorganisation  

Wie bei einem konventionelem Ansatz sind die Testcases in einer suite organisiert. Zus�tzlich ist jede Suite mit einem bestimmten BPEL Prozess verkn�pft. Die Fixture ist in Suite spezifiziert und ist f�r jedem Testcase.

BPELUnit kombiniert TEstcases mit a setup/shutdown (f�r Partner und Client m�ssen deployt und undeployt werden ) Teil und referenziert externe Artefakte (BPELProzess,WSDL-Beschreibungen) 

Test Ausf�hrung 

BPEL Kompositions Testen braucht Ausf�hrung des BPEL Prozesses innerhalb von Testgeschirr die die Input und Output Daten entsprechend des Testspezifikation behandelt.
Real-Life Testen hei�t deployen PUT in einer Engine und Aufrufen mittels Web Service Aufruf. Die Partner werden durch Mocks Web Services ersetzt, die Aufgerufen werden k�nnen und selbst Web Service Aufrufe starten kann. Dabei muss das Deployment von PUT so erfolgen, dass alle Partner URIs ersetzt durch URIs des Mocks werden. Die Engine beruhen auf anbieterspezifischen Deployment descriptors, die nicht standarfisiert sind . 

PUT ist deployt in eine Engine vor dem Testen ind wird undeployt nach dem testen . Es werden Adapter f�r jede Engine gebraucht Also die Aufrufe des Prozesses ind empfangene Aufrufe sind in SOAP kodiert. Simulation der Partners muss die WSDL Beschreibung beachten. Die Adressierung der Antworten bei asynchronen Aufrufen muss beachtet werden.

BILD

SOAP Calls und Encoding 

real life F�r die Kommunikation mit BPEL Prozess die Standards (WSDl -Beschreibung beachten und Kommunikation findet �ber SOAP Nachrichten statt)
Die SOAP Daten m�ssen korrekt kodiert und dekodiert werden. Alle Aktivit�ten sind an WSDL Operation zu binden.


Test Results

Real Life schwierig die Ergebnisse zu sammeln
\begin{itemize}
	\item Failure Fehler auf der Anwendungsebene : zum Beispiel eine erwartete Nachricht von BPEL ist nicht angekommen , oder enth�lt falsche Informationen oder zu viele Nachrichten sind angekommen.
	\item Error Problem mit dem Web Service Stack. Problemen auf Kommunikationsebene.
\end{itemize}
 
 Grundlegendes Resultat eines Unit Tests ist ob der Test gellungen ist