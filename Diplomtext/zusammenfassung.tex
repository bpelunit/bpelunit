\thispagestyle{empty} 
\vspace*{5cm}
\begin{center}
\textbf{Zusammenfassung}
\end{center}

Business Process Execution Language (BPEL) hat sich in der Industrie als Mittel f�r die Kombination der Web Services zu einem Prozess etabliert. Die dabei abgebildeten Gesch�ftsprozesse sind f�r das Unternehmen sehr wichtig. Demzufolge haben die Korrektheit und die Robustheit des BPEL-Prozesses sehr hohe Priorit�t. Aus diesem Grund m�ssen die BPEL-Prozessen besonders gr�ndlich getestet werden. Allerdings gibt es f�r BPEL noch keine praktischen Metriken, die f�r die Bewertung der Testqualit�t oder Testfortschritts geeignet sind. In dieser Arbeit werden Testabdeckungsmetriken wie die aus der klassischen Softwareentwicklung bekannten Statement- und Zweigabdeckung auf BPEL �bertragen und entsprechend angepasst. Zus�tzlich werden BPEL-spezifischen Metriken eingef�hrt, die die wichtigen Eigenschaften der Sprache gesondert ber�cksichtigen. Anschlie�end wird die Erweiterung der BPELUnit Frameworks pr�sentiert, die einfache Sammlung und Pr�sentation der Metriken erm�glicht. Dadurch k�nnen die Tester besser den Fortschritt und die Qualit�t ihrer Arbeit kontrollieren.  

Business Process Execution Language (BPEL) hat sich als ein Quasi-Standard f�r die Komposition mehrerer Web Services zu einem Gesch�ftsprozess etabliert. Trotz der breiten Akzeptanz dieser Sprache existieren noch sehr wenige Werkzeuge, die das Testen von BPEL-Prozessen unterst�tzen. Mit \textit{BPELUnit} wurde ein Framework entwickelt, der das strukturierte Testen einer BPEL-Komposition in Isolation (unabh�ngig von zusammengestellten Diensten) erm�glicht. Die Testabdeckungsanalyse, als Mittel zur Qualit�tskontrolle von Tests, wird durch das Framework nicht durchgef�hrt. Genau an dieser Stelle kn�pft diese Masterarbeit an.  

Es wird in der Arbeit untersucht, inwiefern die bekannten Testabdeckungsmetriken, die in konventionellen Programmiersprachen zur Bewertung der Tests eingesetzt werden, auf die BPEL-Sprache �bertragen werden k�nnen. Au�erdem werden neue f�r BPEL sinnvolle Metriken ausgearbeitet. Anschlie�end soll ein Konzept f�r die Integration dieser Metriken in das BPELUnit-Framework erarbeitet und implementiert werden.


