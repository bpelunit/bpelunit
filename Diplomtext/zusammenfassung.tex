\thispagestyle{empty} 
\vspace*{5cm}
\begin{center}
\textbf{Zusammenfassung}
\end{center}

Business Process Execution Language (BPEL) hat sich in der Industrie als Mittel f�r die Kombination der Web Services zu einem Prozess etabliert. Die dabei abgebildeten Gesch�ftsprozesse sind f�r das Unternehmen sehr wichtig. Demzufolge haben die Korrektheit und die Robustheit der BPEL-Prozesse sehr hohe Priorit�t, was ein gr�ndliches Testen der Prozesse notwendig macht. Allerdings gibt es f�r BPEL noch keine praktischen Metriken, die f�r die Bewertung der Testqualit�t oder Testfortschritts geeignet sind. In dieser Arbeit werden Testabdeckungsmetriken aus der klassischen Softwareentwicklung (Statement- und Zweigabdeckung) auf BPEL �bertragen und entsprechend angepasst. Zus�tzlich werden BPEL-spezifischen Metriken eingef�hrt, die die wichtigen Eigenschaften der Sprache gesondert ber�cksichtigen. Anschlie�end wird die Erweiterung des BPELUnit-Frameworks pr�sentiert, die einfache Sammlung und Pr�sentation der Metriken erm�glicht. Mit dieser Erweiterung k�nnen die Tester einfach und schnell den Fortschritt und die Qualit�t ihrer Arbeit kontrollieren.  

