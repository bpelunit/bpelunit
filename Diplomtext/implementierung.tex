Die Information �ber tats�chliche Ausf�hrung der Aktivit�ten hat einen Mehrwert gegen�ber der Information, dass eine Aktivit�t zur Ausf�hrung stimuliert wurde (aber evtl. nicht erfolgreich ausgef�hrt werden konnte). Deswegen erfolgt das Logging jeweils \textit{nach} der Ausf�hrung der entsprechenden Aktivit�t.  Die Aktivit�ten, die den normalen Kontrollfluss �ndern oder den Prozess beenden, k�nnen allerdings auf diese Weise nicht bzw. nur sehr umst�ndlich registriert werden:  
\begin{itemize}
	\item \textit{throw}
	\item \textit{rethrow}
	\item \textit{compensate}
	\item \textit{compensateScope}
	\item \textit{exit}
\end{itemize}
\textit{Exit} beendet den Prozess und kann logischerweise nur vor der Ausf�hrung erfasst werden. Die \textit{throw-}, \textit{rethrow-}, \textit{compensate-} und \textit{compensateScope-}Aktivit�ten unterbrechen den normalen Kontrollfluss und veranlassen die Ausf�hrung der entsprechenden Handler. Obwohl die tats�chliche Ausf�hrung dieser Aktivit�ten in diesen H�ndlern registriert werden k�nnte, wurde auf Grund eines gro�en zus�tzlichen Aufwands und dazu verh�ltnism��ig kleines Informationsgewinns entschieden, darauf zu verzichten und die Aktivit�ten, wie bei \textit{Exit}, vor der Ausf�hrung zu loggen. 
Aus diesem Grund werden die entsprechenden Aktivit�ten auch dann geloggt (als ausgef�hrt markiert), wenn bei der Ausf�hrung ein Fehler auftreten sollte.


Statement Ziel eines Links Also sequence drum