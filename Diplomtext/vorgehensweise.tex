 




 
 

Ein BPEL-Prozess kann nach au�en grunds�tzlich nur mit Web Services kommunizieren. Demzufolge m�sste der Dienst f�r den Empfang der Nachrichten als Web Service realisiert und w�hrend des Testens erreichbar sein.  Das BPELUnit Framework bringt Funktionalit�ten mit, die es m�glich machen, diesen Dienst einfacher zu implementiert. W�hrend die entsprechende L�sung im Abschnitt genau vorgestellt wird, reicht es zun�chst aus zu wissen, dass die Schnittstelle des Dienstes mit WSDL beschrieben ist, und f�r den PUT aus der Ablaufumgebung genau wie alle anderen Web Services erreichbar ist. 
 Die folgende stellt die Funktionsweise
 
 

 Wie bereits im Abschnitt erkl�rt wurde, wird die Gesch�ftslogik in BPEL durch Aktivit�ten realisiert.  Die Instrumentierung eines BPEL-Prozessen kann demnach nur durch Hinzuf�gen von Aktivit�ten realisiert werden. Der Ablauf bei der Messung der  Testabdeckung: 
 
\begin{itemize}
\item Instrumentierung
\begin{itemize}
\item den BPEL-Prozess analysieren
\item die statischen Daten �ber den Prozess sammeln
\item F�r jede Metrik: den BPEL-Prozess mit Marken annotieren und diese Daten f�r die Auswertung speichern
	\item die Marken von allen Metriken , die im Kontrollflu� unmittelbar hintereinanderliegen, werden zusammengefasst
	\item Neue Variablen im Prozess definieren und die Marken an die Variablen zuordnen (assign-Aktivit�ten)
	\item Invoke-Aktivit�ten, die die Marken an den Empf�nger schicken.		
\end{itemize}
\item die Daten bei der Ausf�hrung des Prozesses sammeln (Nachrichten mit Marken empfangen) und den statischen Daten zuordnen
\item die Daten auswerten und die Metriken ermitteln
\end{itemize}


Die ersten drei Aufgaben k�nnen zusammen in einem Schritt durchgef�hrt werden. Im n�chsten Abschnitt wird dieser Schritt "`Annotation des Prozesses"' genau f�r jede Metrik erl�utert. 

