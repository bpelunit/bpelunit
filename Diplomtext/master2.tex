\documentclass[11pt,a4paper]{scrreprt}

\usepackage[latin1]{inputenc}   
\usepackage[ngerman]{babel}           % deutsche Trennungen und Bezeichnungen
\usepackage{amsmath,amsfonts,amssymb}
%\usepackage{natbib}
%\citestyle{dinat}
\usepackage{typearea}   
\usepackage{url}         % Satzspiegelberechnung: Texth�he und -breite
\typearea{12}                   % je h�her das Argument, desto kleiner die R�nder
\usepackage{graphicx}           % zur Graphikeinbindung
\usepackage{amssymb,amsmath}	% mathematische Symbole
\usepackage{color}
\usepackage{picins} 					%Bilder umflie�en
\usepackage{multicol}
\usepackage{placeins}					%Gleitobjekte erzwingen \FloatBarrier 					
\usepackage{pifont}  					%Symbole
\usepackage{listings} \lstset{numbers=left, numberstyle=\tiny, numbersep=5pt} \lstset{language=XML}
\usepackage{wrapfig}
\usepackage{pdftricks}			% Zeichnen von B�umen
\usepackage{fancyhdr} %Kopf Fu�zeile
\usepackage{amsthm} %definitionen
\usepackage[pagebackref, colorlinks=true, urlcolor=black, linkcolor=black,bookmarksnumbered=true,linktocpage=false,citecolor=black]{hyperref}

  %  *  breaklinks=true/false: Gibt an, ob Links umgebrochen werden d�rfen.
  %  * linktocpage=true/false: im Inhaltsverzeichnis sind nur die Seitenzahlen links, nicht der Text
 %   * colorlinks=true/false: Links werden eingef�rbt (Farben werden mit linkcolor, anchorcolor ... festgelegt)
 %   * linkcolor=Farbe: Farbe des verlinkten Textes, Dokument-interne Links
 %   * citecolor=Farbe: Farbe des verlinkten Textes, Links zum Literaturverzeichnis
 %   * filecolor=Farbe: Farbe des verlinkten Textes, Links auf lokale Dateien
 %   * pagecolor=Farbe: Farbe des verlinkten Textes, Links auf andere Seiten
  %  * urlcolor=Farbe: Farbe des verlinkten Textes, externe URLs
 %   * frenchlinks=true/false: Links werden als smallcaps, anstatt farbig dargestellt.
%    * breaklinks=true/false: Gibt an, ob Links umgebrochen werden d�rfen.

%\begin{psinputs}
%\usepackage{pst-all}
%\end{psinputs}			
\setcounter{secnumdepth}{3} %Gibt an wieviele Ebenen nummeriert werden
%% ... und nimm alle 4 Ebenen in das Inhaltsverzeichnis auf.
\setcounter{tocdepth}{3}
\addtolength{\topmargin}{.5cm}  	% Rand oberhalb der Kopfzeile
%\setlength{\footskip}{1.2cm}		% Abstand Text - Fusszeile
%\oddsidemargin+2mm					% R�nder f�r ungerade Seiten
%\addtolength{\evensidemargin}{-2cm}	% R�nder f�r gerade Seiten
\textheight24cm						% Vertikale Texth�he

\parskip4mm					%Abstand zwischen Abs�tzen
\parindent0mm					%Einr�ckung bei Absatzbeginn


\newtheorem{satz}{Satz}[section]
\newtheorem{lem}{Hilfssatz}

\theoremstyle{definition}
\newtheorem*{Def}{Definition}

\pagestyle{fancyplain}
\fancyfoot[R]{\thepage}
\lhead{}
\chead{}
\cfoot{}
\renewcommand{\chaptermark}[1]{
\markboth{#1}{}}


\begin{document}
%\maketitle			
%\newpage
\thispagestyle{empty}
\vspace*{10cm}
\Large{\textbf{Eidesstattliche Erkl�rung}}\vspace{1cm}

\normalsize{Hiermit versichere ich, Alex Salnikow, die vorliegende Masterarbeit ohne fremde Hilfe und
nur unter Verwendung der von mir aufgef�hrten Quellen und Hilfsmittel angefertigt zu haben.\vspace{1cm}

Hannover, \today  \ \ \ \ \ \ \ \ \ \ \ \ \ \ \ \ \ \ \ \ \ \ \ \ \ \ \ \ \ \ \ \ \ \ \ \ \ \ \ \ \ \ \ \ \ \ ............................\\

}


\begin{titlepage}
\begin{center}
\vspace*{1cm}
		{
		\normalsize\textsc{Gottfried Wilhelm\\
Leibniz Universit�t Hannover\\
Fakult�t f�r Elektrotechnik und Informatik\\
Institut f�r Praktische Informatik\\
Fachgebiet Software Engineering}\\[2.5cm]
		}
		
		\huge{\textbf{Thema}}\\[3cm]
		
		\normalsize{Masterarbeit\\[1cm]
im Studiengang Informatik\\[1cm]
von\\[1cm]
Alex Salnikow}\\[3cm]

\normalsize{Pr�fer: Prof. Dr. Kurt Schneider\\
Zweitpr�fer: \\
Betreuer: Dipl.-Wirt.-Inform. Daniel L�bke}\\[1cm]
\today
\end{center}
\end{titlepage}	
%\vspace{1cm}
%\newpage
%\thispagestyle{empty}
%{\Large \bf Zusammenfassung}
\thispagestyle{empty} 
\vspace*{5cm}
\begin{center}
\textbf{Zusammenfassung}
\end{center}
Business Process Execution Language (BPEL) hat sich als ein Quasi-Standard f�r die Komposition mehrerer Web Services zu einem Gesch�ftsprozess etabliert. Trotz der breiten Akzeptanz dieser Sprache existieren noch sehr wenige Werkzeuge, die das Testen von BPEL-Prozessen unterst�tzen. Mit \textit{BPELUnit} wurde ein Framework entwickelt, der das strukturierte Testen einer BPEL-Komposition in Isolation (unabh�ngig von zusammengestellten Diensten) erm�glicht. Die Testabdeckungsanalyse, als Mittel zur Qualit�tskontrolle von Tests, wird durch das Framework nicht durchgef�hrt. Genau an dieser Stelle kn�pft diese Masterarbeit an.  

Es wird in der Arbeit untersucht, inwiefern die bekannten Testabdeckungsmetriken, die in konventionellen Programmiersprachen zur Bewertung der Tests eingesetzt werden, auf die BPEL-Sprache �bertragen werden k�nnen. Au�erdem werden neue f�r BPEL sinnvolle Metriken ausgearbeitet. Anschlie�end soll ein Konzept f�r die Integration dieser Metriken in das BPELUnit-Framework erarbeitet und implementiert werden.



BPEL Prozesse werden f�r die Verbindung der WEb Services zu einer Komposition verwendet. Diese Kompositionen spiegeln die Gesch�ftsprozesse die sehr wichtig f�r die Unternehmen sind. Daraus folgend die Korrektheit und die Robustheit des BPEL-Prozesses ist sehr wichtig. Das Testen von BPEL-Prozessen ist deswegen zwingend erforderlich. Aber es gibt keine praktischen Metriken, die verwendet werden k�nnen um die Testqualit�t oder Testvortschritt zu messen. In dieser Arbeit werden Testmetriken definiert wie Statementabdeckung und Zweig. Zus�tzlich werden BPEL-spezifischen Metriken eingef�hrt , die auf die wichtigen Eigenschaften der Sprache gerichtet sind. Anschlie�end wird die Erweiterung der BPELUnit Frameworks pr�sentiert. Dieses Unterst�tzt einfache Sammlung und Pr�sentation der Metriken. Dadurch k�nnen die Tester besser den Fortschritt und die Qualit�t ihrer Arbeit kontrollieren.  
%\vspace{1cm}
  
  
  
  

\newpage
%\thispagestyle{empty}~
\tableofcontents

\chapter{Einleitung}
\section{Motivation}
In der heute sehr konkurenzbetonten Weltwirtschaft unterliegen die Gesch�ftsprozesse einer stetigen Ver�nderung: Unernehmen m�ssen st�ndig beobachten, wie sich die Marktbedingungen �ndern, und ihre Strategie ebtsprechend anpassen, um diese Ver�nderungen wiederzuspiegeln. Eine Schl�sselanforderung f�r eine moderne Enterprise IT ist also, dass Ver�nderungen in den IT-Systemen des Unternehmens schnell und effizient ber�cksichtigt werden k�nnen. Wichtige Motivation f�r die Einrichtung der SOA ist das Bestreben, die Agilit�t der Systemen der IT zu steigern. 
Unternehmen sehen sich heute mit sehr dynamischen M�arkten konfrontiert und sind oft
einem starken internationalen Wettbewerb ausgesetzt. Zu den Anforderungen solcher
M�arkte geh�oren oft flexible Gesch�aftsprozesse. Die notwendige Anpassungsf�ahigkeit der
Unternehmen und deren Gesch�aftsprozessen h�angt meistens direkt von der vorhandenen
IT-Infrastruktur ab. Um diesen Anforderungen gerecht zu werden muss die Softwarearchitektur
daf�ur sorgen, dass IT-Systemen zum einen an den Gesch�aftsprozessen ausgerichtet
sind, zum anderen sehr einfach bei Ver�anderungen angepasst werden k�onnen.
Viele setzen dabei auf SOA und Web Service Standards . 

Die Funktionen werden in Services gekapselt, die in Kompositionen zu vollautomatischen Gesch�ftsprozessen verkn�pft werden. F�r die Realisierung der Kompositionen steht mit WS-BPEL ein OASIS-Standard zu Verf�gung, der durch Unterst�tzung vieler wichtiger Softwarehersteller auf dem Weg zu einem Industriestandard ist. Damit �bernimmt BPEL  eine Schl�sselfunktion innerhalb der Konzeption einer serviceorientierten Architektur.

Die entscheidende Rolle der BPEL-Prozessen innerhalb eines Unternehmens erfordert hohe Qualit�tssicherung. Aufgrund der relativen Neuheit des ganzen Technologiebereichs gibt es in diesem Bereich noch wenige Erfahrungen. Viele Methoden, die sich in der konventionellen Softwareentwicklung bew�hrt haben,  sind noch nicht auf SOA �bertragen. W�hrend einige Konzepte und L�sungen f�r das systematische Testen von BPEL-Kompositionen bereits erarbeitet wurden, gibt es noch keine L�sungen zur Erfassung der Testabdeckung. 

Unter Testabdeckung versteht man Metriken f�r das Verh�ltnis zwischen zu testenden und tats�chlich getesteten Elementen des Pr�flings (in diesem Fall eine BPEL-Komposition). Diese Metriken werden  oft als Indikatoren f�r die Qualit�t und Fortschritt des Tests verwendet. 

In dieser Arbeit werden die Testabdeckungsmetriken f�r BPEL-Prozesse formal definiert. Dabei werden die bekannten Metriken (Anweisungs- und Zweigabdeckung) auf BPEL-Prozessen �bertragen und neue BPEL-spezifischen Metriken definiert, die die wichtigen Konzepten der Sprache abdecken. Es wird ein Konzept f�r die Messung der Testabdeckung erarbeitet und als Erweiterung des BPELUnit Frameworks implementiert.


 Allerdings bietet uns das neue Paradigma der Service Orientierten Architektur (SOA) zusammen mit den aktuellen, technologischen Fortschritten aus dem Umfeld der Web Services mittlerweile schon als sehr gut zu bezeichnende Unterst�tzung auf dem Weg zum flexiblen Echtzeitunternehmen. 
\section{Beitrag dieser Arbeit}
Im Rahmen einer Masterarbeit wurde am Fachgebiet Software Engineering der Universit�t Hannover ein Framework BPELUnit entwickelt.  BPELUnit ist Unit Testing Framework f�r BPEL Kompositionen. Die Tests, die durch das Framework ausgef�hrt werden, k�nnen dauerhaft gespeichert und immer wieder ausgef�hrt werden. Durch die Ausf�hrung der Tests kann festgestellt werden, ob die durch BPEL formulierte Kombination der Dienste die funktionalen Anforderungen erf�llt. Die Aussage, dass alle Tests erfolgreich sind, reicht alleine nicht aus, um die Qualit�t der Software zu bewerten. Unter anderem muss die Qualit�t der Tests dabei ber�cksichtigt werden. Dar�ber liefert \nobreak{BPELUnit} aber keine Information. Die Aufgabe dieser Masterarbeit schlie�t sich an dieser Stelle an. Dabei steht die Testabdeckung, als ein wichtiges Merkmal der Softwarequalit�t, im Focus. 

Als Erstes sollen die geeigneten Metriken f�r die Abdeckung der Tests eines BPEL-Prozesses gefunden werden. Es soll �berpr�ft werden, inwiefern die in der Softwaretechnik bekannten Ma�e (z.B. Zweig�berdeckung) sich f�r BPEL-Prozesse eignen. Bei Bedarf  sollen entweder die bekannten Metriken angepasst oder neue definiert werden. Im zweiten Schritt soll ein Konzept erarbeitet werden, mit dem die Messung der Testabdeckung realisiert werden kann. Um eine einfache Integration in das BPELUnit zu erm�glichen, muss die Architektur des Frameworks ber�cksichtigt werden. Danach soll das BPELUnit-Framework entsprechend erweitert und angepasst werden. Die anschlie�ende Anpassung der Clients soll die neue Funktionalit�t  dem Benutzer  zur Verf�gung stellen.

\section{Aufbau der Arbeit}
Anschlie�end wird das Konzept f�r die Umsetzung und Integration in das BPELUnit- Framework vorgestellt.  
\chapter{Grundlagen}
\section{SOA}
  Unternehmen sehen sich heute mit sehr dynamischen M�rkten konfrontiert und sind oft einem starken internationalen Wettbewerb ausgesetzt. Zu den Anforderungen solcher M�rkte geh�ren oft flexible Gesch�ftsprozesse. Die notwendige Anpassungsf�higkeit der Unternehmen und deren Gesch�ftsprozessen h�ngt meistens direkt von der vorhandenen IT-Infrastruktur ab. Das Ziel ist daf�r zu sorgen, dass die IT-Systemen an den Gesch�ftsprozessen ausgerichtet sind und sehr leicht bei Ver�nderungen angepasst werden k�nnen. SOA verspricht diese F�higkeit zur flexiblen und effektiven Unterst�tzung sich immer schneller �ndernder Gesch�ftszielen und -anforderungen. 
  
  SOA(Service Oriented Architecture) ist ein Ansatz zum Entwurf verteilter Systeme. Die Kernidee der SOA, die Unternehmensfunktionalit�t
als Menge voneinander unabh�ngiger Dienste zur Verf�gung
zu stellen, bildet die Basis f�r Integration und Dynamik \cite{THOMAS2005}. Eine m�gliche Definition f�r SOA:
\begin{Def}Eine serviceorientierte Architektur ist ein
Konzept f�r eine Systemarchitektur, in
welchem Funktionen in Form von wieder
verwendbaren, voneinander unabh�ngigen
und lose gekoppelten Services implementiert
werden. Services k�nnen unabh�ngig von
zugrunde liegenden Implementierungen �ber
Schnittstellen aufgerufen werden, deren
Spezifikationen �ffentlich sind. Serviceinteraktion
findet �ber eine daf�r vorgesehene
Kommunikationsinfrastruktur statt.
Quelle Wikipedia 23.01.2006
\end{Def} 
  Die Funktionen einer Anwendung sind als Service organisiert, die 
 beliebig verteilt sein k�nnen und lassen sich dynamisch zu Gesch�ftsprozessen verbinden. Die zugrunde liegenden technischen Plattformen der einzelnen Services spielen dabei keine Rolle. Zu betonen ist, dass SOA eine Systemarchitektur und keine Technologie beschreibt. 


  Die wichtigen Eigenschaften sind:
  
  
\begin{itemize}
	\item Service
	\item wiederverwendbar
	\item unabh�ngig
	\item lose gekoppelt
	\item Schnittstellen unabh�ngig von der Impementierung
\end{itemize}


Prim�rziel ist, die historisch gewachsene, heterogene Systemlandschaft effizient an �nderungen im Gesch�ftsprozess anpassen zu k�nnen. Im Einzelnen soll dadurch Software erstellt werden, die

    * einfach an neue Bed�rfnisse angepasst werden kann (Flexibilit�t)
    * wiederverwendbar ist
    * verteilt installiert werden kann
    * an Gesch�ftsprozesse angepasst ist.

Sekund�rziele sind:

    * Kostenvorteile durch schnelle Optimierung
    * h�here Produktivit�t der Softwareentwickler durch Wiederverwendung von Services
    * schnelle Reaktion auf Herausforderungen m�glich
    * mittelfristige Einsparungen
    * Reduzierung der Komplexit�t durch Aufbrechen monolithischer IT-Systeme
    * schrittweise Restrukturierung komplexer Anwendungssysteme.


\section{Web Services}
Web Services sind aktuell die wichtigste Realisierung einer SOA und werden in der Software Industrie anerkannt. Mittlerweile setzen viele gro�e Softwarekonzerne auf Web Services und haben ihre Produktstrategien entsprechend ausgerichtet.

Als Web-Service bezeichnet man im Allgemeinen eine Softwarekomponente, die
ihre Funktionalit�at �uber Standardinternetprotokolle zur Verf�ugung stellt.
W3C definiert die Web Services folgenderma�en: 
\begin{Def}
A software application identified by a URI, whose interfaces and bindings are capable of being
defined, described, and discovered as XML artifacts. A Web service supports direct interactions
with other software agents using XML-based messages exchanged via Internet-based protocols.(W3C)
\end{Def}

a.) Ein Web Service wird durch einen URI identifiziert.1
b.) Die Schnittstelle eines Web Services ist maschinenlesbar und wird durch
WSDL (siehe n�chsten Abschnitt) beschrieben.
c.) Ein Web Service kommuniziert mit anderen Softwarekomponenten durch
XML Nachrichten. Der Nachrichtenaustausch kann insbesondere mit Hilfe
von Internetprotokollen (z.B. HTTP oder SMTP) stattfinden.

W�hrend die erste Definition die Eigenschaften von Web Services beschreibt,  die zweite Definition auf die wichtigen Standards ein. 
\begin{Def}
A Web service is a software system designed to support interoperable machine-to-machine
interaction over a network. It has an interface described in a machine-processable format
(specifically WSDL). Other systems interact with the Web service in a manner prescribed by its
description using SOAP messages, typically conveyed using HTTP with an XML serialization in
conjunction with other Web-related standards (W3C 2004a).
\end{Def}

Gem�� der beiden Definitionen sind die Beschreibung von Schnittstellen und
der Nachrichtenaustausch wesentliche Aufgaben. SOAP, WSDL und UDDI sind die daf�r vorgesehen Standards und bilden den Kern der Web Services.
\begin{itemize}
	\item SOAP ein standardisiertes, XML-basiertes
Protokoll zum Verpacken von Nachrichten, die zwischen Applikationen ausgetauscht
werden. setzt SOAP
auf die Netzwerk- und Transportschichten auf. Es ist also irrelevant, welche
Transportmechanismen f�ur den eigentlichen Versand verwendet werden.Die gebr�auchlichste Form des Austausches von SOAP-Nachrichten ist die �Ubertragung
�uber HTTP.
\item UDDI
dient zur Lokalisierung und Ver�ffentlichung
von Web-Services im Internet
UDDI = Register f�r Diensteund ihre Beschreibungen+ Suchmethoden+ Publishingmethoden
UDDI-Daten enthalten Kontakt-Informationen, Listen von Business Services und Infos, wie einService via Protokoll angesprochen werden kann
Um einen Web-Service im Internet zu finden, ist ein Verzeichnisdienst notwendig.
F�ur diese Zwecke wurde der Universal Description, Discovery and Integration-
Standard geschaffen. UDDI bietet Standardfunktionen zum Klassifizieren,
Katalogisieren und Verwalten von Daten und Metadaten �uber Web-
Services, so da� diese einfach gefunden und verwendet werden k�onnen.
\item WSDL (Web Services Defnition Language) ist eine funktionale, XML-basierte Beschreibungssprache
f�r die Schnittstellen eines Web Services.(siehe n�chsten Abschnitt)
\end{itemize}

\begin{figure}[htbp]
	\centering
		\includegraphics{bilder/Webservice.png}
		\caption{Kontrollfluss eines BPEL Prozesses}
	\label{fig:ExamlpleBPELProzess}
\end{figure}

Bei Web Services handelt es sich nicht um eine bestimmte Technik, sondern um ein B�ndel von verschiedenen Standards und Spezifikationen.
Web Service Stack ...

Bild

Im n�chsten Abschnitt wird WSDL etwas genauer vorgestellt.

Ein Web Service ist ein durch einen URI eindeutige
identifizierte Softwareanwendung, deren Schnittstellen als
XML-Artefakte definiert, beschrieben und gefundden
werden k�nnen. Ein Web Service unterst�tzt die direkte
Interaktion mit anderen Softwareagenten durch XMLbasierte
Nachrichten, die �ber Internetprotokolle
ausgetauscht werden.W3C 
Web Services bieten eine auf Standards basierende
Technologie, um SOAs zu realisieren


\section{WSDL}
WSDL ist eine XML-basierte Sprache zur Beschreibung von Web-Services.
Sie beschreibt die Schnittstelle des Web-Services, nicht aber den Web-
Service selbst. Neben der Methodenspezifikation beinhaltet sie auch technische
Daten, wie die eigentliche Lage des Dienstes und an welche Transportprotokolle
er gebunden ist.

Obwohl das W3C WSDL 2.0 am 23 May 2007  ver�ffentlicht hat, wird in dieser Arbeit WSDL 1.1 verwendet.  
Der Grund hierf�r ist, dass die BPEL 2.0, um die es in dieser Arbeit in erster Linie geht, auch noch auf WSDL 1.1 aufsetzt.

Abbildung \ref{fig:wsdlStructur} zeigt die Strujtur eines WSDL-Dokuments.
Die Elemente in einem WSDL-Dokument lassen sich in abstrakte Definitionen
und konkrete technische Beschreibungen zu diesen Definitionen unterteilen. 
F�r die sprach- und plattformunabh�ngige Beschreibung
der Schnittstelle (Datentypen, Operationen und Nachrichten) sind die abstrakten Elemente zust�ndig: $<$types$>$-, $<$message$>$- und $<$portType$>$-Elemente.
Die konkreten technischen Details (wo die Services sich befinden und wie sie aufgerufen werden k�nnen) werden durch die Elemente $<$binding$>$ und $<$service$>$ definiert.  Die konkreten Definitionen referenzieren
dabei stets abstrakte Definitionen. Das bedeutet, da� zu einer abstrakten
Definition mehrere konkrete Implementierungen  existieren k�nnen. Damit besteht die M�glichkeit ein Service an mehreren Stellen verf�gbar zu machen und die Verwendung verschiedenen Kommunikations- und Transportprotokollen zu erm�glichen.
Die Elemente


\begin{figure}[htbp]
	\centering
		\includegraphics[width=0.42\textwidth]{bilder/wsdlStructur.pdf}
		\caption{WSDL Dokumentformat}
	\label{fig:wsdlStructur}
\end{figure}

\textbf{$<$definitions$>$} ist das Wurzelelement des WSDL-Dokuments. Es enth�lt den Namen des Services, Namespaces f�r den Service selbst und f�r die verwendeten Standards.

\textbf{$<$types$>$} enth�lt Definitionen f�r eigene Datentypen. Standardm��ig werden die Datentypen aus der XML-Schema-Spezifikation verwendet.Die Flexibilit�t der XML-Schemas, die sprachunabh�ngig und plattformneutral sind, erlaubt es in WSDL die Besonderheiten bestehender Programmiersprachen oder Datenaustauschstandards abzubilden


\textbf{$<$message$>$}  fasst die definierten Typen zu abstrakten unidirektionalen Nachrichten zusammen.

\textbf{$<$portType$>$} beschreibt die Schnittstelle des Services, die eine oder mehrere
Operationen zusammenfasst. Diese ordnen der Operation
die Ein- und Ausgabeparameter zu und legen damit das Nachrichtenaustauschmuster fest.

\textbf{$<$binding$>$} definiert die vom Web Service verwendeten Kommunikationsprotokolle und Nachrichten formate. Die meist benutzten Protokolle sind SOAP, HTTP GET/POST. 

\textbf{$<$service$>$} stellt aus einem oder mehreren ports einen Dienst zusammen.
Ein port ordnet dem Binding einen bestimmten Endpunkt zu. Die genaue Definition h�ngt vom verwendeten Transportprotokoll ab.





\section{Komposition der Services mit BPEL}
   !!!!!!!The ActiveBPEL(tm) Engine 3.0 Final Release provides comprehensive support for the forthcoming WS-BPEL 2.0 Standard, which will be officially published early in 2007. This release also retains side-by-side execution compatibility with processes based on BPEL4WS 1.1, allowing phased migration to WS-BPEL 2.0. Additional features include enhanced message routing based on WS-Addressing and BPEL sub-process execution.


  Neues Standard in Web Service Komposition. Service Komposition ist rekursiv. 
  effiziente und flexible Modellierung von Bussines Processen
  
\begin{itemize}
	\item Choreography
	\item Orchestration
\end{itemize}

Definition BPEL


\begin{itemize}
	\item executable business processes
	\item Abstarct business processes
\end{itemize}


Partners

BPEL ist selbst Web Service und muss mit WSDL beschrieben werden.

Partner werden Web Services genannt die mit BPEL ineragieren. (rufen auf, werden aufgerufen oder beides).
BPEL unterscheidet nicht zwichen client und WS die er aufruft. BPEL kann zu Beispiel den client aufrufen bei asynchronem collback.

Partner Link Type definiert diese Beziehungen (Rollen)

Im Rahme dieser Arbeit ist der Kontrolflu� innerhalb des BPEL Prozesses 


\paragraph{Strukturierende Aktivit�ten}

Nur Sequence erlaubt beliebige Anzahl (>0) von Aktivit�ten direkt im Container zu platzieren, ohne weitere strukturierende Aktivit�ten zu verschachteln.


\section{Testen von BPEL-Kompositionen}\label{sec:framework}
Eine der wichtigsten Aufgaben bei der
Entwicklung von komplexer, umfangreicher Software ist die Pr�fung. In
der Praxis wird daf�r ein erheblicher Anteil des
gesamten Entwicklungsaufwands aufgebracht. Es
existieren zahlreiche Methoden und Techniken f�r die
Pr�fung von Software. Testen ist ein in der konventionellen Softwareentwicklung unentbehrliches Mittel zur Qualit�tssicherung.

\textit{Testen ist ein Prozess, ein Programm mit der Absicht
auszuf�hren, Fehler zu finden.}

Beim Testen von BPEL-Prozessen m�ssen im Vergleich zu anderen Anwendungen eine Besonderheit beachtet werden: externe Abh�ngigkeit in Form von Web Services. Die Durchf�hrung eines kompletten Tests ist nur dann m�glich, wenn alle beteiligten Web Services erreichbar sind und reproduzierbare Ergebnisse liefern. 
Zum richtigen Problem wird das Testen, wenn externe Service-Anbietern eingebunden sind. In der Regel hat man in dieser Situation nur die Beschreibung der Schnittstelle aber kein Quellcode. Aus diesem Grund kann ein Web Service in einer Testumgebung repliziert werden. 
Ein weiteres Problem ist das Testen von Fehlersituationen. Daf�r m�ssen die solche Situationen, wie "`Web Service nicht erreichbar"' oder "`antwortet mit einer Fehlermeldung"' simuliert werden. \cite{LuebketestSOA} 

Die angestrebte Isolation des BPEL-Prozesses f�r das Testen kann mit Hilfe eines Unit-Tests erreicht werden. Im Rahmen einer Masterarbeit \cite{Mayer2006}  ist das BPELUnit-Framework entstanden, der das Testen von BPEL-Kompositionen in isolation erm�glicht.

\subsection{BPELUnit-Framework}
BPELUnit ist ein Framework zum Testen von BPEL-Kompositionen in Isolation. Es erm�glicht wiederholbare \textit{white-box}-Tests zu erstellen, zu speichern und auszuf�hren. 
Das Framework richtet sich prim�r an die Entwickler, die das Testen und Programmieren w�hrend des Entwicklungsprozesses abwechseln machen. 
Dementsprechend bietet das  Framework eine schnelle und einfache M�glichkeit, die erstellten Tests immer wieder auszuf�hren und Ergebnisse auszuwerten.
Der Kern ist unabh�ngig von der Pr�sentationsschicht implementiert und bietet eine Schnittstelle f�r die Implementierung der Clients an. Drei Clients geh�ren geh�ren zum Framework:
\begin{itemize}
	\item Befehlszeilen-Client 
	\item ANT-Integration 
		\item Eclipse-Plugin 
\end{itemize}
 Bei den ersten beiden Clients erfolgt die Konfiguration �ber XML-Dokumente, die im Konfigurationsverzechnis (\textit{conf}) der BPELUnit-Installation abgelegt sind. Eclipse-Plugin erm�glicht neben der grafischen Darstellung des Ergebnisses auch komfortable Erstellung der Tests mit Hilfe eines Editors.
 
 Das Framework unterst�tzt \textit{ActiveBPEL Engine} und \textit{Oracle BPEL Process Manager (Server)}. Weitere Engines k�nnen durch implementieren entsprechenden Adapter integriert werden.

BPELUnit isoliert den zu testenden Prozess, indem es die Partner des Prozesses durch Mock's ersetzt, die diese Partner simulieren. Dadurch k�nnen Abh�ngigkeiten von externen Services aufgel�st werden. Schnellere Tests ist der angenehme Nebeneffekt dieser Vorgehensweise, denn die Simulationen der Web Services meist schneller als reale Implementierungen sind. 

\textbf{BPELUnit-Architektur}. Das Framework basiert auf einer Schichtarchitektur, die in der Abbildung \ref{fig:schichtarchitektur} dargestellt ist. 
 Der erste von unten Schicht ist f�r die Testspezifikation zust�ndig und legt fest, wie die Testdaten und das Verhalten der Partner definiert werden. Die Tests werden in den Test Suiten organisiert, wof�r die \textit{Test Organization}-Schicht verantwortlich ist. Die Ausf�hrung der Tests wird in der dritten Schicht \textit{Test Execution} realisiert. \textit{Test Results}-Schicht sorgt f�r die Sammlung der Daten und Statistiken  w�hrend der Tests.  Im folgenden werden die einzelnen Schichten etwas n�her erl�utert.
 \begin{figure}[htbp]
	\centering
		\includegraphics[width=0.47\textwidth]{bilder/FrameworkLayer.png}
		\caption{Schichtarchitektur des BPELUnit-Frameworks \cite{Mayer2006}}
	\label{fig:schichtarchitektur}
\end{figure}
  
 
 \textbf{Testspezifikation}.
 Ein BPEL-Prozess hat einen Client und beliebige Anzahl von Partner. Die Partner bzw. die Web Services laufen parallel ab und interagieren dabei mit dem BPEL-Prozess. 
Das Verhalten der Partner (Mock's) w�hrend der Tests wird in den so genannten Testf�llen (\textit{eng. Testcases}) definiert. Zu diesem Zweck werden f�r jeden Testfall so genannten \textit{Tracks} definiert: einen \textit{Client-Track} und beliebig viele \textsl{Partner-Tracks}. Die Tracks legen die Interaktionen der jeweiligen Partner mit dem BPEL-Prozess anhand von Sequenzen der atomaren Interaktionen (Aktivit�ten) fest..
Dabei werden folgende Interaktionsmuster unterst�tzt: 
\begin{itemize}
	\item \textit{one way}-Interaktion: \textit{receive} oder \textit{reply},
	\item synchrone \textit{two way}-Interaktion: \textit{receive-reply} und \textit{reply-receive},
	\item asynchrone \textit{two way}-Interaktion: \textit{receive-reply} und \textit{reply-receive}.
\end{itemize}


 \begin{figure}[htbp]
	\centering
		\includegraphics[width=0.8\textwidth]{bilder/TestcaseSequence.pdf}
		\caption{BPELUnit - Sequenzen in einem Testfall \cite{Mayer2006}}
	\label{fig:sequenzen}
\end{figure}

Die Abbildung \ref{fig:sequenzen} zeigt das Zusammenspiel der Sequenzen eines Testfalls.

Durch Daten, die an den BPEL-Prozess verschickt werden, kann der Prozess gesteuert und bestimmte Zweige des Kontrollflusses stimuliert werden. Das Kenntnis der inneren Struktur ist vorausgesetzt (white-box-Test).
 Die Daten, die bei der Interaktion mit BPEL-Prozess verschickt werden, sind in XML spezifiziert. Es k�nnen WSDL-Nachrichten und sowohl einfache Datentypen als auch Elementen der XML Schema verwendet werden. Die Daten, die der BPEL-Prozess verschickt, k�nnen auf die Korrektheit �berpr�ft werden. 
F�r die Auswahl der relevanten Daten wird XPath \cite{W3C1999} verwendet. 

Mit der F�higkeit des Frameworks, die SOAP Faults zu  empfangen und zu senden, steht dem Tester die M�glichkeit zur Verf�gung, die Fault und Compensation Handler des BPEL-Prozesses zu testen.  
 
 
\textbf{Testorganisation}.  
Die einzelnen Testf�lle werden wie in den konventionellen Programmiersprachen in einer \textit{suite} organisiert. \textit{Suite} referenziert alle f�r den Test notwendigen Elemente: zu testende BPEL-Komposition, WSDL-Beschreibungen, XML-Schemata usw. Au�erdem definiert die Suite f�r alle enthaltenen Testf�lle eine globale Testumgebung mit einer \textit{setup} und \textit{shutdown}-Routine, die f�r das \textit{Deployment} und \textit{Undeployment} des Prozesses zust�ndig ist. 

\textbf{Ausf�hrung der Tests}. Zum Testen muss der BPEL-Prozess in einer Testumgebung ausgef�hrt werden, die die Input- und Output-Daten des BPEL-Prozesses entsprechend der Testspezifikation behandelt. \textit{Real-life}-Testen hei�t der Ansatz, der bei BPELUnit-Framework daf�r verwendet wird. Dabei wird der zu testende BPEL-Prozess in eine BPEL-Engine deployt und als Web Service aufgerufen. Die Partner-Web Services m�ssen dabei durch Mock's ersetzt werden, die das Verhalten der Web Services simulieren, d.h., Web Service-Aufrufe empfangen und/oder selbst absetzen.   Beim \textit{Deployen} des BPEL-Prozesses m�ssen alle URI's der Partner-Web Services durch die URI's der Mock's ersetzt werden. 

 \begin{figure}[htbp]
	\centering
		\includegraphics[width=0.75\textwidth]{bilder/BPLEUnitTestHarness.png}
		\caption{BPELUnit - Testumgebung \cite{Mayer2006}}
	\label{fig:loggingservice}
\end{figure} 

Die Testumgebung ist neben der Simulation der Partner auch f�r die Kommunikation mit dem BPEL-Prozess verantwortlich, die �ber SOAP-Nachrichten stattfinden. Zu diesem Zweck simuliert BPELUnit den kompletten Web Service Stack und kommt mit allen wichtigen SOAP-Konstrukten zu recht. Das Framework erledigt sowohl das Verpacken der Daten in SOAP-Nachrichten als auch dekodieren der empfangenen SOAP-Nachrichten (Abbildung \ref{fig:bpelmiddleware}) und �bernimmt damit dem Tester viel Arbeit.
  
  
 \begin{figure}[htbp]
	\centering
		\includegraphics[width=0.98\textwidth]{bilder/BPELUnitMiddleware.pdf}
		\caption{Kommunikation zwischen BPEL-Prozess und simulierten Partnern \cite{Mayer2006}}
	\label{fig:bpelmiddleware}
\end{figure}


F�r das \textit{Deployment} eines BPEL-Prozesses sind anbieterspezifische \textit{Deployment Descriptoren} erforderlich, die nicht standardisiert sind. Das \textit{Deployen} des BPEL-Prozesses ist, wie bereits erl�utert notwendig, und wird durch BPELUnit automatisch ausgef�hrt. Dazu muss f�r die jeweilige Engine ein entsprechendes Adapter implementiert werden.


\textbf{Testergebnisse}.
Beim Testen wird der BPEL-Prozess in einer Engine ausgef�hrt. Der beschr�nkte Zugriff auf den Prozess (bedingt durch die Ausf�hrung in einer BPEL-Engine) w�hrend der Tests erschwert die Sammlung der Information �ber die Ausf�hrung wesentlich. BPELUnit kann nur berichten, ob der Testfall erfolgreich war oder nicht. Im Fehlerfall 
werden zwei Typen von Fehlern unterschieden:
\begin{itemize}
	\item Fehler auf der Anwendungsebene - zum Beispiel eine erwartete Nachricht von BPEL ist nicht angekommen, oder enth�lt falsche Informationen oder zu viele Nachrichten sind angekommen,
	\item Fehler auf der Kommunikationsebene - Problem mit dem Web Service Stack.
\end{itemize}

Eine wichtige Information beim Testen ist die erreichte Testabdeckung. Das BPELUnit-Framework wird im praktischen Teil mit der Funktionalit�t der Ermittlung von  Testabdeckung erweitert.
 
  

\section{Testabdeckung als Ma� f�r die Testqualit�t}
Um die Softwarequalit�t anhand der Test bewerten zu k�nnen, m�ssen zus�tzliche Informationen �ber die Tests bekannt sein. Wichtig sind unter anderem die Anforderungs-, Code- und Datenabdeckung der Tests.
\begin{itemize}
	\item \textbf{Anforderungsabdeckung}(\textit{eng. requirements coverage}) ist eine funktionale
Abdeckung auf Basis einer Spezifikation, die angibt, inwieweit die Testf�lle alle Anforderungen abdecken. Diese Metrik wird meist im Zusammenhang mit \textit{Black-Box-Tests} verwendet. 
	\item \textbf{Codeabdeckung}(\textit{eng. code coverage}) ist eine Metrik f�r das Verh�ltnis zwischen
zu testenden und tats�chlich getesteten Elementen des Pr�flings. Die Tests, die auf der Codeabdeckung basieren, geh�ren zur Gruppe der strukturorientierten Testtechniken und zielen darauf ab, die innere Struktur der Software zu testen (\textit{Glass-Box-Test}). 
\item \textbf{Datenabdeckung}(\textit{eng.data coverage})  ist die Metrik, die angibt wie und wann und in welcher Weise die Variablen verwendet werden (\textit{Glass-Box-Test}).
\end{itemize}
Die Anforderungs-, Code- und Datenabdeckung sind Mittel zur Messung der Qualit�t bzw. Voll"-st�n"-dig"-keit der Tests. Die beiden Testarten (\textit{Black-Box} und \textit{Glass-Box}) erg�nzen sich gegenseitig und werden in der Regel kombiniert angewendet. Da f�r diese Arbeit nur die Codeabdeckung relevant ist, geht es im weiteren Verlauf des Dokuments nur um diesen Aspekt. Der Begriff \textit{Testabdeckung} wird ausschlie�lich als Synonym zu diesem Begriff verwendet. 

Die Unit-Tests werden verwendet, um sicherzustellen, dass der erstellte Code korrekt funktioniert. Jedoch lassen sich dabei nur Aussagen �ber die Codebereiche machen, die im Test aufgerufen oder, anders gesagt, durch den Test abgedeckt wurden. Das zugrundeliegende Prinzip lautet: ist der Code abgedeckt, so ist die Wahrscheinlichkeit h�her, dass es keinen Fehler gibt. Das Ausf�hren aller potentiell fehlerhaften Stellen des Programms ist ein notwendiges Kriterium f�r das Finden der Fehler im Code. Deswegen ist die Angabe der erreichten Codeabdeckung aus der Sicht der Qualit�tssicherung nicht nur sinnvoll sondern unabdingbar.

Bei der Analyse der Codeabdeckung geht es also darum festzustellen, wie gut die Codebereiche durch die Tests abgedeckt sind bzw. welche Bereiche gar nicht oder nicht ausreichend abgedeckt sind. Wenn ein Testentwickler auf diese Information st�ndig und einfach zugreifen kann, dann hat er die M�glichkeit, neue Tests gezielt zu erstellen. Dadurch kann die Qualit�t der Tests schrittweise erh�ht und die Erstellung der �berfl�ssigen Tests vermieden werden. Mit Hilfe der Codeabdeckung kann also ein Optimum zwischen dem wirtschaftlichen Aufwand und der Testqualit�t gefunden werden, was bei der Herstellung der Software auch ein wichtiger Aspekt ist. 

In der folgenden Aussage werden die drei Begriffe \textit{Codeabdeckungsanalyse}, \textit{Testqualit�t} und \textit{Systemqualit�t} in eine Relation zueinander gestellt:
\begin{quotation}
	\textit{W�hrend Unit-Tests dazu dienen, die korrekte
	Funktionsweise der erstellten Software zu verifizieren und die Qualit�t des Gesamtsystems zu steigern, dient die Analyse der Codeabdeckung dazu, die Qualit�t der Tests sicherzustellen und zu erh�hen \cite{Wang2006}}.
\end{quotation}
Demzufolge h�ngt die Codeabdeckung unmittelbar mit der Qualit�t der Tests und indirekten mit der Qualit�t des Gesamtsystems.

 Die bekanntesten Abdeckungsmetriken werden im n�chsten Abschnitt beschrieben.
\subsection{Abdeckungsmetriken }\label{Abdeckungsmetriken}
Es gibt eine Vielzahl von unterschiedlichsten Abdeckungsmetriken. Eine gute �bersicht wird in \cite{Kaner1995} und 
\cite{Zhu1997} gegeben. In diesem Abschnitt werden nur die wichtigsten Abdeckungsmetriken vorgestellt, die f�r diese Arbeit relevant sind.

Als \textit{Testabdeckung} wird �blicherweise das Verh�ltnis zwischen
zu testenden und tats�chlich getesteten Elementen des Pr�flings bezeichnet. Es muss nicht immer 100\% gefordert werden. Je nach festgelegtem Ziel kann auch ein gewisser Anteil festgelegt werden. Die bekanntesten Kriterien sind 
Anweisungs-, Zweig-, Bedingungs- und Pfad�berdeckung.
\subsubsection{Anweisungsabdeckung}
Die Anweisungsabdeckung (\textit{eng. statement coverage}) ist die enfachse kontrollflu�orientierte Metrik.  
Sie wird wie folgt definiert: 
\[
	Anweisungsabdeckung=\frac{\text{\textit{Anzahl der ausgef�hrten Anweisungen}}}{\text{\textit{Anzahl der Anweisungen}}}
\]
Das Ziel der Anweisungs�berdeckung ist die mindestens einmalige Ausf�hrung aller Anweisungen des zu testenden Programms.
Bezogen auf den Kontrollflu�graph hei�t es, dass alle Knoten abgedeckt werden m�ssen. Es ist m�glich, mehrerer Anweisungen (z. B. Anweisungen innerhalb einer Schleife) zu einer Einheit
zusammenzufassen, die keinerlei Verzweigungen im Sinne von \textit{if-else}-Anweisungen enth�lt. zusammenzufassen. Wenn die 100\% Anweisungsabdeckung nicht erreicht werden kann, dann ist das ein Hinweis auf Programmcode, der m�glicherweise unter keinen Umst�nden ausgef�hrt werden kann. 

Anweisungsabdeckung ist ein schwaches Kriterium. Die Fehleridentifizierungsquote ist sehr niedrig. Der Grund daf�r ist, dass Anweisungsabdeckung nur die einzelnen Anweisungen betrachtet, ohne Zusammenh�nge zu ber�cksichtigen.
\subsubsection{Zweigabdeckung}\label{Zweigabdeckung}
Zweigabdeckung (\textit{eng. branch coverage}) gilt als minimales Kriterium f�r kontrollflu�orientierte Testtechniken.
\[
Zweigabdeckung=\frac{\text{\textit{Anzahl der durchlaufenen Zweige}}}{\text{\textit{Anzahl aller Zweige}}}
\]
Das hei�t, dass die hundertprozentige Zweigabdeckung dann erreicht ist, wenn alle Kanten des Kontrollflussgraphen mindestens einmal durchlaufen wurden. Erf�llte Zweigabdeckung stellt sicher, dass alle Verzweigungen des zu testenden Moduls tats�chlich erreichbar sind. Weil das Durchlaufen aller Zweige die Ausf�hrung aller Anweisungen verursacht, impliziert die Zweigabdeckung die Erf�llung der Anweisungsabdeckung. Mit der Zweigabdeckung k�nnen die nichtausf�hrbaren Programmzweige zus�tzlich aufgesp�rt werden. 

Obwohl die Fehlererkennungsquote gegen�ber der Anweisungsabdeckung steigt, werden nicht alle Aspekte des Kontrollflusses erfasst. Bei den Schleifen wird nur ermittelt, ob diese durchlaufen werden k�nnen.
Obwohl bei der vollst�ndigen Zweigabdeckung jede Entscheidung (komplexe Bedingungen), die Verzweigungen im Kontrollfluss ausl�sen, insgesamt die Wahrheitswerte $wahr$ und $falsch$ mindestens einmal annehmen muss, bleiben die Werte der einzelnen Bedingungen unber�cksichtigt. Au�erdem wird jeder Zweig f�r sich alleine betrachtet, ohne die Kombinationen von Zweigen zu ber�cksichtigen. 

Es gibt viele weitere Abdeckungsma�e, die in dieser Arbeit nicht betrachtet werden. 

\nocite{Liggesmeyer2002}
\nocite{Winkler2003}
\section{Zusammenfassung}
In diesem Kapitel wurden die allgemeinen Grundlagen behandelt, die f�r die Definition der Metriken und Ermittlung der Testabdeckung in BPEL-Kompositionen notwendig sind. Nach der Vorstellung der allgemeinen Definition der Serviceorientierten Architekturen wurden Web Services mit den zugeh�rigen Standards erl�utert. Etwas ausf�hrlicher wurde WSDL-Standard vorgestellt, der f�r die Beschreibung der Web Service-Schnittstelle eingesetzt und in dieser Arbeit verwendet wird. Anschlie�end wurde die BPEL-Sprache, die im Fokus dieser Arbeit steht, ausf�hrlich behandelt. Neben der BPEL-Aktivit�ten, die f�r die Realisierung der Gesch�ftslogik
verwendet werden, wurden das Link-Konzept, die Fehlerbehandlung und Kompensation vorgestellt. Die detaillierte Beschreibung der Konzepten der Konzepten der Sprache dient dem Verst�ndnis der formalen Definition der Metriken (Kapitel \ref{chap:testabdceckungInBPEL}) und der Umsetzung der Abdeckungsmessung (Kapitel \ref{chap:umsetzung}).

Mit der Testabdeckung in BPEL-Kompositionen behandelt diese Arbeit einen weiter gehenden Aspekt des Testen von BPEL-Prozessen. Dementsprechend wurden die Grundlagen des Testen von BPEL-Kompositionen ebenfalls in diesem Abschnitt behandelt. Insbesondere wurde dabei auf das BPELUnit-Framework eingegangen, das das Testen von BPEL-Kompositionen unterst�tzt und in dieser Arbeit durch die Testabdeckungsmetriken erweitert wird.

Zum Schluss wurde der Begriff Abdeckung erl�utert und zwei wichtigen Abdeckungsmetriken Anweisungs- und Zweigabdeckung vorgestellt.


\chapter{Testabdeckung in BPEL}\label{chap:testabdceckungInBPEL}
Durch die Berechnung der Testabdeckung will der Tester erfahren, wie viel von dem Pr�fling beim Testen ausgef�hrt wurde. F�r die Definition des Pr�flings, in diesem Fall BPEL-Prozesses, wird die BPEL Formalisierung von Ouyang in \cite{Chun2005} verwendet. Nach der Pr�sentation der wichtigsten Punkten dieser Formalisierung folgen die Definition der  Aktivit�t- und Zweigabdeckung. Dar�ber hinaus werden zwei speziell f�r BPEL definierten Metriken vorgestellt: Link- und Handler-Abdeckung.  
\section{Formale Definition von BPEL}\label{sec:bpelformal}
Die hier vorgestellte formale Beschreibung von WS-BPEL-Process Model ist nicht
vollst�ndig und fasst nicht den gesamten Umfang von BPEL um. Es werden nur
f�r diese Arbeit relevante Teile des Modells beschrieben. Als Grundlage dient
eine Definition aus  \cite{Chun2005}, die abgesehen von der Anpassung an den
Standard WS-BPEL 2.0 und einigen Erweiterungen, unver�ndert �bernommen wurde.  

Als Erstes wir eine Auswertungsfunktion definiert. F�r die Definition einer entsprechenden Metrik wird eine Auswertungsfunktion
definiert). Sei $f$ eine boolsche Funktion (oder ein boolscher Ausdruck),
$Var(f)$ enth�lt aussagenlogische Variablen, die in $f$ vorkommen. Sei $F$ die Menge
der boolschen Funktionen und $B$ die Menge (\textit{true}, \textit{false}), dann ist die Variablenzuordnung von $F$ ist die Abbildung $assign: Var(F)\rightarrow B$, und eine Menge aller m�glichen Variablenzuordnungen von $F$ ist $Assign(F)$. Die Auswertungsfunktion ist $eval: F \times Assign(F)\rightarrow B$.

\begin{Def}[WS-BPEL Process Model \cite{Chun2005}] Ein BPEL-Prozessmodell ist ein Tupel\\
$\mathcal{W=(A,\ E,\ C,\ L,\ HR,\ }type_{\mathcal{A}},\ type_{\mathcal{E}},\
instance,\ name,\ <_{seq},\ <_{if},\ serialscp,\ process,\\ trigger,\ scp_c,\
trigger_c,\ scp_t,\ trigger_{tf},\ \mathcal{LR},\ joincon,\ transitionCondition,\ supjoinf,\\ trigger_{jf})$ wobei: 
\end{Def}
\begin{itemize}\itemsep2pt
  \item $\mathcal{A}$ ist eine Menge von Aktivit�ten,
  \item $\mathcal{E}$ ist eine Menge von Ereignissen,
  \item $\mathcal{C}$ ist eine Menge von Bedingungen,
  \item $\mathcal{L}$ ist eine Menge von Links (innerhalb von Flow-Aktivit�ten),
  \item sei $\mathcal{B=E\cup C} \cup\{\bot\} $ eine Menge von Marken, wobei $\bot$ die leere Marke ist, dann ist $\mathcal{HR\subseteq A\times B\times A}$ ein annotierter Baum, der die Realaton zwischen einer Aktivit�t und ihren direkten Subaktivit�ten definiert,
	\item $\forall a\in \mathcal{A},\ let\ \mathcal{HR}_{p}=_{\pi 1,3}\mathcal{HR}$ (Projektion auf zwei Aktivit�tsmengen von $\mathcal{HR}$),\\
	$children(a)=\{a'\in \mathcal{A}|\mathcal{HR}_p(a,a')\}$ ist die Menge von unmittelbaren Nachfolgern von $a$,
	\item $type_\mathcal{A}:\mathcal{A\rightarrow T_A}$ ist Funktion, die den Aktivit�ten die Aktivit�tstypen zuordnet 
	
	\begin{itemize}\itemsep2pt
		\item $\mathcal{T_A=T_B\cup T_S}$
		\item $\mathcal{T_B}=\{receive,\ reply,\ wait,\ assign,\ validate,\ empty,\
		throw,\ rethrow,\\ \ compensate,\ compensateScope,\ exit\}$
 		\item $\mathcal{T_S}=\{sequence,\ flow,\ pick,\ if,\ while,\ repeatUntil,\
		forEach,\ scope\}$
    \end{itemize}
    
	\item $\forall t\in \mathcal{T_A},\ \mathcal{A}_t=\{a\in \mathcal{A}|type_\mathcal{A}(a)=t\}$ ist die Menge aller Aktivit�ten des Types $t$,
 	\item $type_\mathcal{E}:\mathcal{E\rightarrow T_E}$ ist die Funktion, die den Ereignissen die Typen aus der Menge $\mathcal{T_E}$  zuordnet, wobei $\mathcal{T_E}=\{message,\ alarm,\ fault,\ compensation,\ termination\}.$
 	\item $\forall t\in \mathcal{T_E},\ \mathcal{E}_t=\{e\in \mathcal{E}|type_\mathcal{E}(e)=t\}$ ist die Menge der Ereignissen des Types $t$,
%\item $instance:\mathcal{A}_{receive}\cup \mathcal{A}_{pick}\rightarrow \mathcal{B}$ is a %function which sddigns a boolean value to the createInstance attribete of a receive or a %pick activity.
\item sei $\mathcal{A}^{structured}=\mathcal{A}_{sequence}\cup \mathcal{A}_{flow}\cup \mathcal{A}_{if}\cup \mathcal{A}_{while}\cup \mathcal{A}_{repeatUntil}\cup \mathcal{A}_{forEach}\cup \mathcal{A}_{pick}\cup \mathcal{A}_{scope}$ ist die Menge der Strukturierten Aktivit�ten, $\forall_{s\in \mathcal{A}^{structured}}(children(s)\neq \emptyset)$, das hei�t, sie sind die internen Knoten des $\mathcal{HR}$ Baumes,
\item sei $\mathcal{A}^{basic}=\mathcal{A}_{invoke}\cup \mathcal{A}_{receive}\cup \mathcal{A}_{reply}\cup \mathcal{A}_{wait}\cup \mathcal{A}_{assign}\cup \mathcal{A}_{validate}\cup \mathcal{A}_{empty}\cup \mathcal{A}_{validate}\cup \mathcal{A}_{throw}\cup \mathcal{A}_{rethrow}\cup \mathcal{A}_{compensate}\cup \mathcal{A}_{compensateScope}$ ist die Menge der Basisaktivit�ten, $\forall_{s\in \mathcal{A}^{basic}}(children(s)= \emptyset)$, d.h., sie sind die Bl�tter des $\mathcal{HR}$ Baumes,
\item gegeben $\mathcal{A'}=\mathcal{A}_{sequence}\cup \mathcal{A}_{flow},\
\mathcal{HR} \cap (\mathcal{A}'\times B\times \mathcal{A})=\mathcal{HR} \cap
(\mathcal{A'}\times \{\bot\}\times \mathcal{A})$, diese $\mathcal{HR}$-Elemente repr�sentieren automatische Kontroll�bergabe von einer Aktivit�t an ihre Subaktivit�ten. repr�sentiert,
\item $\forall s\in \mathcal{A}_{sequence},\exists$ ist  $<^s_{seq}$ eine strenge totale Ordnung �ber $children(s)$,
\item $\mathcal{HR}\cap (\mathcal{A}_{pick}\times \mathcal{B}\times \mathcal{A})=\mathcal{HR}\cap (\mathcal{A}_{pick}\times \mathcal{E}^{normal}\times \mathcal{A})$, wobei $\mathcal{E}^{normal}=\mathcal{E}_{message}\cup \mathcal{E}_{alarm}$ eine Menge von normalen Ereignissen ist,
\item gegeben $\mathcal{A'}=\mathcal{A}_{if}\cup \mathcal{A}_{while}\cup
\mathcal{A}_{forEach},\ \mathcal{HR}\cap (\mathcal{A'}\times \mathcal{B}\times
\mathcal{A})=\mathcal{HR}\cap (\mathcal{A'}\times \mathcal{C}\times
\mathcal{A})$, so dass  \textit{if}-, \textit{while}- and \textit{forEach}-Aktivit�ten bei der �bergabe der Kontrolle an ihre Subaktivit�ten eine Bedingung auswerten m�ssen,
\item gegeben $\mathcal{A'}=\mathcal{A}_{repeatUntil},\ \mathcal{HR}\cap (\mathcal{A'}\times \mathcal{B}\times \mathcal{A})=\mathcal{HR}\cap (\mathcal{A'}\times \{\bot\}\times \mathcal{A}))$,
\item $\forall s\in \mathcal{A}_{if},\ \exists$ ist $<^{s}_{if}$ eine strenge totale Ordnung �ber $children(s)$,
\item $\forall s\in \mathcal{A}_{if}$, 
\begin{itemize}
	\item sei $last(s)\in children(s)$ die Subaktivit�t im letzten Zweig so, dass
	\item $\neg \exists_{a\in
children(s)}(last(s)<^{s}_{if}a)$, 
\item sei $c\in \mathcal{C}\mathcal{HR}(s,c,last(s))\Rightarrow \forall_{assign(c)\in
Assign(\mathcal{C})}eval(c,assign(c))=true$,
\end{itemize}
 $last(s)$ repr�sentiert den \textit{else}-Zweig in \textit{if}-Aktivit�t, was  sichert zum einem, dass ein \textit{else}-Zweig immer vorhanden ist und zum anderen mindestens ein Zweig in der Aktivit�t aktiviert wird,
\item $\forall s\in \mathcal{A}_{while},\ \left| \mathcal{HR}\cap (\{s\}\times \mathcal{C}\times \mathcal{A})\right|=1$, d.h., hat exakt eine Subaktivit�t,
\item $\mathcal{HR}\cap (\mathcal{A}_{scope}\times \mathcal{B}\times \mathcal{A})=\mathcal{HR}\cap (\mathcal{A}_{scope} \times (\mathcal{E}\times\{\bot \})\times \mathcal{A})$, wobei: $\forall s \in \mathcal{A}_{scope}$,
\begin{itemize}\itemsep2pt
	\item $\left|\mathcal{HR}\cap (\{s\}\times \{\bot\}\times \mathcal{A})\right|=1$, d.h., jeder \textit{Scope} hat genau eine Aktivit�t,
	\item $\left|\mathcal{HR}\cap (\{s\}\times \{\mathcal{E}_{fault}\}\times \mathcal{A})\right|\geq 1$, d.h., jeder Scope hat mindestens einen \textit{Fault Handler},
		\item $\left|\mathcal{HR}\cap (\{s\}\times \{\mathcal{E}_{compensation}\}\times \mathcal{A})\right|\leq 1$, d.h., jeder Scope hat maximal einen \textit{Compensation Handler},
\end{itemize}
\item $process\in \mathcal{A}_{scope}$ ist der Wurzel des $\mathcal{HR}$ Baumes ,
\item $trigger_{tf}:\mathcal{A}_{throw}\cup \mathcal{A}_{rethrow}\rightarrow \mathcal{E}_{fault}$ ist die Funktion, die jede \textit{throw}-Aktivit�t auf ein Fehlerereignis abbildet, das durch diese Aktivit�t ausgel�st wird,
\item $scp_c:\mathcal{E}_{compensation}\rightarrow \mathcal{A}_{scope}\backslash \{process\}$ ist injektive Funktion, die das \textit{Compensation}-Ereignis auf ein Scope so abbildet, dass das Auftreten dieses Ereignisses die Kompensation dieses Scopes initiiert,
\item $trigger_c:\mathcal{A}_{compensate}\rightarrow \mathcal{E}_{compensation}$ ist injektive Funktion, die jede \textit{compensate}-Aktivit�t auf ein Compensation-Ereignis abbildet, das durch dises Aktivit�t ausgel�st wird,
\item $\mathcal{LR}\subseteq \mathcal{A}\times \mathcal{L}\times \mathcal{A}$ ist annotierter azyklischer Graph, der die Relation zwischen \textit{source}- und \textit{sarget}-Aktivit�t eines Links beschreibt,
\item sei $\mathcal{A}^{source}=\{a\in \mathcal{A}|\exists_{l\in \mathcal{L}}((a,l)\in\pi_{1,2}\mathcal{LR})\}$ die Menge von \textit{source}-Aktivit�ten aller Links, und $\mathcal{A}^{target}=\{a\in \mathcal{A}|\exists_{l\in \mathcal{L}}((l,a)\in \pi_{2,3}\mathcal{LR})\}$  die Menge von \textit{target}-Aktivit�ten aller Links, dann $\forall a\in \mathcal{A}^{source},\mathcal{L}_{out}(a)=
\{l\in \mathcal{L}|\exists_{a'\in \mathcal{A}}\mathcal{LR}(a,l,a')\}$ ist die Menge aller ausgehenden von $a$ Links, und $\forall a\in \mathcal{A}^{target}, \mathcal{L}_{in}(a)=\{l\in \mathcal{L}|\exists_{a'\in \mathcal{A}}\mathcal{LR}(a',l,a)\}$ ist die Menge aller eingehenden bei $a$ Links,
\item sei $a\in \mathcal{A}^{target},\ joincon(a)$, die die \textit{join condition} von $a$ definiert, ist boolsche Funktion �ber $\mathcal{L}_{in}(a)$(d.h. $Var(joincon(a))=\mathcal{L}_{in}(a))$,
\item sei $l\in_{ \pi 2}\mathcal{LR},\ transitionCondition(l)$,  die die \textit{transition condition} des Links $l$ definiert, ist boolsche Funktion,
\item Dead-path-elimination (DPE). 
Wird eine Aktivit�t aufgrund einer zu \textit{false} ausgewerteten \textit{join condition} oder eines nicht abgearbeiteten Zweiges einer $if-$ oder $pick-$Aktivit�t nicht ausgef�hrt, so wird f�r alle ausgehenden Links die \textit{transition condition} auf \textit{false }gesetzt. Dieses Verhalten wird als \textit{Dead-Path-Elimination} bezeichnet. 
\end{itemize}

Betrachtet man die Kontrollstruktur eines BPEL-Prozesses, so kann die $\mathcal{HR}$-Relation als �bergabe der Kontrolle von einer Aktivit�t an ihre Subaktivit�ten interpretiert werden. Damit entspricht ein Element dieser Relation einem Zweig des Kontrollflussgraphen. Allerdings deckt diese Relation nicht alle Zweige des Graphen ab, es fehlen n�mlich die Zweige, die die Kontrolle innerhalb der Schleifen von der inneren Aktivit�t an die Schleife zur�ckgeben. Die folgende Relation $\mathcal{HBR}$ beschreibt genau solche Beziehungen:
\begin{itemize}\itemsep2pt
	\item $\mathcal{HBR}\subseteq \mathcal{A}\times(\mathcal{C}\cup
	\{\bot\})\times \mathcal{A}$ beschreibt die Relation zwischen einer Aktivit�t und ihrer \textit{Parent}-Aktivit�t
	\item $\forall a,a'\in \mathcal{A},\ \forall(a,a')\in_{\pi
	1,3}\mathcal{HBR}\Rightarrow a\in children(a')$ 
\end{itemize}

Folglich gilt f�r die Schleifen:
\begin{itemize}\itemsep2pt
	\item gegeben $\mathcal{A'}=\mathcal{A}_{while}\cup \mathcal{A}_{forEach},\
	\mathcal{HBR}\cap (\mathcal{A}\times \mathcal{B}\times
	\mathcal{A'})=\mathcal{HBR}\cap (\mathcal{A}\times \{\bot \}\times 
	\mathcal{A'}),$   
	\item gegeben $\mathcal{A'}=\mathcal{A}_{repeatUntil},\ \mathcal{HBR}\cap
	(\mathcal{A}\times \mathcal{B}\times \mathcal{A'})=\mathcal{HBR}\cap
	(\mathcal{A}\times \mathcal{C}\times \mathcal{A'}).$
\end{itemize}


Die Relationen $\mathcal{HR}$ und $\mathcal{HBR}$ beschreiben zusammen alle Kanten des Kontrollflussgraphen des zugeh�rigen BPEL-Prozesses.

\section{Definition der Abdeckungsmetriken f�r BPEL}\label{sec:metrikdefinition}
 In diesem Abschnitt werden die Metriken f�r die Messung der Testabdeckung in BPEL definiert. Dabei werden die in Abschnitt \ref{Abdeckungsmetriken} beschrieben Abdeckungsmetriken an BPEL-Kontext angepasst und neue spezifische Metriken eingef�hrt. Anschlie�end wird eine Umsetzung vorgestellt, auf dem Instrumentierungsverfahren basiert.

  \subsection{Statementabdeckung}
  Mit BPEL l�sst sich ein Prozess beschreiben, der in der Lage ist, verschiedene Dienste (Web SerServices) zu einer Gesamtanwendung zu verkn�pfen. Die Kommunikation mit den Diensten wird durch Basisaktivit�ten realisiert. Die Basisaktivit�ten sind die elementaren Aktivit�ten eines Prozesses, die keine weiteren Aktivit�ten enthalten. Es muss beim Testen zu mindestens sichergestellt werden, dass alle Basisaktivit�ten durch die Tests abgedeckt sind. 
  
  Insbesondere ist diese Statistik im Bezug auf einzelne Aktivit�ten interessant. So ist die Information, ob alle \textit{Invokes} (Web Service-Aufrufe) beim Testen ausgef�hrt wurden, sehr wichtig, um die erste Einsch�tzung �ber die Testqualit�t zu machen. In dieser Hinsicht ist die Definition der Abdeckungsmetriken f�r die einzelnen Basisaktivit�ten sinnvoll. Die Statementabdeckung\footnote{Die Aktivit�ten in BPEL entsprechen den Statements in anderen Programmiersprachen.} f�r die \textit{invoke}-Aktivit�ten wird wie folgt definiert: 
 \[
	Anweisungsabdeckung_{invoke}=\frac{\text{Anzahl der ausgef�hrten \textit{invoke}-Aktivit�ten}}{\text{Anzahl der \textit{invoke}-Aktivit�ten im Prozess}}
\]
Entsprechend sind die Abdeckungsmetriken f�r alle anderen Basisaktivit�ten definiert.  

Die gesamte Statementabdeckung wird in dieser Arbeit auf die Messung der Basisaktivit�ten eingeschr�nkt:  
\[
	Anweisungsabdeckung=\frac{\text{Anzahl der ausgef�hrten Basisaktivit�ten}}{\text{Anzahl aller Basisaktivit�ten im Prozess}}
\] 

Die  Abdeckung der strukturierten Aktivit�ten, die den Kontrollfluss des BPEL-Prozesses definieren, wird im Zusammenhang mit der Zweigabdeckung in folgenden Abschnitten behandelt.

Die Information �ber tats�chliche Ausf�hrung der Aktivit�ten hat einen Mehrwert gegen�ber der Information, dass eine Aktivit�t zur Ausf�hrung stimuliert wurde (aber evtl. nicht erfolgreich ausgef�hrt werden konnte). Deswegen erfolgt das Logging jeweils \textit{nach} der Ausf�hrung der entsprechenden Aktivit�t.  Die Aktivit�ten, die den normalen Kontrollfluss �ndern oder den Prozess beenden, k�nnen allerdings auf diese Weise nicht bzw. nur sehr umst�ndlich registriert werden:  
\begin{itemize}
	\item \textit{throw}
	\item \textit{rethrow}
	\item \textit{compensate}
	\item \textit{compensateScope}
	\item \textit{exit}
\end{itemize}
\textit{Exit} beendet den Prozess und kann logischerweise nur vor der Ausf�hrung erfasst werden. Die \textit{throw-}, \textit{rethrow-}, \textit{compensate-} und \textit{compensateScope-}Aktivit�ten unterbrechen den normalen Kontrollfluss und veranlassen die Ausf�hrung der entsprechenden Handler. Obwohl die tats�chliche Ausf�hrung dieser Aktivit�ten in diesen H�ndlern registriert werden k�nnte, wurde auf Grund eines gro�en zus�tzlichen Aufwands und dazu verh�ltnism��ig kleines Informationsgewinns entschieden, darauf zu verzichten und die Aktivit�ten, wie bei \textit{Exit}, vor der Ausf�hrung zu loggen. 
Aus diesem Grund werden die entsprechenden Aktivit�ten auch dann geloggt (als ausgef�hrt markiert), wenn bei der Ausf�hrung ein Fehler auftreten sollte.

Die Statementabdeckung ist ein schwaches Kriterium, weil dabei nur die einzelnen Anweisungen betrachtet, ohne Zusammenh�nge zu ber�cksichtigen. 


Statement Ziel eines Links Also sequence drum

\subsection{Zweigabdeckung}
Die Definition der Zweigabdeckung aus dem Abschnitt \ref{Zweigabdeckung} wird f�r BPEL �bernommen. Der Kontrollflussgraph muss dagegen an BPEL angepasst werden.
\subsubsection{Normaler Kontrollfluss}
Den Kontrollfluss bestimmen in BPEL, wie bereits mehrmals erw�hnt, die strukturierten Aktivit�ten. In den folgenden Bildern sind die entsprechenden Kontrollflussgraphen abgebildet. Die grauen Elementen deuten an, wie die einzelnen Graphen in den Gesamtfluss eingebunden werden. F�r die Zweigabdeckung relevanten Kanten sind schwarz dargestellt.
\begin{figure}[htbp!]
	\centering
		\includegraphics[width=0.65\textwidth]{bilder/h1.png}
	\label{fig:h1}
\end{figure}
\begin{figure}[htbp!]
	\centering
		\includegraphics[width=0.65\textwidth]{bilder/h2.png}
	\label{fig:h2}
\end{figure}

Bei den Aktivit�ten, deren Graphen in der Abbildung vorhanden sind, entspricht der Kontrollfluss den Konstrukten, die aus vielen Programmiersprachen bekannt sind. Die Restlichen sind BPEL-spezifisch und m�ssen gesondert betrachtet werden.


\parpic(6.5cm,5cm)[r]{\includegraphics[width=0.30\textwidth]{bilder/pick.png}} 
\textbf{\textit{Pick}-Aktivit�t.}\\
Bei dieser strukturierenden Aktivit�t wird ein Zweig aus mehreren anhand eines Ereignisses (Nachricht oder Alarm) ausgew�hlt. Wurde ein zutreffendes Ereignis empfangen, so wird die zugeh�rige Aktivit�t ausgef�hrt und alle nachfolgenden Ereignisse verworfen. Aus der Sicht des Kontrollflusses stimmt dieses Verhalten mit dem einer IF-\-Akti\-vi\-t�t �berein. Deswegen werden diese beiden Aktivit�ten im Kontrollfluss durch gleiche Graphen repr�sentiert.

\textbf{\textit{Flow}-Aktivit�t.}\\
  Mit der \textit{Flow}-Aktivit�t wird in BPEL paralleler Ablauf mehrerer Aktivit�ten realisiert. Au�erdem gibt es mit dem Link-Konzept, ein m�chtiges Instrument, das erm�glicht, Syn\-chro\-ni\-sa\-tions\-ab\-h�n\-gig\-keiten aufzustellen und damit komplexe Abl�ufe zu realisieren. Die rechte Abbildung zeigt den zugeh�rigen Kontrollfluss. 
  \parpic(6.4cm,3.6cm)[r]{\includegraphics[width=0.33\textwidth]{bilder/flow.png}}
  Die Links (gestrichelte Pfeile) unterscheiden sich semantisch von den normalen Kanten des Kontrollflussgraphes.
  Wie es bereits im Abschnitt .. geschildert wurde, haben die Links trotzdem einen erheblichen Einfluss auf den Kontrollfluss und sind damit f�r die Zweigabdeckung relevant. 

  Aus der Sicht der Abdeckungsmessung ist die Tatsache besonders, dass der Status der Links drei Werte annehmen kann: \textit{true}, \textit{false} und \textit{unset}  Die normalen Kontrollflusskanten k�nnen dagegen nur zwei Werte annehmen: entweder \textit{aktiviert} oder \textit{nicht aktiviert}(\textit{unset}). F�r die Links bedeutet das, dass \textit{true}- und \textit{false}-Status bei der Messung der Abdeckung erfasst werden m�ssen. 

\textbf{\textit{ForEach}-Aktivit�t.}
Die \textit{ForEach}-Aktivit�t sorgt f�r mehrmaliges Ausf�hren des enthaltenen Scopes. Die Anzahl der Ausf�hrungen wird durch Start- und Stopvariablen vorgegeben. Die Besonderheit dabei ist, dass die Ausf�hrung entweder sequenziell oder parallel erfolgen kann. Die Ausf�hrungsart wird durch den  Wert der Variable \textit{parallel} (\textit{yes}/\textit{no}) geregelt. 

Dieses Verhalten kann durch zwei Graphen dargestellt werden. Die Abbildung ... repr�sentiert sequentielle Ausf�hrung, die einer for-Schleife aus den konventionellen Programmiersprachen entspricht. Der Graph daneben repr�sentiert eine parallele Ausf�hrung.
\begin{figure}[h!]
	\centering
		\includegraphics[width=0.73\textwidth]{bilder/forEeach2.png}
	\label{fig:forEeach2}
\end{figure}\\

W�hrend der regul�re Kontrollfluss des Programms den Steuerfluss f�r alle regul�re Situationen vorgibt, gibt es in BPEL ein (\textit{FaultHandler})-Kontrollfluss, der f�r die Behandlung zur Laufzeit aufgetretener Fehler vorgesehen ist. Inwiefern dieses Verhalten in die Zweigabdeckung mit einflie�en kann, wird im n�chsten Abschnitt diskutiert. 

\subsubsection{Kontrollfluss bei Ausnahmebehandlung}
BPEL-Sprache hat wie viele moderne Programmiersprachen ein Konzept zur strukturierten Behandlung von Laufzeitfehlern. 
Die Umschaltung von normalen auf \textit{FaultHandler-}Kontrollfluss erfolgt automatisch beim Auftreten eines Fehlers. Was nichts anderes hei�t, als dass der zugeh�rige \textit{FaultHandler} ausgef�hrt wird. Die Fehler k�nnen auch in den \textit{Compensation} und in den \textit{FaultHandler} selbst auftreten. In die Abbildung \ref{fig:ScopeFaultCompensation2} wird der Zusammenhang zwischen \textit{Handler} und \textit{Scopes} graphisch dargestellt und durch die Pfeile den m�glichen Kontrollfluss zwischen diesen Einheiten gezeigt. 
\begin{figure}[h!]
	\centering
		\includegraphics[width=0.35\textwidth]{bilder/ScopeFaultCompensation2.png}
	\caption{Zusammenhang zwischen Scopes, Fault und Compensation Handler}
	\label{fig:ScopeFaultCompensation2}
\end{figure}

Vor allem die vielf�ltige Fehlerquellen und die M�glichkeit die Fehler weiterzuleiten machen es schwierig alle m�glichen �bergaben der Aus\-f�h\-rungs\-kon\-tro\-lle an die FaultHandler zu ber�cksichtigen. Aus diesem Grund werden die Verbindungskanten zwischen den Scopes, Fault und Compensation Handler in dieser Arbeit bei der Zweigabdeckung nicht ber�cksichtigt. 

Das Gleiche gilt f�r CompemsationHandler, die nur im Rahmen der Fehlerbehandlung ausgef�hrt werden.

Dass die Fault- und Compensation Handler, sowie die interne Logik dieser Handler, trotzdem durch die Tests abgedeckt werden sollten, ist unumstritten. Demzufolge ist es sinnvoll den Kontrollfluss \textit{innerhalb} der Handler bei der Messung der Abdeckung zu ber�cksichtigen. In der Abbildung .. wird der Kontrollfluss des gesamten Prozesses angedeutet. Die rote Farbe markiert den aktuellen Durchlauf. W�hrend alle durchgezogenen Kanten (auch in \textit{Fault}- und \textit{CompensationHandler}) f�r die Zweigabdeckung relevant sind, gehen die gestrichelten, wie bereits erl�utert, nicht in die Rechnung ein.  
\begin{figure}[htbp]
	\centering
		\includegraphics[width=0.9\textwidth]{bilder/Fault_Compensation.png}
	\label{fig:Fault_Compensation}
\end{figure}
 
 In den folgenden Abschnitten werden die Metriken vorgestellt, die den Grad der durch die Tests abgedeckten Fault- bzw. Compensation Handler wiedergeben.
\subsection{Fault und Compensation Handler Abdeckung}
Da die Gesch�ftsprozesse in der Regel langlebig sind und dar�ber hinaus sensitive
Daten verarbeiten k�nnen, ist eine ausreichende Fehlerbehandlung zwingend notwendig.
Im Rahmen einer Fehlerbehandlung ist oft das Zur�cksetzen vorangegangener �nderungen erw�nscht. Daf�r sind CompensationHandler in BPEL vorgesehen, die das R�ckg�ngig-machen von eigentlich in sich erfolgreichen Aktionen �bernehmen.

Aufgrund dieser besonderen Wichtigkeit der Fehlerbehandlung muss das Verhalten des Systems im Fehlerfall umfassend getestet werden. Die in diesem Abschnitt vorgestellten Metriken k�nnen dabei als Indikator f�r die Testqualit�t bez�glich der Fehlerbehandlung und Kompensation dienen.
 
\textbf{FaultHandler.}
Eine wichtige Information ist zum Beispiel, ob alle durch den Programmierer vogesehen Fehlerbehandlungen in Form von \textit{catch}- bzw \textit{catchAll}-Bl�cken durch die Tests stimuliert werden. 
Die folgende Definition legt die dazugeh�rige Metrik fest: 
\[
	FaultHandlerAbdeckung=\frac{\text{Anzahl der getesten \textit{catch}- und \textit{catchAll}-Bl�cken }}{\text{Anzahl der \textit{catch}- und \textit{catchAll}-Bl�cken}}
\]

Die impliziten \textit{FaultHandler} werden nicht ber�cksichtigt. Die sogenannten \textit{Inline-FaultHandler}, die direkt in die \textit{invok}e-Aktivit�ten integriert sind, m�ssen dagegen in die Berechnung einflie�en.

\textbf{Compensation Handler.}
Diese Metrik gibt den Abdeckungsgrad der \textit{CompensationHandler} an: 
\[
	CompensateHandlerAbdeckung=\frac{\text{Anzahl der getesten \textit{CompensationHandler}}}{\text{Anzahl der \textit{CompensationHandler}}}
\]
Die \textit{Inline-CompensateHandler} werden ebenfalls ber�cksichtigt. \\
\\
\\


\section{Zusammenfassung}
In diesem Kapitel wurden nach der Vorstellung der formalen Beschreibung des BPEL-Prozessmodels  die Testabdeckungsmetriken f�r BPEL formal definiert:
\begin{itemize}
\item Aktivit�tsabdeckung: zeigt, wie viele Basisaktivit�ten bei den Tests ausgef�hrt wurden,
\item Zweigabdeckung: zeigt, wie viele Zweige bei den Tests aktiviert wurde,
\item Linkabdeckung: zeigt, wie viele Links bei den Tests ausgewertet wurden,
\item Fault und Compensation Handler-Abdeckung: zeigt, wie viele Handler bei den Tests
ausgel�st wurden.
\end{itemize}
Bei der Definition der Metriken wurde darauf geachtet, dass f�r jeden BPEL-Prozess 100\%-ge
Abdeckung erreichbar ist.
\chapter{Verfahren f�r die Messung der Testabdeckung}

In diesem Abschnitt werden einige Ans�tze f�r die Messung der Testabdeckung in BPEL-Prozessen betrachtet und bewertet. Zum Schluss wird eine geeignete L�sung f�r die Integration in BPELUnit-Framework ausgew�hlt.  


Die Verfahren, die in konventionellen Sprachen f�r die Messung der Abdeckung verwendet werden, k�nnen nicht direkt auf BPEL �bertragen werden. Es m�ssen einige BPEL-spezifische Restriktionen beachtet werden:
\begin{itemize}
\item BPEL-Prozesse werden in speziellen Ablaufumgebungen (\textit{BPEL-Engine}) ausgef�hrt.
\item BPEL-Prozesse k�nnen nur mit Web Services (\textit{partner}) kommunizieren.
\item Gesch�ftslogik wird in BPEL-Prozessen durch Aktivit�ten realisiert. 
\end{itemize} 


Es gibt mehrere M�glichkeiten die Informationen �ber Ablauf eines BPEL-Prozesses zu sammeln: 
\begin{itemize} 
	\item \textbf{Instrumentierung}. Der Quellcode wird vor dem dem Deployen und Ausf�hren modifiziert, indem bestimmte Aktivit�ten hinzugef�gt werden, die die Ausf�hrung des entsprechenden Codebereichen signalisiert. 
	\item \textbf{Tracing}. Der Ablauf des Programms wird �ber eine Debug-API der entsprechenden Ablaufumgebung  mitverfolgt.
	\item \textbf{Web Service-Mock's des BPELUnit-Frameworks}(\cite{Mayer2006}). W�hrend der Ausf�hrung k�nnen aufgerufene Mock-Partner (Web Services, die durch Mock's ersetzt wurden) k�nnen ihre Interaktionen mit dem BPEL-Prozess protokollieren. 
	\item \textbf{Log-Dateien der BPEL-Ausf�hrungsumgebung}(\cite{Mayer2006}). Aus den Log-Dateien der Ausf�hrungsumgebung kann Information �ber den Ablauf des BPEL-Prozesses extrahiert werden. 
\end{itemize}

    
Da die dritte und vierte M�glichkeit sich f�r die Messung der Abdeckung nicht besonders gut eignen und daher keine wirkliche Alternative darstellen, werden sie an dieser Stelle nur ganz kurz behandelt. Die Web Service-Mock's k�nnen die Interaktionen mit BPEL-Prozess protokollieren. Zum einen muss zus�tzliche Logik in die Mock's eingebaut werden. Der gr��te Nachteil, der diesen Ansatz f�r die Messung der Abdeckung unbrauchbar macht, ist aber, dass nur die Aktivit�ten dokumentiert werden k�nnen, die nach au�en kommunizieren. Die Log-Dateien der Ablaufumgebungen, wenn diese �berhaupt existieren, sind anbieterabh�ngig und nicht standardisiert, was diese M�glichkeit auch ausschlie�t. 

Der Instrumentierungs- und Tracingansatz sind als Verfahren f�r die Messung der Codeabdeckung aus dem Bereich der konventionellen Programmiersprachen bekannt. Die beiden Ans�tze wurde von P.Dul in \cite{Dul2005} im Bezug auf Programmiersprache Java detailliert untersucht und verglichen.
 
\paragraph{Instrumentierungsansatz.}
Bei der Instrumentierung handelt es sich um ein einfaches Verfahren zur Messung der Testabdeckung, das von vielen Werkzeugen eingesetzt wird. Dabei werden in den Quellcode zus�tzliche Anweisungen eingef�gt, die die Ausf�hrung bestimmter Codebereiche dokumentieren. Damit der Originalcode unver�ndert bleibt, wird die Instrumentierung auf einer Kopie durchgef�hrt. 

W�hrend der Instrumentierung werden die relevanten statistischen Daten des Quellcodes, die f�r die Auswertung der Ergebnisse notwendig sind, gesammelt und gespeichert. Anhand dieser Daten und der Information, die w�hrend der Ausf�hrung gesammelt wird, kann die Auswertung erfolgen und die Codeabdeckung ermittelt werden. Der Instrumentierungsansatz f�hrt zu einer erheblichen Vergr��erung des Programms.  

\paragraph{Tracingansatz.}
Die modernen Entwicklungsumgebungen verf�gen �ber Debugger, die unter anderem eine Tracing-Funktionalit�t mit sich mitbringen.
\begin{quotation}
	\textit{Tracing bezeichnet man in der Programmierung eine Funktion zur Analyse von Fehlersuche von Programmen.
Dabei wird z.B. bei jedem Einsprung in eine Funktion, sowie bei jedem Verlassen eine Meldung ausgegeben, sodass der Programmierer mitverfolgen kann, wann und von wo welche Funktion aufgerufen wird...Wikipedia}
\end{quotation}
Diese Funktionalit�t erlaubt den Ablauf des Programms zu verfolgen und kann zur Messung der Testabdeckung genutzt werden. Auch bei diesem Ansatz wird der Quellcode im Vorfeld analysiert, um alle relevanten statischen Informationen zu sammeln. 

Der Vergleich der beiden Ans�tze aus \cite{Dul2005} gilt im Wesentlichen auch f�r BPEL. Die Tabelle \ref{Vergleichstabelle} beruht auf seinen Ergebnissen und stellt die beiden Verfahren im BPEL-Context gegen�ber.
\begin{table}[h!]
\begin{tabular}{p{6cm}p{0.5cm}p{6cm}}
\textbf{Instrumentierungsansatz}&\ &\textbf{Tracingansatz}\\[0.1cm]
\hline \\
Basiert auf dem Hinzuf�gen von
Instrumentierungsanweisungen in
eine Kopie des zu untersuchenden
Quellcodes.&\ &
 Basiert auf den F�higkeiten des
Debuggers, Informationen
�ber das ablaufende
Programm zu erhalten.\\
\\
 Es besteht grunds�tzlich die
Gefahr den Programmablauf oder
die Programmlogik durch das
Hinzuf�gen von Instrumentierungs-
anweisungen zu
ver�ndern.&\ &
 Es besteht keine Gefahr, dass der
Programmablauf oder die Programmlogik
beim Tracingansatz
ver�ndert werden.\\
\\
 Es besteht grunds�tzlich die
Gefahr von Namenskonflikten
oder der Verletzung der Spezifikation.&\ &
 Es besteht keine Gefahr, dass
Namenskonflikte oder eine
Verletzung der Spezifikation
auftreten.\\
\\
 Overhead wird durch zus�tzliche
Aktivit�ten im Quellcode
generiert.&\ &
 Overhead wird durch die
Anwendung der Debug-API generiert.\\
\\
 Es ist eine aktivit�tsbasierte
Messung der Testabdeckung
m�glich.&\ &
 Es ist nur eine zeilenbasierte
Messung der Testabdeckung
m�glich.\\
\\
Die Messung bleibt unabh�ngig von 
Anbieter der Ablaufumgebung.&\ & 
Die Messung ist nur m�glich, wenn der Anbieter der Ablaufumgebung eine Tracing-Funktion unterst�tzt. Die L�sung ist anbieterspezifisch.
\end{tabular}
\caption{Vergleich der Ans�tze\cite{Dul2005}}
\label{Vergleichstabelle}
\end{table}

Zahlenm��ig weist der Tracingansatz weniger Nachteile auf. Jedoch fehlt es teilweise Unterst�tzung der BPEL-Ablaufumgebungen f�r Debugging bzw. es gibt keine Tracing-Funktionalit�t, die die Verfolgung des Ablaufs erm�glichen w�rde. Noch gr��eres Problem ist, dass diese Schnittstelle in keinster Weise standardisiert ist. Alle Nachteile des Instrumentierungsansatzes entstehen aus der Tatsache, dass die Manipulation des Quellcodes stattfindet. 

Bei der Fallstudie in der oben erw�hnten Masterarbeit wurde festgestellt, dass die Messung beim Tracingansatz wesentlich langsamer ist, als beim Instrumentierungsansatz. Diese Aussage hat aber f�r den BPEL-Prozess, der die Dienste �ber Netzwerk anspricht, keine G�ltigkeit. Der Grund daf�r ist der Netzwerkzugriff, der f�r die Kommunikation notwendig ist. Auch wenn BPELUnit-Framework die M�glichkeit anbietet, den BPEL-Prozess isoliert von den einzelnen Web Servives zu testen, werden die Mock's, die Web Services simulieren, immer noch �ber Netzwerkschicht angesprochen. Die Programmeinheit, die die Meldungen �ber Ausf�hrung bestimmter Codeteile empf�ngt,muss ebenfalls �ber Netzwerkschicht angesprochen werden (siehe Abschnitt). Au�erdem muss die Zeit f�r die Instrumentierung ber�cksichtigt werden. Im Abschnitt .. werden f�r die vorgestellten Beispiele gemessene Zeitdifferenzen angegeben. 


\paragraph{Auswahl des Ansatzes f�r die Realisierung der Abdeckungsmessung in BPEL}\ \\
Da nur die ersten beiden Ans�tze (Instrumentierungs- und Tracingansatz) die Realisierung der Abdeckungsmessung f�r alle definierten Abdeckungsmetriken (siehe Abschnitt \ref{metrikdefinition}) im n�tigen Umfang erm�glichen, reduziert sich die Auswahl auf diese Beiden.

Aufgrund der vorgesehenen Integration der L�sung in das BPELUnit Framework muss bei der Auswahl des Verfahrens auf die Kompatibilit�t zum Framework und den darunterliegenden Konzepten geachtet werden.  Eine der zentralen Ziele bei der Konzeption des BPELUnit Frameworks war die Anbieterunabh�ngigkeit. In dieser Hinsicht hat man sich entschieden, den Ansatz \textit{real-life Testen} zu verwenden und damit auf die Verwendung von Debug-API zu verzichten. Das Deployment ist zwar immer noch anbieterspezifisch, die Abh�ngigkeit wurde aber mit dieser Entscheidung auf einen Minimum reduziert. 
Der Einsatz der Tracing-Funktionalit�t f�r die Messung der Testabdeckung w�rde einen gro�en R�ckschritt in der Anbieterunabh�ngigkeit bedeuten. Die Verwendung des Frameworks mit BPEL-Ablaufumgebungen, die keine Traicing-Funktionalit�t unterst�tzen, w�re ausgeschlossen. Demzufolge eignet sich f�r die Messung der Abdeckung im BPELUnit Framework der Instrumentierungsansatz am Besten. 




\chapter{Umsetzung der Abdeckungsmessung}\label{chap:umsetzung}
In diesem Kapitel wird die Umsetzung der Abdeckungsmessung f�r BPEL mit dem Instrumentierungsansatz vorgestellt. Die Ermittlung der Testabdeckungsmetriken nach diesem Verfahren erfolgt, wie bereits im Abschnitt \ref{sec:instrumentierungsansatz} erl�utert, im Wesentlichen in drei Phasen:
\begin{itemize}
	\item Instrumentierung (in diesem Fall der BPEL-Datei),
	\item Sammlung der Daten w�hrend der Testausf�hrung,
	\item Auswertung der Daten und Berechnung der Metriken.
\end{itemize}

Die Instrumentierung ist die Kernaufgabe und wird im folgenden Abschnitt ausf�hrlich vorgestellt. Anschlie�end im Abschnitt \ref{sec:auswertung} werden die zwei anderen Phasen erl�utert.  
\section{Instrumentierung des BPEL-Prozesses}\label{sec:umsetzungInstr}
Instrumentierung erfolgt in zwei Schritten. Zuerst wird die BPEL-Datei mit speziellen Marken annotiert. Jede Marke repr�sentiert dabei ein f�r die jeweilige Metrik relevantes Element und wird in Form eines Kommentars in den Kontrollfluss eingebettet. Im zweiten Schritt wird daf�r gesorgt, dass diese Marken, sobald der Kontrollfluss sie bei der Ausf�hrung erreicht, an spezielle Einheit �bertragen werden. Diese verwendet die empfangenen Daten anschlie�end f�r die Berechnung der Metriken.

F�r die korrekte Ermittlung der Testabdeckung ist die richtige Platzierung der Marken entscheidend. Dieses Problem kann im ersten Schritt behandelt werden, ohne die BPEL-spezifischen Aspekte der Kommunikation ber�cksichtigen zu m�ssen, die erst im zweiten Schritt zum Tragen kommen.
\subsection{Annotation des Prozesses}
Bei der Instrumentierung einer BPEL-Datei sind mehreren Aufgaben zu erledigen:
\begin{itemize}
\item Analyse der BPEL-Datei,
\item Sammlung und Speicherung der statischen Daten (f�r die sp�tere Auswertung),
\item Annotation des BPEL-Prozesses mit Marken.
\end{itemize}
Die Ausf�hrung der Aufgaben erfolgt in drei Phasen, die sequentiell abgearbeitet werden. 

\textbf{Phase 1.} W�hrend der ersten Phase wird der BPEL-Prozess analysiert. Dabei werden f�r jede Metrik alle relevanten Elemente des Prozesse bestimmt und gespeichert. Dadurch wird gesichert, dass die Elemente, die w�hrend der Instrumentierung in den Prozess eingef�gt werden, bei der Ermittlung der Metriken unber�cksichtigt bleiben. Die relevanten Elemente k�nnen anhand der Definitionen der jeweiligen Metrik (siehe Abschnitt \ref{sec:metrikdefinition}) bestimmt werden:
\begin{itemize}
	\item Aktivit�tsabdeckung - alle Basisaktivit�ten,
	\item Zweigabdeckung - alle Strukturierten Aktivit�ten,
	\item Linkabdeckung - alle Links mit \textit{transition condition}, die nicht konstant \textit{true} oder \textit{false} ist,
	\item Fault Handler-Abdeckung - alle \textit{catch}- und \textit{catchAll}-Elemente,
	\item Compensation Handler-Abdeckung - alle Compensation Handler. 
\end{itemize}

\textbf{Phase 2.} Die zweite Phase wird ausf�hrlich in den folgenden vier Abschnitten f�r jede Metrik einzeln vorgestellt. An dieser Stelle werden lediglich einige Bemerkungen gemacht, die f�r jede Metrik gelten. 

In Phase 2 erfolgt die Annotation des Prozesses mit speziellen Marken. Jede Marke ist eindeutig identifizierbar und geh�rt zu einer bestimmten Metrik. Au�erdem identifiziert jede Marke ein Element des Kontrollflusses, das f�r die Abdeckungsmessung relevant ist. Liegen die Marken w�hrend des Tests auf dem Ausf�hrungspfad, so werden diese an eine Komponente �bertragen. Damit protokolliert jede Marke Stimulation eines bestimmten Kontrollflusselements. 
Alle eingef�gten Marken werden gespeichert und stehen f�r die Auswertung und Berechnung der Metriken zur Verf�gung. Bei der Instrumentierung ist zu beachten, dass es grunds�tzlich m�glich ist, die interne Programmlogik des Prozesses  durch die Manipulation zu ver�ndern. Die syntaktische Korrektheit darf nat�rlich auch nicht beeintr�chtigt werden.

In der zweiten Phase wird die BPEL-Datei soweit vorbereitet, dass in der darauf folgenden Phase 3 jede Marke lediglich durch Aktivit�ten ersetzt werden, welche die Marken an speziellen Web Service schicken.

Obwohl die Realisierung der Daten�bertragung erst im zweiten Schritt erfolgt, wird zur Vereinfachung bereits in diesem Abschnitt statt einer Marke auch von \textit{Logging} oder \textit{Logging-Aktivit�ten} gesprochen.
Grafisch werden die Marken ebenfalls bereits als \textit{invoke}-Aktivit�ten\footnote{Eigentlich geh�rt zu jeder \textit{invoke}-Aktivit�t eine \textit{assign}-Aktivit�t, die die Marken an eine Variable zuordnet. F�r bessere �bersichtlichkeit werden die zugeh�rigen \textit{assign}-Aktivit�ten aber nicht dargestellt.} dargestellt. 
Die Komponente zum Empfangen von Marken wird \textit{Coverage Logging Service} genannt. 

Zu erw�hnen ist noch, dass die Metriken in beliebigen Reihenfolge abgearbeitet werden k�nnen, weil alle ben�tigten Elemente des Originalprozesses f�r alle Metriken in der ersten Phase gespeichert wurden. 


\textbf{Phase 3.} In dieser Phase werden BPEL-Aktivit�ten in den Prozess eingef�gt, die daf�r sorgen, dass die Marken, die beim Testen auf dem Ausf�hrungspfad liegen, an den Web Service �bertragen werden. Dank Vorbereitungen, die in der zweiten Phase gemacht wurden, k�nnen die Marken einfach durch entsprechende Aktivit�ten ersetzt werden, ohne dabei auf die Semantik und Syntax des Prozesses zu achten. 

 
  \subsubsection{Aktivit�tsabdeckung}\label{sec:aktivitaetsabdeckung}
  Zu betonen ist, dass es  bei der Messung der Testabdeckung nicht darum geht, ob die Aktivit�t \textit{erfolgreich} ausgef�hrt wurde, sondern lediglich darum, ob es zur Ausf�hrung der Aktivit�t kommt. Die Logging-Aktivit�ten, welche die Ausf�hrung der Basisaktivit�ten signalisieren sollen, werden im Kontrollfluss direkt vor der jeweiligen Aktivit�t platziert. Bei der \textit{receive}-Aktivit�t kann die Protokollierung ausnahmsweise erst nach der Ausf�hrung stattfinden. Die \textit{receive}-Aktivit�t wird erst ausgef�hrt, wenn sie die Kontrolle hat und eine Nachricht von au�en ankommt. Bei diesem Verhalten darf die Ausf�hrung der Aktivit�t erst nach dem Eintreffen einer Nachricht registriert werden. 
  
  Befindet sich eine Basisaktivit�t in einer \textit{sequence}, so kann die zugeh�rige Logging-Aktivit�t direkt davor oder danach eingef�gt werden. Die Semantik der \textit{sequence}-Aktivit�t sorgt daf�r, dass die Logging-Aktivit�t und die Basisaktivit�t im Kontrollfluss direkt hintereinander erreicht werden. Ist eine Basisaktivit�t in einer anderen Strukturierten Aktivit�t eingebettet, so wird eine \textit{sequence} eingef�gt, die diese Basisaktivit�t und die zugeh�rige Loggingaktivit�t umschlie�t. Durch das Schachtelungsprinzip der Strukturierten Aktivit�ten ist diese Vorgehensweise an jeder Stelle des Prozesses m�glich.
  
  Das Link-Konzept braucht, wie bereits angek�ndigt, braucht eine  besondere Betrachtung. 
  Innerhalb der \textit{flow}-Aktivit�ten erm�glicht das Link-Konzept die Syn\-chro\-ni\-sa\-tions\-ab\-h�n\-gig\-keit\-en zwischen den Aktivit�ten zu definieren und durch die darauf aufbauenden Bedingungen die Ausf�hrung der Aktivit�ten zu steuern (Abschnitt \ref{sec:linkkonzept}). Befindet sich eine eingef�gte Logging-Aktivit�t im Kontrollfluss direkt vor oder nach der Basisaktivit�t, so besteht die Gefahr, dass die Ausf�hrung falsch erfasst wird:
\begin{itemize}
	\item Protokollierung findet vor der Basisaktivit�t statt und die Aktivit�t  selbst wird aufgrund der \textit{transition} oder \textit{join condition} nicht ausgef�hrt.
	\item Protokollierung findet nach der Basisaktivit�t statt, die Aktivit�t selbst wird aufgrund der \textit{transition} oder \textit{join condition} nicht ausgef�hrt und das Attribut \textit{suppress join} ist mit dem Wert \textit{"`yes"'} belegt.
\end{itemize}
In den beiden F�llen wird die Ausf�hrung einer Aktivit�t registriert, obwohl die Aktivit�t gar nicht ausgef�hrt wurde.

Um zu erreichen, dass die Aktivit�t und das Logging entweder beide ausgef�hrt oder beide nicht ausgef�hrt werden, muss die Synchronisation und damit der Link sich auf beide beziehen. Zu beachten ist, dass ein Link genau eine \textit{target}- und eine \textit{source}-Aktivit�t verbinden kann. Damit ist es nicht m�glich, denselben Link f�r die Aktivit�t selbst und f�r die Protokollierung zu verwenden.  Schlie�t man die zu protokollierende Aktivit�t und die Logging-Aktivit�t in eine \textit{sequence} ein und definiert diese als Ziel des Links, so ist das gew�nschte Verhalten erreicht, ohne die Gesch�ftslogik des BPEL-Prozesses zu beeinflussen.  Das gilt f�r beide Werte des \textit{suppress join}-Attributes. 
\begin{figure}[htbp]
	\centering
		\includegraphics[width=0.85\textwidth]{bilder/ActivityIsTarget.png}
		\caption{Erfassung von Basisaktivit�ten in flow-Umgebung}
	\label{fig:activityAsTarget}
\end{figure}
Die Definition und die Quelle des Links (\textit{source}-Aktivit�t) bleiben unver�ndert. Die Abbildung \ref{fig:activityAsTarget} stellt die Manipulation grafisch dar. Die Legende ist im Anhang zu finden Die neu eingef�gten Aktivit�ten sind unterstreichen.

\newpage
Die entsprechenden �nderungen des XML-Codes:
      %\begin{lstlisting}[caption=\textit{Invoke}-Aktivit�t als Ziel\\ eines Links,numbers=none,label=Listing1]{Name}

\setlength{\columnsep}{1cm}
\begin{multicols}{2}
\lstset{emph={joinCondition,target }, emphstyle=\color{red}}
 
     \begin{lstlisting}[numbers=none,label=Listing1]{Name}
<flow>

	 ...
  <invoke ... joinCondition=...>
    <target linkName="linkA"/>
  </invoke>  
  
  
</flow>    
  \end{lstlisting}
\lstset{emph={[2]sequence}, emphstyle=[2]\color{blue}}
      \begin{lstlisting}[numbers=none]{Name}          
<flow>
	 ...
  <sequence joinCondition=...>
    <target linkName="linkA"/>
    <!--Logging_of_invoke-->
    <invoke .../>
  </sequence>
  ...
</flow>      
\end{lstlisting}
\end{multicols}


\subsubsection{Zweigabdeckung}\label{sec:zweigabdeckung}
Die Strukturierten Aktivit�ten bestimmen in BPEL den Kontrollfluss und spielen damit eine entscheidende Rolle f�r die Ermittlung der Zweigabdeckung. Die Zweige werden n�mlich bei der Analyse der Strukturierten Aktivit�ten des Prozesses identifiziert. Die Protokollierung der Zweigen erfolgt im Wesentlichen nach dem Schema, das f�r die Erfassung der Aktivit�tsabdeckung vorgestellt wurde. An den entsprechenden Stellen in der BPEL-Datei werden Marken eingef�gt, die das Aktivieren der entsprechenden Zweigen markieren. Damit die Datei bei diesen Manipulationen syntaktisch g�ltig bleibt, wird ebenfalls, falls notwendig, 
eine umschlie�ende \textit{sequence}-Aktivit�t eingef�gt. 

Ausf�hrlicher werden einige Strukturierten Aktivit�ten vorgestellt, die im Bezug auf die Zweigabdeckung einige Besonderheiten aufweisen.

\textbf{\textit{If}-Aktivit�t.}
Bei diesem Konstrukt muss darauf geachtet werden, dass der \textit{else}-Zweig erfasst wird. Ist der Zweig nicht explizit im Quellcode vorhanden, dann muss er hinzugef�gt und wie alle anderen Zweige mit einer Marke annotiert werden.

\textbf{\textit{Flow}-Aktivit�t.}
Bei der Messung der Zweigabdeckung in einer \textit{flow}-Aktivit�t m�ssen einige Details beachtet werden. Die Abbildung \ref{fig:zweigabdeckungInFlow} zeigt ein kleines Beispiel, an dem die Problemen und entsprechende L�sung erl�utert werden. Die breiten Pfeile repr�sentieren die Zweige \textit{branch\_1} und \textit{branch\_2}, deren Aktivierung protokolliert werden soll. Jeder BPEL-Prozess muss 
\begin{wrapfigure}[11]{r}{7,2cm}
\centering%
\includegraphics[width=0.45\textwidth]{bilder/FlowCreateInstance.png}%
\caption{Beispiel - Zweigabdeckung in \textit{flow}-Aktivit�t}
\label{fig:zweigabdeckungInFlow}
\end{wrapfigure}mindestens eine Aktivit�t enthalten, die die Instanziierung des Prozesses initiieren kann (Attribut \textit{createInstance}=\textit{"'yes"'}). In diesem Beispiel ist das die \textit{receive}-Aktivit�t. Alle anderen Basisaktivit�ten m�ssen im Kontrollfluss hinter dieser Aktivit�t liegen. Der Beispielprozess ist nur aufgrund des Synchronisationslinks \textit{linkA} g�ltig. Der Link sorgt daf�r, dass die \textit{receive}-Aktivit�t vor \textit{activity\_1} abgearbeitet wird. 

F�r die �berwachung dieser Zweige wird der Prozess nach dem bisherigen Schema erweitert. Die Abbildung \ref{fig:ZweigabdReceive} (links) stellt den modifizierten Prozess grafisch dar. Es werden zwei Logging-Aktivit�ten eingef�gt, die jeweils in eine \textit{sequence}-Aktivit�t eingeschlossen sind. Der entstandene Prozess ist nicht mehr g�ltig: die \textit{receive}-Aktivit�t ist nicht mehr die erste Basisaktivit�t im Kontrollfluss.

 Die Protokollierung der Zweige kann erst nach der Erzeugung einer Prozessinstanz erfolgen. Die zugeh�rigen Logging-Aktivit�ten d�rfen im Kontrollfluss nur nach der receive-Aktivit�t platziert werden. Die rechte Abbildung zeigt den Prozess, der diesen Anforderungen entspricht. 
\begin{figure}[htbp]
  \centering
  \begin{minipage}[b]{.485\textwidth}
    \includegraphics[width=\linewidth]{bilder/FlowCreateInstance2.png}  
  \end{minipage}
  \hspace{0.1cm}
  \begin{minipage}[b]{.485\textwidth}
    \includegraphics[width=\linewidth]{bilder/FlowCreateInstance3.png}  
  \end{minipage}
      \caption{Erfassung der Zweigabdeckung in einer \textit{flow}-Aktivit�t}
      \label{fig:ZweigabdReceive}
\end{figure}
Die Protokollierung des Zweiges \textit{branch\_1} darf erst nach der Ausf�hrung der \textit{receive}-Aktivit�t stattfinden. Zu Beachten ist, dass die komplette \textit{sequence}-Aktivit�t aufgrund des Links nach der \textit{receive}-Aktivit�t ausgef�hrt wird. Diese Anordnung stellt sicher, dass ein Zweig nur dann als "`abgedeckt"' gilt, wenn die Zielaktivit�t dieses Zweiges zur Ausf�hrung kommt. Es ist sinnvoll, die Aktivierung eines Zweiges im Zusammenhang mit Abdeckungsmetriken wie folgt zu definieren:

\textit{Ein Zweig ist genau dann "`aktiviert"', wenn die Aktivit�t, zu der der Zweig f�hrt, startet bzw. bei \textit{receive}-Aktivit�t wartet auf Nachricht.}

Nur in diesem Sinne "`aktivierte"' Zweige gelten beim Test als abgedeckt.
Das hei�t, dass die Synchronisationslinks ber�cksichtigt werden m�ssen. 
 
 
 \textbf{\textit{RepeatUntil}-Aktivit�t.}
 Die Auswertung der Bedingung erfolgt bei dieser Wiederholungsschleife erst nach der Ausf�hrung des Schleifenk�rpers. In der Schleife kann man ohne weiteren Logik nicht feststellen, ob das die erste oder bereits wiederholte Ausf�hrung ist. Denn im Falle einer wiederholten Ausf�hrung muss die Aktivierung des Zweiges von der Bedingung zum Schleifenk�rper (in der Abbildung \ref{fig:repeatUntil2} \textit{branch\_2}) signalisiert werden. Die entsprechende �berpr�fung kann mit Hilfe eines Z�hlers, der bei jedem Schleifendurchlauf hochgez�hlt wird, und einer \textit{if}-Abfrage realisiert werden.  Die Z�hlvariable muss nat�rlich im Prozess deklariert und vor der Schleife initialisiert werden. Die entsprechende Modifikation der \textit{repeatUntil}-Aktivit�t ist in der Abbildung dargestellt. Die Deklaration der Variable und die Initialisierung vor der Schleife sind nicht abgebildet.
 \begin{figure}[htbp]%
	\centering
		\includegraphics[width=0.65\textwidth]{bilder/repeatUntilZweig.png}
      \caption{Erfassung der Zweigabdeckung in einer \textit{repeatUntil}-Aktivit�t}
	\label{fig:repeatUntil2}
\end{figure}
\FloatBarrier 
\subsubsection{Linkabdeckung}\label{sec:Linkabdeckung}
Es geht bei dieser Metrik um Synchronisationslinks mit einer \textit{transition condition}, die nicht mit einem konstanten Wert \textit{true} oder \textit{false} belegt ist. Die Ermittlung dieser Metrik ist ebenfalls mit Hilfe der Links realisiert. Um die korrekte Erfassung des Linkstatus (\textit{transition condition}) zu erreichen m�ssen folgende Aspekte beachtet werden:
\begin{itemize}
	\item \textit{boundary crossing}-Restriktionen,
	\item unterschiedliche Wertebelegung des \textit{suppress join}-Attributes und daraus resultierendes Verhalten beim Auftreten von \textit{Join Failure},
	\item \textit{Dead-Path-Elimination}.
\end{itemize} 
Diese Aspekte wurden im Abschnitt \ref{restrictions} ausf�hrlich erkl�rt.

\textbf{Beispiel}. Als Grundlage f�r die Erl�uterungen dient folgendes Szenario. Ein Unternehmen  hat einige Strategien entwickelt, um die Gro�kunden trotz gro�er Konkurenz an sich langfristig zu binden. Als besonderen Service wird  zum Beispiel eine bestimmte Lieferdauer f�r diese Kunden  garantiert. Aus diesem Grund ist man  bereit, die Engp�sse auf dem Lager f�r die Gro�kunden durch Bestellungen bei Konkurenz zu �berbr�cken und auf diese Weise kurzfristig sogar Verluste in Kauf zu nehmen. Der Gesch�ftsprozess wird entsprechend der Strategie angepasst und in den IT-Systemen (mit BPEL) umgesetzt. Ein vereinfachter Ausschnitt des zugeh�rigen BPEL-Prozesses ist in der Abbildung \ref{fig:LinkAbdeckungBeispiel} grafisch Dargestellt.

\begin{wrapfigure}[13]{r}{7,5cm}
\centering%
\includegraphics[width=0.48\textwidth]{bilder/FlowBeispiel.png}%
      \caption{Linkabdeckung - Beispielprozess }
\label{fig:LinkAbdeckungBeispiel}
\end{wrapfigure} 
Zwei Web Services: \textit{customer identification}- und \textit{external order}-Service gibt es, welche von BPEL in dem Prozess aufgerufen werden.   Die daf�r zust�ndigen \textit{invoke}-Aktivit�ten sind in einer \textit{flow}-Umgebung eingebettet. Die \textit{if}-Abfrage (\textit{orderSize}$>$\textit{inventory}) stellt sicher, dass externe Unternehmen nur dann beauftragt werden, wenn die Ware nicht im Lager vorhanden ist. Da die externe Bestellungen nur bei Gro�kunden in Frage kommen, muss erst der Kunde identifiziert werden. Die entsprechende Reihenfolge der Web Service-Aufrufe wird durch das Synchronisationslink \textit{link1} gesichert. Die \textit{transition condition} des Links \textit{customerType='major'} sichert, dass nur die Gro�kunden diesen Service genie�en k�nnen.  

Die Erfassung des Status jedes relevanten Links ist f�r die Ermittlung der Linkabdeckung notwendig. Abh�ngig von dem Wert der \textit{transition condition} muss w�hrend der Ausf�hrung die Marke an den \textit{Coverage Logging Service} verschickt werden, die auf den Wert der \textit{transition condition} zur�ck schlie�en l�sst.  Als m�gliche L�sungen werden drei Ans�tze mit ihren Vor- und Nachteilen vorgestellt und anschlie�end eine ausgew�hlt.

\underline{Variante 1.}\\
Zwei zus�tzliche Links werden definiert (\textit{link1Copy} und \textit{link1Neg}). Dabei �ber\-nimmt der \textit{link1Neg} die negierte  \textit{transition condition} des Originallinks ( \textit{not(customerType='major')}) und  der \textit{link1Copy} die unver�nderte \textit{transition condition}. Die Negierung erfolgt in den meisten Anfragesprachen mit Hilfe der Funktion \textit{not()}. Das gilt sowohl f�r \textit{XPath 1.0} \cite{W3C1999}, die \textit{default}-Sprache in BPEL ist, als auch f�r \textit{XQuery 1.0} \cite{W3C2007a}. Beim Einsatz weiterer Anfragesprachen muss �berpr�ft werden, ob die Funktion \textit{not()} vorhanden ist, und, falls das nicht der Fall ist, muss entsprechende Erweiterungen vorgenommen werden. Die Logging-Aktivit�ten werden bei diesem Ansatz direkt in der \textit{flow}-Umgebung platziert. 
  Dadurch sind sie bereit f�r die Ausf�hrung, sobald der Kontrollfluss die \textit{flow}-Aktivit�t erreicht. Die \textit{source}- und \textit{target}-Aktivit�ten des Originallinks k�nnen dabei beliebig tief in Strukturierte Aktivit�ten eingebettet sein. Durch die am Anfang definierten Links, wird die \textit{source}-Aktivit�t des Originallinks mit den Logging-Aktivit�ten verlinkt. Sobald der Status der Links bekannt ist, wird eine Logging-Aktivit�t ausgef�hrt.

Die \textit{default join condition} der Logging-Aktivit�ten (\textit{or}-Verkn�pfung der Status aller eingehenden Links) und komplement�ren \textit{transition conditions} der beiden neuen Links garantieren, dass nur eine Logging-Aktivit�t gleichzeitig ausgef�hrt werden kann.
\begin{figure}[htbp]
	\centering
		\includegraphics[width=0.98\textwidth]{bilder/FlowBeispielV1.png}
      \caption{Erfassung der Linkabdeckung (erste Variante)}
\label{fig:LinksAnsatz1Designer}
\end{figure}


\begin{table}[h!]
\begin{tabular}{p{6cm}p{0.5cm}p{6cm}}
\textbf{Vorteile}&\ &\textbf{Nachteile}\\[0.1cm]
\hline \\
Logging kann stattfinden, unabh�ngig davon, ob \textit{external order}-Service aufgerufen wird&\ &\textit{boundary crossing}-Restriktionen k�nnen verletzt werden\\
\end{tabular}
%\caption{Vergleich der Ans�tze}
\label{Vergleichstabelle}
\end{table} 
\FloatBarrier
Die Variante erm�glicht den Linkstatus unabh�ngig von dem Aufruf des Web Services \textit{external order} zu erfassen. Das hei�t, der Linkstatus wird auch dann erfasst, wenn die Bedingung \textit{orderSize}$>$\textit{inventory} zu \textit{false} ausgewertet wird und der Zweig mit \textit{external order}-Service nicht ausgef�hrt wird. Damit wird die formale Definition der Metrik umgesetzt.

Allerdings funktioniert diese Vorgehensweise nicht in allen Situationen. Die Bearbeitung mehrerer Auftr�ge von verschiedenen Kunden kann in BPEL mittels einer Schleife umgesetzt werden. Die \textit{invoke}-Aktivit�t \textit{customer identification} und die komplette \textit{if}-Aktivit�t k�nnen in die Schleife eingebettet werden, um die Bestellungen auf die gleiche Weise nacheinander zu bearbeiten. In dieser Situation w�rde die Instrumentierung der BPEL-Datei nach dem vorgestelltem Verfahren die Syntaxregeln der BPEL-Sprache verletzen: ein Link darf die Grenzen einer Schleife nicht schneiden (\textit{boundary crossing}-Restriktionen, Abschnitt \ref{sec:linkkonzept}).  

\underline{Variante 2.}\\
Ziel dieses Ansatzes ist, die \textit{boundary crossing}-Restriktionen bei der Erfassung der Linkabdeckung einzuhalten. Die Konstruktion der Links \textit{link1Copy} und \textit{link1Neg} erfolgt auf dieselbe Weise.
Bei dieser Variante wird die \textit{target}-Aktivit�t des Links (\textit{invoke}-Aktivit�t \textit{external identification}) mit den beiden Logging-Aktivit�ten in eine \textit{flow}-Aktivit�t eingeschlossen.
\begin{figure}
	\centering
	\includegraphics[width=0.95\textwidth]{bilder/FlowBeispielV2.png}
      \caption{Erfassung der Linkabdeckung (zweite Variante)}
	\label{fig:LinksAnsatz2Designer}
\end{figure}


\begin{table}[h!]
\begin{tabular}{p{6cm}p{0.5cm}p{6cm}}
\textbf{Vorteile}&\ &\textbf{Nachteile}\\[0.1cm]
\hline \\
\textit{boundary crossing}-Restriktionen werden eingehalten&\ &Die Links k�nnen nur dann geloggt werden, wenn Web Service \textit{external order} aufgerufen wird.
\\
\end{tabular}
%\caption{Vergleich der Ans�tze}
\label{Vergleichstabelle}
\end{table}
Bei diesem Verfahren schneiden der Originallink und die neu eingef�gten Links die selben Grenzen der Strukturierten Aktivit�ten.
Setzt man eine syntaktisch g�ltige Ausgangs-BPEL-Datei voraus, so werden keine \textit{boundary crossing}-Restriktionen verletzt. Die eingef�gte \textit{flow}-Aktivit�t kann ebenfalls zu keiner Verletzung f�hren.

Wird die Bedingung \textit{orderSize}$>$\textit{inventory} zu \textit{false} ausgewertet, so k�nnen die Logging-Aktivit�ten nicht ausgef�hrt werden. Der Linkstatus kann in dieser Situation nicht erfasst werden.
\FloatBarrier
\underline{Variante 3.}\\
Bei dieser Variante wird die \textit{source}-Aktivit�t des Links (\textit{customer identification}) mit einer \textit{flow}-Aktivit�t umschlossen. Einerseits sorgt die Platzierung der Logging-Anweisungen in der selben \textit{flow}-Umgebung f�r die Einhaltung der \textit{boundary crossing}-Restriktionen. Andererseits kann das Logging auch stattfinden, wenn die Zielaktivit�t nicht ausgef�hrt wird. Da der Status der Links nach der Ausf�hrung der Quellaktivit�t gesetzt wird, sind die Logging-Aktivit�ten zu diesem Zeitpunkt startbereit und k�nnen gleich die entsprechenden Marken an den \textit{Coverage Logging}-Service verschicken.

\begin{figure}[htbp]
	\centering
		\includegraphics[width=0.95\textwidth]{bilder/FlowBeispielV3.png}
      \caption{Erfassung der Linkabdeckung (dritte Variante)}
	\label{fig:LinksAnsatz3Designer}
\end{figure}

W�hrend die ersten beiden Varianten keine korrekte Messung der Linkabdeckung in allen Situationen erm�glichen, setzt das dritte Verfahren die Definition der Linkabdeckung in allen Situationen ohne Ausnahmen um.
\FloatBarrier
\subsubsection{Fault und Compensation Handler-Abdeckung}
F�r die Fault und Compensation Handler werden die Marken auf dieselbe Weise (mit Hilfe einer umschlie�enden \textit{sequence}-Aktivit�t) in den \textit{catch}- und \textit{catchAll}-Bl�cken bzw. in Compensation Handler platziert. Die folgende Abbildung zeigt entsprechende Modifikationen am Beispiel von Fault Handler.

\begin{figure}[htbp]%
	\centering
		\includegraphics[width=0.65\textwidth]{bilder/FaultHandler.png}
      \caption{Erfassung der \textit{Fault Handler}-Abdeckung}
	\label{fig:LinksAnsatz3Designer}
\end{figure}


Zuvor m�ssen \textit{inline}-Handler (sowohl Fault- als auch Compensation-Handler), die in \textit{invoke}-Aktivit�ten direkt eingebettet sind, umgewandelt werden. Dabei wird die betroffene \textit{invoke}-Aktivit�t in einen expliziten Scope eingeschlossen, der die Handler auf die �bliche Weise definiert. Die entsprechenden �nderungen des XML-Quellcodes: 

\begin{multicols}{2}
\lstset{emph={invoke}, emphstyle=\color{red}}
      \begin{lstlisting}[numbers=none,label=Listing1]{Name}


<invoke name=...>
    <catch faultName="...">
    ...
    </catch>
    <catchAll>...</catchAll>
    <compensationHandler>
    ....
    </compensationHandler>
</invoke>      
  \end{lstlisting}
\lstset{emph=[2]{scope, faultHandlers}, emphstyle=[2]\color{blue}}
      \begin{lstlisting}[numbers=none]{Name}
      
      
       
<scope ...>
    <faultHandlers>
        <catch faultName=...>
        ...
        </catch>
        <catchAll>
        ...
        </catchAll>
    </faultHandlers>
    <compensationHandler>
    ...
    </compensationHandler>
    <invoke name=.../>
</scope>     
\end{lstlisting}
\end{multicols}

\FloatBarrier
\subsection{Erweiterung des BPEL-Prozesses f�r die �bertragung der Testabdeckungsdaten}\label{sec:markenuebertragung}
Ein BPEL-Prozess kann nach au�en grunds�tzlich nur mit Web Services kommunizieren. Demzufolge muss der Dienst f�r den Empfang der Nachrichten als Web Service realisiert und w�hrend des Testens erreichbar sein. 
 \begin{figure}[htbp]
	\centering
		\includegraphics[width=0.75\textwidth]{bilder/bpelProzess.png}
		\caption{Partner eines BPEL-Prozesses}
	\label{fig:loggingservice}
\end{figure}

\textbf{Web Service zum Empfangen von Marken (\textit{Coverage Logging Service})}.

F�r die Kommunikation des BPEL-Prozess mit einem Web Service sind folgende Daten notwendig:
\begin{itemize}
	\item WSDL-Beschreibung,
	\item \textit{Partner Link Type},
	\item \textit{Partner Link}.
\end{itemize}

\underline{WSDL-Beschreibung}. Die einzige Funktionalit�t, die der Service �ffentlich anbieten muss, ist der Empfang von Marken in Form von Zeichenketten. Dementsprechend einfach sieht die Schnittstelle dieses Services aus. Der \textit{Port Type} hat nur eine Funktion \textit{receiveMarkers}, die die Zeichenketten empfangen kann.  Als Nachrichtenprotokoll wird SOAP verwendet. Die Abbildung \ref{fig:webserrvice} zeigt die grafische Darstellung der Schnittstelle.
\begin{figure}[htbp]
	\centering
		\includegraphics[width=0.7\textwidth]{bilder/WebServiceWSDL.png}
		\caption{WSDL Beschreibung des \textit{Coverage Logging Service}}
	\label{fig:webserrvice}
\end{figure}

\underline{\textit{Partner Link Type}}. Der \textit{Partner Link Type} sowie die WSDL-Beschreibung des \textit{Coverage Logging Services} sind f�r die Kommunikation des BPEL-Prozesses mit dem Service notwendig. Wird der \textit{Partner Link Type} in die WSDL-Datei integriert, was zul�ssig ist, so steht dieser Link in allen BPEL-Dateien, die diese WSDL-Beschreibung referenzieren, zur Verf�gung. Damit entf�llt die Notwendigkeit, den Link explizit zu referenzieren. Aus diesem Grund wird diese L�sung vorgezogen.
\begin{verbatim}
<plnk:partnerLinkType
     xmlns:plnk="http://docs.oasis-open.org/wsbpel/2.0/plnktype"
     name="PLT_CoverageLoggingService_">
   <plnk:role name="MarkersReceiver" 
                     portType="_CoverageLoggingService_" />
</plnk:partnerLinkType>
\end{verbatim}

\underline{\textit{Partner Link}}. \textit{Partner Link} spezifiziert die in \textit{Partner Link Type} festgelegten  Rollen und wird in BPEL-Datei definiert. 
\begin{verbatim}
<partnerLink name="PL_CoverageLoggingService_" 
             partnerLinkType="PLT_CoverageLoggingService_" 
             partnerRole="MarkersReceiver" />
\end{verbatim}

\textbf{Datenaufbereitung und -�bertragung}.
Mit Hilfe von invoke-Aktivit�ten kann ein BPEL-Prozess mit einem Web Service kommunizieren. Die \textit{invoke}-Aktivit�t hat folgende Attributen:
\begin{itemize}
	\item \textbf{partnerLink}: identifiziert \textit{Partner Link}, der f�r die verwendet werden soll,
\item \textbf{portType}:  WSDL-\textit{Port Type} des Zielservices an,
\item \textbf{operation}: WSDL-Operation,
\item \textbf{inputVariable}: Variable, die die zu sendenden Daten enth�lt,
\item \textbf{outputVariable}: Variable, die die Antwort des Services speichert.
\end{itemize}

Soll eine asynchrone Kommunikation realisiert werden oder eine Operation aufgerufen werden, die keine R�ckgabedaten erzeugt, so wird der Attribut \textit{outputVariable} weggelassen. Eine \textit{invoke}-Aktivit�t, die die Marken an den Service verschickt, sieht folgenderma�en aus:  
\begin{verbatim}
<invoke inputVariable="_YXZ_0" 
        operation="receiveMarkers" 
        partnerLink="PL_CoverageLoggingService_" 
        portType="_CoverageLoggingService_" />
\end{verbatim}

Zuvor muss die Variable \textit{\_YXZ\_0} deklariert werden. Damit die Variable im ganzen Prozess benutzt werden kann, wird sie als globale Variable (im \textit{variables}-Element des Prozesses) definiert. Zur Vermeidung von Namenskonflikten werden die Variablennamen aus einem kryptischen Pr�fix und einem Z�hler, der bei jeder Variable hochgez�hlt wird, erzeugt. Da in BPEL durch die \textit{flow}-Aktivit�t die nebenl�ufige Ausf�hrung der Aktivit�ten m�glich ist, muss f�r jeden Zweig eine neue Variable deklariert und als \textit{inputVariable} bei \textit{invoke}-Aufrufen verwendet werden. Die Zuordnung der Marken erfolgt mit Hilfe einer \textit{assign}-Aktivit�t:
\begin{verbatim}
<assign>
   <copy>
      <from>
         <literal>marker1#marker2#markers3</literal>
      </from>
      <to part="markers" variable="_YXZ_0"/>
   </copy>
</assign>
   \end{verbatim}

\section{Auswertung der Daten und Berechnung der Metriken}\label{sec:auswertung}
Beim Instrumentieren der BPEL-Datei werden Marken generiert, die die entsprechenden Aspekte des Kontrollflusses eindeutig identifizieren und zu einer bestimmten Metrik geh�ren. Alle Marken, die in den Kontrollfluss eingef�gt wurden, werden gespeichert und stehen w�hrend und nach der Ausf�hrung des Prozesses zur Verf�gung.
W�hrend der BPEL-Prozess zum Testen ausgef�hrt wird, werden die Marken an den \textit{Coverage Logging}-Service gesendet. 
Die empfangenen Marken liegen auf dem Ausf�hrungspfad und wurden damit durch die Tests abgedeckt. 

Anhand der bei der Instrumentierung gespeicherten und w�hrend der Ausf�hrung gesammelten Daten kann die Testabdeckung ermittelt und Statistiken erstellt werden. Dabei gibt es die M�glichkeit sowohl die Statistik einer ganzen \textit{Testsuite} als auch der einzelnen Testf�llen zu generieren.

\section{Zusammenfassung}
In diesem Kapitel wurde ausf�hrlich erl�utert, wie die Ermittlung der Testabdeckung mit dem Instrumentierungsverfahren umgesetzt werden kann.
Die Ermittlung der Testabdeckung wurde dabei in drei Phasen aufgeteilt: Instrumentierung der BPEL-Datei, Sammlung der Daten w�hrend der Testausf�hrung und Auswertung der Daten mit anschlie�ender Berechnung der Metriken. Die Instrumentierung, als Kernaufgabe des Verfahrens, wurde dabei besonders ausf�hrlich beschrieben. 

Nach der Identifizierung der Aufgaben dieser Phase wurde sie wiederum in drei Phasen aufgeteilt, die f�r jede Metrik erl�utert wurden. Anhand von verschiedenen Szenarien wurden Problemen aufgedeckt und m�gliche L�sungen diskutiert.
% 




 
 

Ein BPEL-Prozess kann nach au�en grunds�tzlich nur mit Web Services kommunizieren. Demzufolge muss der Dienst f�r den Empfang der Nachrichten als Web Service realisiert und w�hrend des Testens erreichbar sein.  Der Web Service kann dabei auf die Funktionalit�ten, die das BPELUnit Framework f�r die Mocks zur Verf�gung stellt, zugreifen. 
Die Abbildung \ref{fig:loggingservice} stellt die Funktionsweise dar. 
 \begin{figure}[htbp]
	\centering
		\includegraphics[width=0.90\textwidth]{bilder/loggingservice.pdf}
		\caption{Service}
	\label{fig:loggingservice}
\end{figure}

Die Abbildung \ref{} skizziert den gesamten Ablauf der Abdeckungsmessung.
\begin{itemize}
\item Referenzierung des Web Services im BPEL-Prozess.
\item Annotation des PBPEL-Prozesses durch Marken. Diese Marken sind eindeutig und signalisieren je nach Metrik entweder die Ausf�hrung der Aktivit�ten oder Aktivierung bestimmter Kontrollflusse. Gleichzeitig m�ssen die Daten f�r die sp�tere Auswertung gespeichert werden.
\item Hinzuf�gen von Aktivit�ten, die die Marken an den Service verschicken.		
\item Sammlung der Daten w�hrend der Ausf�hrung des Prozesses.
\item Auswertung der Daten und Ermittlung von Metriken.
\end{itemize} 

Im Abschnitt \ref{sec:empfaengerWS} wird erl�utert, was notwendig ist, damit WS als Kommunikationspartner dem Prozess zur Verf�gung steht. 
Der zweite und der dritte Schritt geh�ren zu der Instrumentierung und werden im Abschnitt \ref{sec:instrumentierung} erl�utert.
Die letzten zwei Schritte werden in Abschnitt erkl�rt.

\section{Referenzierung des Web Services}\label{sec:empfaengerWS}
F�r die Kommunikation des BPEL-Prozess mit dem WS sind f�r den Prozess selbst oder die Engine folgende Daten notwendig:
\begin{itemize}
	\item WSDL Beschreibung
	\item PartnerLinkType
	\item PartnerLink
	\item die konkrete Adresse, unter der der service erreicht werden kann.
\end{itemize}

\textbf{WSDL Beschreibung}.Die einzige Funktionalit�t, die der Service �ffentlich anbieten muss, ist der Empfang von Marken in Form von Strings. Dementsprechend einfach sieht die Schnittstelle dieses Services aus. Der \textit{Port Type} hat nur eine Funktion \textit{receiveCoverageLabels}, die String empfangen kann. Der Service ist unter einer festen Addresse "`http:\/\/localhost:7777\/ws\/\_CoverageReportingService\_"' erreichbar.  Die Abbildung \ref{fig:webserrvice} zeigt die grafische Darstellung der Schnittstelle.
\begin{figure}[htbp]
	\centering
		\includegraphics[width=0.7\textwidth]{bilder/WebServiceWSDL.png}
		\caption{WSDL Beschreibung vom Service zum Empfangen von Marken}
	\label{fig:webserrvice}
\end{figure}

\textbf{PartnerlinkType}. Die einfachste M�glichkeit in disem Fall ist, den PartnerLink direkt in der WSDL-Datei zu platzieren.

\textbf{PartnerLink}. PartnerLink spezifiziert die in PartnerLinkType festgelegten  Rollen und wird in BPEL-Datei definiert. 

\textbf{Adresse}. F�r den PartnerlinkType m�ssen WSDL service und port sowie konkrette Adresse des Partners festgelegt werden. Diese Information muss der Engine zur Verf�gung stehen und wird in Deployment Descriptor festgelegt.

Der Web service wird steht lokal zur Verf�gung.


\section{Instrumentierung}\label{sec:instrumentierung}
Die Instrumentierung eines BPEL-Prozesses besteht aus mehreren Aufgaben:
\begin{itemize}
\item Analyse des BPEL-Prozesses.
\item Sammlung und Speicherung der statischen Daten (f�r die sp�tere Auswertung).
\item Annotation des BPEL Prozesses mit Marken. Jede Marke identifiziert ein relevantes f�r die jeweilige Metrik Aspekt (entweder eine Aktivit�t oder Kontrollfluss�nderung). Erreicht der Kontrollflu� beim Testen die Stelle, an der die Marke positioniert ist, so ist der zugeh�rige Aspekt durch den Test abgedeckt.
\item die Marken von allen Metriken , die im Kontrollflu� unmittelbar hintereinanderliegen, werden zusammengefasst (Optimierung).
\item Hinzuf�gen der BPEL-Aktivit�ten, f�r die �bertragung der Marken an den Web Service f�r Empfang von Marken (Protokollierung).	
\end{itemize}
Die Ausf�hrung der Aufgaben erfolgt in drei Phasen, die sequentiell abgearbeitet werden k�nnen. 


\begin{figure}[htbp]
	\centering
		\includegraphics[width=0.5\textwidth]{bilder/Phasen.png}
		\caption{Phasen}
	\label{fig:phasen}
\end{figure}

\textbf{Phase 1.} W�hrend der ersten Phase wird der BPEL-Prozess analysiert. Dabei werden f�r jede Metrik alle f�r sie relevanten Elemente des Prozesse bestimmt und gespeichert. Die Definitionen der Metriken(siehe Abschnitt ..) legen die relevanten Elemente des Prozesses f�r die jeweilige Metrik fest:
\begin{itemize}
	\item Aktivit�t Abdeckung \- alle Basisaktivit�ten.
	\item Zweigabdeckung \- alle strukturierten Aktivit�ten.
	\item Link Abdeckung \- alle Links mit \textit{transition condition}, die nicht kostant \textit{true} oder \textit{false} ist.
	\item Fault Handler Abdeckung \- alle catch- und catchAll-Elemente.
	\item Compensation Handler Abdeckung \- alle Compensation Handler. 
\end{itemize}

\textbf{Phase 2.} Diese Phase wird ausf�hrlich in den folgenden 4 Abschnitten f�r jede Metrik vorgestellt. An dieser Stelle werden lediglich einige Bemerkungen gemacht, die f�r jede Metrik gelten. 

In dieser Phase werden die statische Daten des Prozesses, die f�r die Ermittlung der Metriken notwendig sind, gesammelt und gespeichert. Gleichzeitig  erfolgt die Annotation des Prozesses mit speziellen Marken. Die Platzierung des Marken im Kontrollflu� muss so erfolgen, dass sowohl die Marke als auch der zugeh�rige Aspekt  bei der Ausf�hrung erreicht werden  oder beide nicht. Das ist die wichtigste Anforderung an die Platzierung der Marken.   Wie die einzelnen Metriken bei der Annotation des Prozesses genau vorgehen, wird in den folgenden Abschnitten erl�utert.  

Bei der Instrumentierung ist zu beachten, dass es grunds�tzlich m�glich ist, die interne Programmlogik durch die Manipulation zu ver�ndern. Die syntaktische Korrektheit darf nat�rlich auch nicht beeintr�chtigt werden. Obwohl die Marken als Kommentare eingef�gt werden und damit weder Programmlogik noch die syntaktische Struktur des Programmes �ndern, werden diese Aspekte bereits in dieser Phase beachtet. Der Grund daf�r ist, dass jede Metrik um bestimmte Aspekte zu erfassen zus�tzliche Logik braucht (insbesondere im Zusammenhang mit dem Link-Konzept).  Also ist es sinnvoll bereits bei der Platzierung der Marken f�r die semantische und syntaktische Korrektheit des Prozesses zu sorgen, so dass die Marken in der letzten Phase einfach durch entsprechende Aktivit�ten ersetzt werden k�nnen.

Dadurch dass in der ersten Phase alle ben�tigten Elemente des Originalprozesses f�r alle Metriken gespeichert wurde, k�nnen
die Metriken in beliebigen Reihenfolge abgearbeitet werden. 

\textbf{Phase 3.} In dieser Phase werden die Marken, die im Kontrollflu� hintereinander liegen,  zusammengefasst. Anschlie�end werden alle entstandenen Markengruppen  durch BPEL-Aktivit�ten ersetzt, die die Marken an den Web Service verschicken. Die Tatsache, dass der Prozess in der zweiten Phase so aufbereitet wurde, dass die Marken einfach durch Aktivit�ten ersetzt werden k�nnen, vereinfacht diese Phase wesentlich.       


% In diesem Abschnitt wird die Umsetzung der besprochenen Abdeckungsmetriken durch vorgestellt.  Wie im Abschnitt ... bereits erl�utert wurde, wird dabei der Instrumentierungsansatz verwendet. Zu beachten ist, dass es bei diesem Ansatz die M�glichkeit besteht, durch die Manipulation die interne Programmlogik zu ver�ndern. Aus diesem Grund muss man bei allen Eingriffen in die Struktur der BPEL-Datei daf�r sorgen, dass die Gesch�ftslogik unver�ndert bleibt. Die syntaktische Korrektheit darf nat�rlich auch nicht verletzt werden.
  \subsection{Statementabdeckung}
  Fast alle strukturierenden Aktivit�ten sind so konstruiert, dass sie in ihren Komponenten genau eine Aktivit�t erlauben. Die Ausnahmen sind  \textit{flow}- und \textbf{sequence}-Aktivit�t. Durch das Schachtelungsprinzip sind trotzdem beliebig komplexe Strukturen realisierbar. Wenn nur eine Basisaktivit�t an so einer Stellen  vorhanden ist, dann w�rde das Einf�gen einer zus�tzlichen Aktivit�t zur Verletzung der Syntaxregeln f�hren. Also muss man vor der Instrumentierung eine \textit{umschlie�ende strukturierende} Aktivit�t hinzuf�gen. Da das Logging entweder direkt davor oder danach stattfinden soll, ist die Sequence-Aktivit�t zu diesem Zweck gut geeignet. 
  
  Auch die Basisaktivit�ten, die sich direkt in \textit{flow}-Umgebungen befinden, werden in \textit{sequence} eingeschlossen, obwohl die Syntaxregeln das Hinzuf�gen von zus�tzlichen Aktivit�ten erlauben. Sonst w�rde es zu einer parallelen Ausf�hrung der Aktivit�t selbst und des Loggings f�hren. Das Link-Konzept macht die Instrumentierung innerhalb der flow-Aktivit�t zu einer nicht trivialen Aufgabe und erfordert eine besondere Betrachtung.
  
\textbf{\textit{Flow}-Aktivit�t}   
  Innerhalb der Flow-Aktivit�ten erm�glicht das Link-Konzept die Syn\-chro\-ni\-sa\-tions\-ab\-h�n\-gig\-keit\-en zwischen den Aktivit�ten zu definieren und durch die darauf aufbauenden Bedingungen die Ausf�hrung der Aktivit�ten zu steuern. So kann das einfache Einf�gen der Protokollieranweisungen zu  falschen Aussagen �ber ausgef�hrte oder nicht ausgef�hrte Aktivit�ten f�hren. Ist \textit{die zu protokollierende Aktivit�t Ziel eines oder mehreren Links}, so f�hren folgende Konstellationen zu falschen Aussagen:
\begin{itemize}
	\item Protokollierung findet vor der Aktivit�t statt und die Aktivit�t wird aufgrund der \textit{transition-} oder \textit{joinCondition} nicht ausgef�hrt.
	\item Protokollierung findet nach der Aktivit�t statt, die Aktivit�t wird aufgrund der \textit{transition-} oder \textit{joinCondition} nicht ausgef�hrt und die Variable suppressJoin ist mit dem Wert \textit{yes} belegt.
\end{itemize}
In den beiden F�llen wird eine Aktivit�t zu viel als "`ausgef�hrt"' registriert.

Um zu erreichen, dass die Aktivit�t und das Logging \textit{beide} entweder ausgef�hrt oder nicht ausgef�hrt werden, muss der Link sich auf beides beziehen. Zu beachten ist, dass ein Link nur eine Quelle und ein Ziel besitzen darf. Schlie�t man die zu protokollierende Aktivit�t und die Logginganweisung in eine \textit{sequence} ein und definiert diese als Ziel des Links (durch \textit{target}-Element), so ist das gew�nschte Verhalten erreicht. Das gilt f�r beide Werte der \textit{suppressJoin}-Varaible. 
Das urspr�ngliche \textit{target}-Element muss entfernt werden. Die Definition und die Quelle des Links bleiben unver�ndert. Die Abbildung .. stellt die entsprechenden �nderungen des Quellcodes an einem Beispielauschnitt dar.
 
\lstset{emph={joinCondition,target }, emphstyle=\color{blue}}
      \begin{lstlisting}[caption=Beispielcode]{Name}
<flow>
	 ...
  <invoke ... joinCondition=...>
    <target linkName="CtoD"/>
  </invoke>
</flow>      \end{lstlisting}

\lstset{emph={[2]sequence}, emphstyle=[2]\color{red}}
      \begin{lstlisting}[caption=Beispielcode][firstnumber=1]{Name}
<flow>
	 ...
  <sequence joinCondition=...>
    <target linkName="CtoD"/>
    <!--Logging-->
    <invoke .../>
    <!--Logging-->
  </sequence>
</flow>      \end{lstlisting}
Zu erw�hnen ist noch, dass wenn der Link in dem Ausgangsprozess keine \textit{CrossingBoundary}-Bedingung verletzt hat, so �ndert sich das durch diesen Eingriff nicht.

\subsection{Zweigabdeckung}
Bei der Erfassung der Zweigabdeckung m�ssen durch das Einf�gen von \textit{sequence} an den richtigen Stellen die gleichen Vorbereitung wie bei der Statementabndeckung getroffen werden. In diesem Abschnitt wird auf die Besonderheiten einiger strukturierender Aktivit�ten eingegangen.

\textbf{\textit{If}-Aktivit�t.}
Es muss bei diesem Konstrukt darauf geachtet werden, dass der \textit{else}-Zweig erfasst wird. Das heisst, falls dieser Zweig nicht vorhanden ist, muss er mit der Logging-Anweisung hinzugef�gt werden.

\textbf{\textit{Flow}-Aktivit�t.}
Das Komplizierteste bei der Instrumentierung innerhalb der flow-Aktivit�t ist das Link-Konzept und die Restriktionen, die daraus entstehen:
\begin{itemize}
	\item \textit{boundary-crossing}-Restriktionen
	\item Wertebelegung der \textit{suppressJoin}-Variable
	\item Dreiwertige Status der Links
	\item \textit{Dead-Path-Elimination}
\end{itemize}
Es geht insbesondere darum, die Links mit dem zugeh�rigen Status zu erfassen. Es werden drei verschiedene Ans�tze mit ihren Vor- und Nachteilen vorgestellt. Das Beispiel aus der Abbildung ... dient als Grundlage f�r die Erl�uterung. 

\underline{Variante 1.}\\
Zwei zus�tzliche Links werden definiert (\textit{LinkACopy} und \textit{LinkACopyNegiert}). Dabei �bernimmt der \textit{LinkACopyNegiert} die negierte  transitionCondition des Originallinks und  der \textit{LinkACopy} die unver�nderte. Die Logging-Anweisungen werden bei diesem Ansatz direkt in der flow-Umgebung platziert. Dadurch sind sie bereit f�r die Ausf�hrung, sobald die Ausf�hrung der flow-Aktivit�t startet. Durch die am Anfang definierten Links, wird die Source-Aktivit�t des Originallinks mit den Logging-Anweisungen verlinkt. F�r eine Logging-Anweisung ein Link. Sobald der Status der Links bekannt ist, wird eine Logging-Anweisung ausgef�hrt.

Die \textit{Default-JoinCondition} der Logging-Aktivit�ten (\textit{or}-Verkn�pfung von Status aller eingehenden Links) und die gegenseitig negierten \textit{transitionCondition} garantieren, dass nur ein Logging gleichzeitig ausgef�hrt werden kann.

Bei dieser Vorgehensweise kann aber der negative Status des Links, der aufgrund der \textit{Dead-Path-Elimination}(DPE) propagiert wird, nicht erfasst werden. Der Grund daf�r ist, dass der Status aller Links auf \textit{false} gesetzt wird und aus diesem Grund keine von beiden Logging-Anweisungen ausgef�hrt wird. 

Durch eine zus�tzliche Logging-Anweisung und ein Link kann auch dieser Fall ber�cksichtigt werden. Als Quelle wird dieselbe Aktivit�t als bei den anderen Links verwendet. Die \textit{transitionCondition} wird auf den konstanten Wert \textit{true} und die joinCondition auf den negierten Status des Links gesetzt. In dieser Konstellation findet das Logging nur in einer Situation statt, und zwar genau dann, wenn der Status des Links auf \textit{false} gesetzt wurde, was in diesem Fall nur durch DPE m�glich ist.
\begin{table}[h!]
\begin{tabular}{p{6cm}p{0.5cm}p{6cm}}
\textbf{Vorteile}&\ &\textbf{Nachteile}\\[0.1cm]
\hline \\
Logging der kann stattfinden, unabh�ngig davon, ob Activity2 ausgef�hrt wird&\ &\textit{boundary-crossing}-Restriktionen werden verletzt\\
Links werden auch bei DPE erfasst&\ &
\end{tabular}
%\caption{Vergleich der Ans�tze}
\label{Vergleichstabelle}
\end{table} 

\underline{Variante 2.}\\
Ziel dieses Ansatzes ist, die boundary-crossing restriction einzuhalten.
Bei dieser Variante wird die Aktivit�t, die das Ziel eines oder mehreren Links ist in eine flow-Aktivit�t eingeschlossen. Durch die Platzierung der Anweisungen in derselben flow-Umgebung wird erreicht, dass die \textit{boundary-crossing}-Restriktionen durch die neu-hinzugef�gten Links nicht verletzt werden k�nnen.
Die Konstruktion der neuen Links erfolgt nach derselben Schema (Varainte 1).  
In der Tabelle werden die Vortele und Nachteile dieser Variante gegen�bergestellt.
\begin{table}[h!]
\begin{tabular}{p{6cm}p{0.5cm}p{6cm}}
\textbf{Vorteile}&\ &\textbf{Nachteile}\\[0.1cm]
\hline \\
\textit{boundary-crossing}-Restriktionen werden eingehalten&\ &Die Links k�nnen nur dann geloggt werden, wenn die Activity3 ausgef�hrt werden kann
\\
Links werden auch bei DPE erfasst&\ &
\\
\end{tabular}
%\caption{Vergleich der Ans�tze}
\label{Vergleichstabelle}
\end{table} 
Der Nachteil kommt zum Tragen, wenn Activity2 zum Beispiel aufgrund eines Fehlers nicht ausgef�hrt wird, die Links wurden aber bereits aktiviert und k�nnten damit geloggt werden.

\underline{Variant3.}\\
Bei dieser Variante wird die Sourceaktivit�t des Links durch eine \textit{flow}-Umgebung umschlossen. Einerseits sorgt die Platzierung der Logging-Anweisungen in derselben \textit{flow}-Umgebung f�r die Einhaltung der \textit{boundary-crossing}-Restriktionen. Andererseits kann das Logging auch stattfinden, wenn die Zielaktivit�t nicht ausgef�hrt wird. Da der Status der Links normalerweise nach der Ausf�hrung der Quellaktivit�t erfolgt, sind die Logging-Aktivit�ten zu diesem Zeitpunkt startbereit und k�nnen gleich nach dem Setzen des Status ihre Aufgabe ausf�hren.

In einer Situation funktioniert auch diese Variante nicht. Befindet sich die Quellaktivit�t zum Beispiel in einem Zweig einer \textit{If}-Aktivit�t, so kann es vorkommen, dass dieser Zweig nicht aktiviert wird. Die Logging-Aktivit�ten k�nnen in dieser Konstellation genauso wie die Quellaktivit�t nicht ausgef�hrt werden. Damit kann die Propagation des \textit{false}-Status auf allen ausgehenden Links nicht registriert werden.

\begin{table}[h!]
\begin{tabular}{p{6cm}p{0.5cm}p{6cm}}
\textbf{Vorteile}&\ &\textbf{Nachteile}\\[0.1cm]
\hline \\
\textit{boundary-crossing}-Restriktionen werden eingehalten&\ &Links  bei DPE k�nnen nicht immer erfasst
\\
Logging kann stattfinden, unabh�ngig davon, ob Activity2 ausgef�hrt wird&\ &
\\
\end{tabular}
%\caption{Vergleich der Ans�tze}
\label{Vergleichstabelle}
\end{table} 

Als L�sung f�r die Abdeckungsmessung der Links eignet sich diese Variante am Besten. Um Konsistenz und Vollst�ndigkeit der Abdeckungsangaben f�r alle BPEL-Prozesse zu gew�hrleisten, werden die DPE bei der Bestimmung der Zweigabdeckung nicht ber�cksichtigt.

\textbf{\textit{ForEach}-Aktivit�t.}

otherwise oder else einf�gen

sequenzen Einf�gen

 Flow: beim Fehler.
 
 \subsection{Fault Handler Abdeckung}

\subsection{Compensation Handler Abdeckung}
\chapter{Design und Implementierung}
In disem Abschnitt werden zuerst zwei M�glichkeiten f�r die Integration der Messung der Testabdeckung in das Framework vorgestellt. Nach einem Vergleich der beiden Ans�tze wird einen als L�sung ausgew�hlt

Die Aufgabe 

die SOAP-Nachrichten, die an Coverage Receiver gerichtet sind m�ssen dekodiert werden und  an den Servece (Mock) weitergeleitet werden.  

Ansatz1
Es kann ein spezieller Partner Track f�r den Service definiert und zu jedem TestCase zugeordnet werden. Zu jedem TestCase kann dann ein Thread gestartet werden, der die Coverage Nachrichten empf�ngt. Es reicht aus jedem TestCase aus. Es muss die Logik f�r den Manager der Marken implementiert werden. Daf�r kann entweder die receive Aktivit�t mehrmals hintereinander geschaltet werden, oder eine neue Aktivit�t, die alle Nachrichten akzeptiert.



 Das Framework dekodiert und leitet weiter , man braucht sich keine Gedanken zu machen. 

nachteil gr��ere Eingriffe in das Framework und an vielen Stellen:
\begin{itemize}
	\item beim Start jedes Testcases muss ein zus�tzliches Thread gestartet werden
	\item beim shutdown unber�cksichtigt 
	\item beim Auswerten des Ergebnissen muss dieser Thread unber�cksichtigt bleiben 
\end{itemize}



Ansatz2

Direkt nachdem Empfangen der Nachricht wird diese an den Receiver weitergeleitet. Dieser muss sich um die Dekodierung der SOAP Nachrichten k�mmern


Dadurch dass beim ersten Ansatz die ganze Funktionalit�t  der Kommunikation (der �bermittlung und Dekodierung der Nachrichten) durch das Framework erledigt werden erscheint diese M�glichkeit als eleganter. Aber Velangsamen





 
 \begin{figure}[htbp]
	\centering
		\includegraphics[width=0.9\textwidth]{bilder/bpelunit-architecture-with-coverage.pdf}
		\caption{Kontrollfluss eines BPEL Prozesses}
	\label{fig:ExamlpleBPELProzess}
\end{figure}

BPELUnit Framework wird erweitert durch hinzuf�gen einer Phase vor dem tats�chlichen Testen. In dieser Phase wird der BPEL Prozess annotiert , invoke-Aktivit�ten werdenhnzugef�gt. Diese Aktivit�teb sind Markierungen f�r jede Aktivit�t und jede M�glichkeit des Kontrollflusses. Die Marken haben eindeutige Identifizierungen, die verwendet werden umd die Relation zwischen Marken und Aktivit�ten bzw. dem Kontrollflu� 

Zur Laufzeit diese Aktivit��ten rufen ein Web Service des BPELUnit Frameworks . Bei diesen Aufrufen werden die Marken transportiert. , die signalisieren den Asf�hrungspfad zum BPELUnit Framework. Diesse Marken werden auf den BPEL Prozess abgebildet. Nach der Ausf�hrung k�nnen mit Hilfe der Marken und statischen Informationen �ber BPEL Prozess die Abdeckungsmetriken berechnet werden. Es ist ein neuer Modul hinzugekommen, der alle Aufgaben, die mit der Messung der Abdeckung zu tun haben, managt. So enth�lt dieser Modul einen Instrumentierer der den BPEL file analysiert und Annotationen hinzuf�gt. Der alte File wird durch den neuen ersetzt.  Zus�tzlich ist ein Web Service f�r das Empfangen von Marken in diesem Modul enhalten
\section{Zusammenfassung}
In diesem Kapitel wurden einige Aspekte des Designs und der Implementierung der Komponente f�r die Ermittlung der Testabdeckung vorgestellt. Dabei wurden die  Designentscheidungen erkl�rt und begr�ndet sowie die wichtigen Schnittstellen und Klassen besprochen. 
Anschlie�end wurde die Integration  in das BPELUnit-Framework und die entsprechende Erweiterungen der Clients beschrieben. 
%Ein BPEL-Prozess besteht aus der Gesch�ftslogik (definiert in BPEL), Servicebeschreibung (WSDL) und optional aus weiteren Datentypen (XML Schema). Diese Informationen werden zusammengefasst und in eine BPEL Engine deployt. Die Engine ist unter anderem daf�r verantwortlich, die \textit{Endpoints} der Partner des BPEL Prozesses zu spezifizieren. F�r jeden Link muss bekannt sein, welche WSDL service und port und welche Adresse f�r die Kommunikation benutzt werden sollen. Diese Information wird in so genannten Deployment descriptoren festgelegt.
   Das Deployment und die Syntax der Deploymentdescriptoren ist nicht spezifiziert.



F�r das Testen, wie bereits erl�utert, muss BPEL-Prozess in eine Engine deployt werden. Die BPEL-Dateien mit  den Deploymentinformation werden in einem Archive zusammengefasst und an die BPEL Engine �bergeben. Damit der Web Service f�r den BPEL-Prozess erreichbar ist, muss die WSDL Beschreibung des Web Services in dem Archive enthalten sein. Au�erdem muss f�r die Kommunikation mit diesem Web service ein PartnerLink definiert. F�r diesen Link muss in dem Deployment descriptor die WSDL service und port sowie die konkrete Adresse (in diesem Fall eine feste Adresse auf dem lokalen Host) festgelegt werden.
   Das Deployment und die Syntax der Deploymentdescriptoren ist nicht spezifiziert.

PARTNERLINKTYPE wird in der WSDL Platziert

 Hierzu m�ssen von der jeweiligen BPEL Engine abh�ngige Deployment-Informationen bereitgestellt werden. 

In disem Abschnitt werden zuerst zwei M�glichkeiten f�r die Integration der Messung der Testabdeckung in das Framework vorgestellt. Nach einem Vergleich der beiden Ans�tze wird einen als L�sung ausgew�hlt

Die Aufgabe 

die SOAP-Nachrichten, die an Coverage Receiver gerichtet sind m�ssen dekodiert werden und  an den Servece (Mock) weitergeleitet werden.  

Ansatz1
Es kann ein spezieller Partner Track f�r den Service definiert und zu jedem TestCase zugeordnet werden. Zu jedem TestCase kann dann ein Thread gestartet werden, der die Coverage Nachrichten empf�ngt. Es reicht aus jedem TestCase aus. Es muss die Logik f�r den Manager der Marken implementiert werden. Daf�r kann entweder die receive Aktivit�t mehrmals hintereinander geschaltet werden, oder eine neue Aktivit�t, die alle Nachrichten akzeptiert.



 Das Framework dekodiert und leitet weiter , man braucht sich keine Gedanken zu machen. 

nachteil gr��ere Eingriffe in das Framework und an vielen Stellen:
\begin{itemize}
	\item beim Start jedes Testcases muss ein zus�tzliches Thread gestartet werden
	\item beim shutdown unber�cksichtigt 
	\item beim Auswerten des Ergebnissen muss dieser Thread unber�cksichtigt bleiben 
\end{itemize}



Ansatz2

Direkt nachdem Empfangen der Nachricht wird diese an den Receiver weitergeleitet. Dieser muss sich um die Dekodierung der SOAP Nachrichten k�mmern


Dadurch dass beim ersten Ansatz die ganze Funktionalit�t  der Kommunikation (der �bermittlung und Dekodierung der Nachrichten) durch das Framework erledigt werden erscheint diese M�glichkeit als eleganter. Aber Velangsamen
\chapter{Beispiel}\label{beispiele}
In diesem Abschnitt werden die Metriken an einem Besipiel demonstriert. Als Beispiel dient ein sehr einfacher BPEL-Prozess zur Verarbeitung von Bestellungen (Abbildung \ref{fig:Bespielprozess}).
\begin{figure}[htbp]
	\centering
		\includegraphics[width=0.6\textwidth]{bilder/Beispiel.png}
		\caption{Beispielprozess}
	\label{fig:Bespielprozess}
\end{figure}

Abh�ngig vom Liefertyp (\textit{express} oder \textit{normal}) werden die Bestellungsdaten unterschiedlich verarbeitet. Wenn der Kunde einen besonderen Status (VIP) hat, was mit dem \textit{Customer Identification}-Service festgestellt wird, dann wir der Kunde und sein Handelsvertreter �ber die Bestellung benachrichtigt.
Das wird mit Hilfe des Synchronisationslinks \textit{link\_1} modelliert, der eine explizite \textit{transition condition} aufweist: \textit{customerType='VIP'}.
F�r den Fehlerfall ist ein einfacher Fault Handler implementiert.

Der Prozess bekommt in einem Testfall eine Bestellung eines \textit{VIP}-Kunden mit folgenden Daten:
\begin{verbatim}
<order>
    <shipment>express</shipment>
    <customerId>12345</customerId>
    <productId>46298</productId>
</order>
\end{verbatim}

Es ist nur wichtig zu wissen, dass der Kunde den VIP-Status hat. Bei der Bearbeitung der Daten wird der mit roten Pfeilen markierte Pfad durchlaufen.
Mit der Erweiterung des BPELUnit-Frameworks kann die Testabdeckung ermittelt werden. Der Befehlszeilen-Client liefert die Testabdeckung im XML-Format:
\begin{verbatim}
<testingCoverage xmlns="http://www.bpelunit.org/schema/coverageResult">
  <FileStatistics filename="bpel/prozess.bpel">
    <statistic name="ActivityCoverage" totalItems="10" testedItems="7" />
    <statistic name="ActivityCoverage: assign" totalItems="5" 
                                                          testedItems="3" />
    <statistic name="ActivityCoverage: invoke" totalItems="3" 
                                                          testedItems="2" />
    <statistic name="ActivityCoverage: receive" totalItems="1" 
                                                          testedItems="1" />
    <statistic name="ActivityCoverage: reply" totalItems="1" 
                                                          testedItems="1" />
    <statistic name="BranchCoverage" totalItems="10" testedItems="8" />
    <statistic name="CompensationHandlerCoverage" totalItems="0" 
                                                          testedItems="0" />
    <statistic name="FaultHandlerCoverage" totalItems="1" testedItems="0" />
    <statistic name="LinkCoverage" totalItems="2" testedItems="1" />
    <statistic name="LinkCoverage: negativLinks" totalItems="1" 
                                                          testedItems="0" />
    <statistic name="LinkCoverage: positivLinks" totalItems="1" 
                                                          testedItems="1" />
  </FileStatistics>
</testingCoverage>
\end{verbatim}


Obwohl es sich um einen sehr einfachen BPEL-Prozess handelt, deckt ein Testfall   bereits 70\% der Basisaktivit�ten und 80\% der Zweige ab. Es wird allerding nur 50\% der Linkabdeckung erreicht und der Fault Handler wird gar nicht getestet. Der Tester kann diese Ergebnisse nutzen, um neue Testf�lle zu definieren und fehlende Bereiche abzudecken.
\chapter{Fallstudie}\label{chap:fallstudie}
Zur �berpr�fung der praktischen Einsetzbarkeit der entwickelten L�sung wurde eine Fallstudie durchgef�hrt. Die Web Service-basierte Anwendung und die zugeh�rigen Tests, die dabei verwendet wurden, sind aus einem studentischen Projekt entstanden. An der Universit�t Hannover am Fachgebiet Software Engineering wurden im Wintersemester 2006/07 das Projekt "`Entwicklung einer Weservice-basierten Anwendung"' durchgef�hrt. 
Innerhalb eines Semesters haben 8 Studenten die Anwendung entworfen, entwickelt und getestet. Aufgrund der eingesetzten Technologien und Werkzeuge, wie BPEL (Kompositionsprache), BPELUnit-Framework (Testwerkzeug) und ActiveBPEL Engine (BPEL-Engine), hat sich das Projekt als Fallstudie f�r diese Arbeit angeboten. 

Entwickelt wurde ein System f�r Fach Gebiet Software Engineering , das den Anmeldeprozess f�r studentische Abschlussarbeiten unterst�tzt. Das System stellt dabei sicher, dass alle wesentliche Bestandteile der Betreuung, wie z.B. Ver�ffentlichung und Anmeldung der Arbeit, Planung von Zwischen- und Endvortr�gen, Bewertung usw., eingehalten werden.
F�r die Fallstudie sind vor allem die entstandenen BPEL-Prozesse und die Tests interessant. 

Zur besseren Einsch�tzung der Gr��e der entstandenen Anwendung und Einordnung dieser Fallstudie werden einige statistischen Daten der BPEL-Prozesses und (BPELUnit-) Tests vorgestellt.  

Der Gesamtprozess besteht aus 12 Teilprozessen. Insgesamt wurden 13 BPEL-Dateien erstellt: 12 f�r Teilprozesse und eine f�r den Hauptprozess, der alle Teilprozesse (als Web Services) zu einer Anwendung verkn�pft. Die Abbildung \ref{fig:beispielprozess} stellt den Zusammenhang grafisch dar.

\begin{figure}[htbp]
	\centering
		\includegraphics[width=\textwidth]{bilder/Fallstudie.png}
		\caption{BPEL-Prozess. }
	\label{fig:beispielprozess}
\end{figure}



Die folgenden Statistiken beziehen sich auf alle 13 BPEL-Dateien und zeigen welche Aktivit�ten bei der Prozessmodellierung verwendet wurden.

Es wurden insgesamt 508 strukturierten Aktivit�ten und Scopes verwendet:
\begin{table}[h!]
\begin{tabular}{p{2cm}p{0.5cm}p{2cm}}
\textbf{Aktivit�t}&\ &\textbf{Anzahl}\\[0.1cm]
\hline \\
flow&\ &237\\
pick&\ &1\\
switch&\ &71\\
while&\ &19\\
secuence&\ &0\footnote{Sequentielle Abl�ufe wurden in Flow-Aktivit�ten mit Synchronisationslinks realisiert.}\\
scope&\ &180\\
\end{tabular}
\end{table}

750 Basisaktivit�ten:
\begin{table}[h!]
\begin{tabular}{p{4cm}p{4cm}p{4cm}p{4cm}}
\textbf{Aktivit�t}&\textbf{Anzahl}&\textbf{Aktivit�t}&\textbf{Anzahl}\\[0.1cm]
\hline \\
assign&387&compensate&0\\
empty&109&invoke&56\\
receive&21&reply&55\\
terminate&14&throw&98\\
wait&10&&\\
\end{tabular}
\end{table}

Zum Testen des Prozesses wurde BPELUnit Framework eingesetzt.
Es wurden zwei Testsuites erstellt:
\begin{itemize}
	\item \textbf{Suite1} testet den gesamten Prozesses. In diesem Fall werden au�schlie�lich die Kommuniaktionsp�artner (Web Services) der Teilprozesse durch Mock's ersetzt. Die zugeh�rige Aufrufe sind in der Abbildung \ref{fig:beispielprozess} durch kurze Pfeile angedeutet. Es werden also nicht einzelne Prozesse (einzelne BPEL-Datei) isoliert und getestet, sondern gleich mehrere zusammenh�ngende Prozesse. Das widerspricht der Gedanke des Unittests und hat Auswirkungen auf Messung der Testabdeckung (siehe weiter unten).
	\item \textbf{Suite2} testet die einzelnen Teilprozesse. In dieser suite werden die einzelnen Prozesse isoliert und getestet, so wie es bei Unittest auch gefordert wurd. Zus�tzlich zu den Partnern, die bei suite1 durch Mock's ersetzt wurde, wurde auch der Hauptprozess, der in diesem Fall Client ist, durch ein Client-Mock ersetzt. 
\end{itemize}

Die Teilprozesse werden sowohl in suite1 als auch in suite2 getestet. Um die gesamte Testabdeckung zu ermitteln, die durch beide suites zusammen erreicht wird, m�ssen zum einen beide Testsuites ausgef�hrt werden und zum anderen dieselbe Marken verwendet werden, um die abgedecktem Prozesselemente in beiden Testl�ufen identifizieren zu k�nnen. Das BPELUnit Framework ist so konzipiert, dass nur eine suite pro Testlauf ausgef�hrt wird und die Testl�ufe voneinander unabh�ngig sind. Die Messung der Testabdeckung erfolgt ebenfalls suite-bezogen. Das heisst, die Abdeckungen, die bei Ausf�hrung der einzelnen Suites erreicht werden, k�nnen ermittelt werden. allerdings kann die gemeinsame Abdeckung nicht bestimmt werden.

Sind alle Tests, die ein Prozess betreffen, in derselben Testsuite organisiert und alle Prozesse mit Hilfe der Mock's einzeln isoliert, so kann die Gesamtabdeckung gemessen werden. Es ist nat�rlich auch m�glich, das BPELUnit Framework so zu erweitern, dass suites in einem Testlauf ausgef�hrt werden, damit die Gesamttestabdeckung ermittelt werden kann.  
Trotzdem kann die praktische Tauglichkeit des Verfahrens und der Implementierung an diesem Beispiel �berpr�ft werden.

\textbf{Instrumentierung}. Bei der Instrumentierung werden die BPEL-Dateien (alle 13) f�r alle in dem Abschnitt \ref{} definierten Metriken mit Marken annotiert. Dabei  werden insgesamt  2857 Marken platziert.

Ergebnisse:

Die Testabdeckung l�sst, wie bereits erl�utert nur f�r die Datei  "`ThesisProzess.bpel"' berechnen. Die Ergebnisse sind in der folgenden Tabelle dargestellt.

Insgesamt wurden f�r beide suites folgende Metriken ermittelt:


\textbf{Zeitmessungen}.


\chapter{Zusammenfassung und Ausblick}
   Visualisierung
   
   Beurteilung des Testqualit�ts ist in der Softwareentwicklung wichtig. Es wird noch Wichtiger, wenn ein kritisches System entwickelt wird. BPEL-Prozess geh�rt normalerweise zu dieser Kategorie. Als Ergebnis, die F�higkeit der Testabdeckungsmessung ist sehr n�tzlich. In dieser Arbeit pr�sentiert Testabdeckungsmetriken, die aus dem klassischen Softwareentwicklung auf BPEL adoptiert wurden und neue BPE-Spezifischen..
   
   Jede Metrik ist so definiert, dass die Tests f�r jeden BPEL-Prozess eine 100\% abdeckung erreichen k�nnen, was theoretisch ptimal ist. F�r die praktische Anwendung wurde BPELUnit Framework erweitert. Die Instrumentierung und Berechnung der Metriken erfolgt transparent beim Ablaufen der Tests.
   Die Erweiterung des BPELUnit Frameworks um die M�glichkeit die gemessene Testabdeckung zu visualisieren, ist der n�chste logische Schritt. Durch die Pr�sentation nicht abgedeckten Strukturen bekommt der Tester Hinweise, welche Tests noch fehlen und damit die M�glichkeit die Tests gezielt zu entwickeln. F�r das Erreichen der gew�nschten oder geforderter Testabdeckung ist diese Vorgehensweise viel effizienter und kann positive Auswirkungen auf die Entwicklungszeit und -Kosten haben.
   
   Anforderungsbasierte Testabdeckung.     
%\subsection{Anforderungen}

Die Semantik darf nicht ge�ndert werden. Die Metriken sollen einzeln gemessen werden.


Die prim�re Aufgabe der Code-Coverage-Analyse liegt darin, die im Code durch Tests abgedeckten Bereiche  zu identifizieren und einige Kennzahlen der Abdeckungen zu ermitteln
(z.B. Anzahl der Durchl�ufe einer Codezeile). Daher werden bei der Code-Coverage-Analyse w�hrend der Testausf�hrung
verschiedene Messungen durchgef�hrt, die jeweils Informationen �ber eine bestimmte Art der Abdeckung sammeln. Im
Folgenden werden die wichtigsten Messmetriken bei der Code-Coverage-Analyse aufgelistet.
wieviel des zu untersuchenden Codes tats�chlichdurch die Tests ausgef�hrt wurde. Es wird auch oft der Begriff \textit{Testabdeckung} verwendet.



\subsubsection{Normaler Kontrollfluss}
Den Kontrollfluss bestimmen in BPEL, wie bereits mehrmals erw�hnt, die strukturierten Aktivit�ten. In den folgenden Bildern sind die entsprechenden Kontrollflussgraphen abgebildet. Die grauen Elementen deuten an, wie die einzelnen Graphen in den Gesamtfluss eingebunden werden. F�r die Zweigabdeckung relevanten Kanten sind schwarz dargestellt.
\begin{figure}[htbp!]
	\centering
		\includegraphics[width=0.65\textwidth]{bilder/h1.png}
	\label{fig:h1}
\end{figure}
\begin{figure}[htbp!]
	\centering
		\includegraphics[width=0.65\textwidth]{bilder/h2.png}
	\label{fig:h2}
\end{figure}

Bei den Aktivit�ten, deren Graphen in der Abbildung vorhanden sind, entspricht der Kontrollfluss den Konstrukten, die aus vielen Programmiersprachen bekannt sind. Die Restlichen sind BPEL-spezifisch und m�ssen gesondert betrachtet werden.


\parpic(6.5cm,5cm)[r]{\includegraphics[width=0.30\textwidth]{bilder/pick.png}} 
\textbf{\textit{Pick}-Aktivit�t.}\\
Bei dieser strukturierenden Aktivit�t wird ein Zweig aus mehreren anhand eines Ereignisses (Nachricht oder Alarm) ausgew�hlt. Wurde ein zutreffendes Ereignis empfangen, so wird die zugeh�rige Aktivit�t ausgef�hrt und alle nachfolgenden Ereignisse verworfen. Aus der Sicht des Kontrollflusses stimmt dieses Verhalten mit dem einer IF-\-Akti\-vi\-t�t �berein. Deswegen werden diese beiden Aktivit�ten im Kontrollfluss durch gleiche Graphen repr�sentiert.

\textbf{\textit{Flow}-Aktivit�t.}\\
  Mit der \textit{Flow}-Aktivit�t wird in BPEL paralleler Ablauf mehrerer Aktivit�ten realisiert. Au�erdem gibt es mit dem Link-Konzept, ein m�chtiges Instrument, das erm�glicht, Syn\-chro\-ni\-sa\-tions\-ab\-h�n\-gig\-keiten aufzustellen und damit komplexe Abl�ufe zu realisieren. Die rechte Abbildung zeigt den zugeh�rigen Kontrollfluss. 
  \parpic(6.4cm,3.6cm)[r]{\includegraphics[width=0.33\textwidth]{bilder/flow.png}}
  Die Links (gestrichelte Pfeile) unterscheiden sich semantisch von den normalen Kanten des Kontrollflussgraphes.
  Wie es bereits im Abschnitt .. geschildert wurde, haben die Links trotzdem einen erheblichen Einfluss auf den Kontrollfluss und sind damit f�r die Zweigabdeckung relevant. 

  Aus der Sicht der Abdeckungsmessung ist die Tatsache besonders, dass der Status der Links drei Werte annehmen kann: \textit{true}, \textit{false} und \textit{unset}  Die normalen Kontrollflusskanten k�nnen dagegen nur zwei Werte annehmen: entweder \textit{aktiviert} oder \textit{nicht aktiviert}(\textit{unset}). F�r die Links bedeutet das, dass \textit{true}- und \textit{false}-Status bei der Messung der Abdeckung erfasst werden m�ssen. 

\textbf{\textit{ForEach}-Aktivit�t.}
Die \textit{ForEach}-Aktivit�t sorgt f�r mehrmaliges Ausf�hren des enthaltenen Scopes. Die Anzahl der Ausf�hrungen wird durch Start- und Stopvariablen vorgegeben. Die Besonderheit dabei ist, dass die Ausf�hrung entweder sequenziell oder parallel erfolgen kann. Die Ausf�hrungsart wird durch den  Wert der Variable \textit{parallel} (\textit{yes}/\textit{no}) geregelt. 

Dieses Verhalten kann durch zwei Graphen dargestellt werden. Die Abbildung ... repr�sentiert sequentielle Ausf�hrung, die einer for-Schleife aus den konventionellen Programmiersprachen entspricht. Der Graph daneben repr�sentiert eine parallele Ausf�hrung.
\begin{figure}[h!]
	\centering
		\includegraphics[width=0.73\textwidth]{bilder/forEeach2.png}
	\label{fig:forEeach2}
\end{figure}\\

W�hrend der regul�re Kontrollfluss des Programms den Steuerfluss f�r alle regul�re Situationen vorgibt, gibt es in BPEL ein (\textit{FaultHandler})-Kontrollfluss, der f�r die Behandlung zur Laufzeit aufgetretener Fehler vorgesehen ist. Inwiefern dieses Verhalten in die Zweigabdeckung mit einflie�en kann, wird im n�chsten Abschnitt diskutiert. 
 In den folgenden Abschnitten werden die Metriken vorgestellt, die den Grad der durch die Tests abgedeckten Fault- bzw. Compensation Handler wiedergeben.
\\



Anweisungs�uberdeckung wird im allgemeinen nicht als hinreichendes Kriterium
f�ur die Vollst�andigkeit eines Tests betrachtet, ihre Auswertung ergibt sich als
Nebenprodukt der Auswertung umfassenderer �Uberdeckungskriterien. Empirische
Untersuchungen zeigen eine Fehlererkennungsrate von 15\% bis unter 20\% f�ur
reine Anweisungs�uberdeckung. Sie liefert allerdings in allen Untersuchungen h�ohere
Prozents�atze als die statische Analyse des Quelltextes.
Die Kanten�uberdeckung gilt als Standard bei kontrollflussbasierten �Uberdeckungsma�en
und wird oft als Kriterium an Tests gestellt. Sie wird als leistungsf
�ahiger Test vor allem zum Auffinden von logischen Fehlern betrachtet. Untersuchungen
bez�uglich der Leistungsf�ahigkeit streuen in einem weiten Bereich von
20\% bis 70\% gefundener Fehler. Zumeist liegen die Prozents�atze im unteren Drittel
dieses Intervalls. Studien, die ebenfalls Anweisungs�uberdeckung betrachten, zeigen
f�ur Zweig�uberdeckung eine Fehlererkennungsrate, die um 50% bis 100% �uber der
der Anweisungs�uberdeckung liegt.
Bedingungs�uberdeckung findet vor allem bei Systemen mit komplexer Verarbeitungslogik
Verwendung, die in der Regel kompliziert aufgebaute Bedingungen
verwende.



D

Erfolgsquote:
..Anweisungs�berdeckungstest 18%
..Zeig�berdeckungstest 32%
..Bedingungs�berdeckungstest ?
..Pfad�berdeckungstest 65%


Beim Erstellen von Tracedaten unterscheidet man zwischen Methoden, die es erfordern
den bestehenden Source- bzw. Objektcode zu instrumentieren und Methoden,
die automatisch Traceausgaben erzeugen. Moderne Entwicklungsumgebungen bieten
meist Techniken an, um Traceausgaben ohne Sourcecode-�Anderung zu erstellen.


Die Testabdeckungsanalyse gibt an, in welchem Umfang die durchgef�hrten Tests die zu testende Software erfasst haben.



Bis jetzt wurde Instrumentierungsansatz allgemein betrachtet, ohne auf eine bestimmte Sprache einzugehen. Jetzt werden BPEL-spezifische Einschr�nkungen untersucht. Als Erstes d�rfen nur BPEL-Aktivit�ten eingef�gt werden, sonst kann der BPEL-Prozess von der BPEL-Engine nicht ausgef�hrt werden. Um Protokollierung zu realisieren, muss eine Kommunikation von der Ausf�hrungsumgebung nach au�en stattfinden. Die einzige daf�r geeignete Aktivit�t ist \textit{invoke}. Au�erdem muss ein (lokales) Web Service eingerichtet werden, der f�r das Protokollieren zust�ndig ist. Bevor zu tief in die Details eingestiegen wird, werden zuerst die anderen L�sungsm�glichkeiten vorgestellt und besprochen.	
	
	
	\begin{itemize}
\item BPEL-Prozesse werden in speziellen Ablaufumgebungen (\textit{BPEL-Engine}) ausgef�hrt.
\item BPEL-Prozesse k�nnen nur mit Web Services (\textit{partner}) kommunizieren.
\begin{itemize}
	\item Die entsprechenden WSDL-Beschreibungen sind notwendig.
	\item Die \textit{partner link types} und \textit{partner liks} m�ssen definiert werden
\end{itemize}
	\item Businesslogik wird in BPEL-Prozesse durch BPEL-Aktivit�ten (s. Abschnitt \ref{}) realisiert. 
\end{itemize}

\begin{figure}
	\centering
		\includegraphics[width=0.90\textwidth]{bilder/loggingservice.pdf}
	\label{fig:loggingservice}
\end{figure}



\newpage
\begin{figure}[h!]
	\includegraphics{bilder/StrukturedAktvities/sequence.pdf}
	\label{fig:sequence}
\end{figure}
\begin{figure}[h!]
	\includegraphics{bilder/StrukturedAktvities/pick.pdf}
	\label{fig:pick}
\end{figure}
\begin{figure}[h!]
	\includegraphics{bilder/StrukturedAktvities/while.pdf}
	\label{fig:while}
\end{figure}
\begin{figure}[h!]
	\includegraphics{bilder/StrukturedAktvities/flow.pdf}
	\label{fig:flow}
\end{figure}


Protokolliert man eine Aktivit�t, die das Ziel eines oder mehreren Links ist, davor, so kann es sein, dass sie aufgrund der transition- bzw. joinCondition gar nicht ausgef�hrt wird.
      \lstinline|print "hello world"|
      
      
      \lstset{emph={joinCondition,target }, emphstyle=\color{blue}}
      \begin{lstlisting}[caption=Beispielcode]{Name}
<flow>
  <links>
    <link name="CtoD"/>
  </links>
  <receive name="C" ...>
    <source linkName="CtoD"/>
  </receive>
  <invoke ... joinCondition=...>
    <target linkName="CtoD"/>
  </invoke>
</flow>      \end{lstlisting}

\lstset{emph={[2]sequence}, emphstyle=[2]\color{red}}
      \begin{lstlisting}[caption=Beispielcode][firstnumber=1]{Name}
<flow>
  <links>
    <link name="CtoD"/>
  </links>
  <receive name="C" ...>
    <source linkName="CtoD"/>
  </receive>
  <sequence joinCondition=...>
    <target linkName="CtoD"/>
    <!--Logging-->
    <invoke .../>
    <!--Logging-->
  </sequence>
</flow>      \end{lstlisting}

F�r die Messung heisst es, w�hrend es bei normalen Kanten ausreichend ist, die aktivierten Kanten zu erfassen, muss es bei den Links zus�tzlich zwischen \textit{true}- und \textit{false}-Status unterschieden werden. 


Drei Elemente bilden die Grundlage von Web Services:
Die �bermittlung: Die Kommunikation zwischen Service-Anbieter, Service-Konsument und ggf. Service-Verzeichnisdienst erfordert die �bermittlung von Nachrichten. 
Es hat sich SOAP durchgesetzt.  Dieser von W3C verwaltete Standard basiert auf XML und liegt in der Version 1.2 vor.



  Die
Nutzung etablierter und standardisierter Internetstandards f�ur Transport undDatenstrukturierung stellt dabei die Interoperabilit�at bei der Kommunikation
zwischen Softwarekomponenten sicher.
Die Entwicklung der eingesetzten Standards wird vom W3C und anderen
Standardisierungsgremien geleitet, wobei die Aufgabe haupts�achlich in der Definition
und Pflege bereits etablierter Standards der Web-Servicearchitektur
besteht. Die Web-Servicearchitektur des W3C ist dabei kein fest vorgegebener
Rahmen, sondern definiert lediglich die minimalen Anforderungen welche eine
Softwarekomponente erf�ullen mu�, um als Web-Service zu gelten.
Trotz der Offenheit der Web-Servicearchitektur haben sich die Beschreibunsstandards
Simple Object Access Protocoll (SOAP), Web-Service Description
Language (WSDL) und Universal Description, Discovery and Integration
(UDDI) als Kern dieser Architektur etabliert.
Diese Standards basieren auf XML und XML-Schema, so da� XML-f�ahige
Applikationen Dokumente welche auf diesen Standards beruhen verarbeiten
k�onnen. Zusammen mit der Nutzung von etablierten Netzwerk�ubertragungsprotokollen
ergeben sich somit gute Kommunikationseigenschaften.
Mit SOAP steht dabei eine Beschreibungssprache zur strukturierten Daten-
�ubertragung zur Verf�ugung. Es wird heute haupts�achlich zur strukturierten
Beschreibung von Nachrichten und ihren Parametern eingesetzt. WSDL ist
ein plattformunabh�angiger Standard zur Beschreibung der Schnittstellen von
Softwarekomponenten. UDDI ist ein Beschreibungsstandard zur Kategorisierung
und Beschreibung von Eintr�agen in einem Verzeichnissystem f�r Dienste
im Allgemeinen. UDDI erlaubt es dabei neben allgemeinen Informationen zur
ver�offentlichenden Institution, technische Daten zur Beschreibung der angebotenen
Dienste zu publizieren. Diese drei Standards werden in den folgenden
Abschnitten eingehend behandelt.Den Kern der Architektur bilden die zentral dargestellten Komponenten zum
Nachrichtenaustausch (Messages), zur Schnittstellenbeschreibung von Services
(Descriptions) und zur Komposition von Services zu komplexen Gesch�aftsprozessen
(Processes). Aus der Abbildung ist ersichtlich, da� die genannten Komponenten
auf XML und XML-Schema basieren, was eine plattform�ubergreifende
Nutzbarkeit garantiert.
Die vom W3C definierte Web-Servicearchitektur soll in erster Linie die Interoperabilit
�at der angebotenen Dienste sicherstellen. Details zur Implementierung
oder zu Transport und Strukturierung der Daten werden dabei bewu�t
nicht definiert. Die Unterst�tzung des Nachrichtenaustausches von Softwarekomponenten
�uber Netzwerke charakterisieren Web-Services als Middleware f�r
verteilte Systeme. Wichtigste Eigenschaften von Web-Services sind dabei ihre
lose Kopplung, ihre Programmierbarkeit �ber selbstbeschreibende Schnittstellen,
sowie Orts- und Protokolltransparenz.


Die Beschreibung:  Der Service und die von ihm zur Verf�gung gestellten Methoden m�ssen beschrieben werde, damit der Service-Konsument wei�, was der Dienst ihm bietet. WSDL ist der Standard. wird wie SOAP von W3C verwaltet. Version 1.2 ...

Der Verzeichnisdienst:

Die Standards f�r diese drei Grundelemente bilden das Fundament f�r Web Services.

\begin{itemize}
	\item SOAP ein standardisiertes, XML-basiertes
Protokoll zum Verpacken von Nachrichten, die zwischen Applikationen ausgetauscht
werden. setzt SOAP
auf die Netzwerk- und Transportschichten auf. Es ist also irrelevant, welche
Transportmechanismen f�ur den eigentlichen Versand verwendet werden.Die gebr�auchlichste Form des Austausches von SOAP-Nachrichten ist die �Ubertragung
�uber HTTP.
\item UDDI
dient zur Lokalisierung und Ver�ffentlichung
von Web-Services im Internet
UDDI = Register f�r Diensteund ihre Beschreibungen+ Suchmethoden+ Publishingmethoden
UDDI-Daten enthalten Kontakt-Informationen, Listen von Business Services und Infos, wie einService via Protokoll angesprochen werden kann
Um einen Web-Service im Internet zu finden, ist ein Verzeichnisdienst notwendig.
F�ur diese Zwecke wurde der Universal Description, Discovery and Integration-
Standard geschaffen. UDDI bietet Standardfunktionen zum Klassifizieren,
Katalogisieren und Verwalten von Daten und Metadaten �uber Web-
Services, so da� diese einfach gefunden und verwendet werden k�onnen.
\item WSDL (Web Services Defnition Language) ist eine funktionale, XML-basierte Beschreibungssprache
f�r die Schnittstellen eines Web Services.
\end{itemize}


Web S basieren, zumindestens in der Grundfunktionalit�t (Nachrichten�bermittlung, Beschreibung und Verzeichnisdienste) auf anerkannten standards. 
Web Services k�nnen folgenderma�en definiert werden:


Drei Grundlegende Standards.
SOAP
Die SOAP-Spezifikation definiert ein Rahmenwerk f�r die �bertragung von XML-Nachrichten �ber ein standardisiertes und anerkanntes Transportprotokoll.
  Nachrichtenaustausch
Anwendungen k�nnen Nachrichten (z.B. Aktienkurse, Warenbestellungen) in XML codieren undviaSOAP-Dokumenten austauschen
Durch XML liefert SOAP eineplattformunabh�ngigesProtokoll
f�r den Nachrichtenaustausch
SOAP stellt Konventionen f�r eine standardisierte Darstellungsweiseder Informationen in XML f�r den Datenaustausch in heterogenen Systemen zur Verf�gung
SOAP liefert zwei Ans�tze f�r den Nachrichtenaustausch:
Remote Procedure Calls
Funktionsaufrufe entfernter Prozeduren im Sinne verteilter Architekturen
ElectronicDocument Interchange
Dokument-basiertes SOAP, bei dem fachliche Dokumente z.B. Steuererkl�rung, Warenbestellungen ausgetauscht werden
 Bei der Entwicklung legte man
gro�en Wert auf Erweiterbarkeit, Offenheit und Heterogenit�t des Protokolls.
SOAP besteht im Wesentlichen aus vier Teilen:
 der Spezifikation f�r einen Umschlag (Envelope). F�r ihn ist definiert, was in einer
Nachricht enthalten ist, von wem es wie verarbeitet werden soll und ob einzelne Daten
optional sind oder enthalten sein m�ssen.
 ein Satz von Regeln, die vorschreiben, wie Daten in dem Umschlag repr�sentiert werden
sollen (Serialisierung)
 eine Konvention, um Remote Procedure Calls und eventuelle Antworten auf diese zu
repr�sentieren
 eine Vorschrift, wie SOAP via HTTP �bertragen werden soll. Dank des erweiterbaren
Entwurfs k�nnen aber beliebige Transportprotokolle verwendet werden.
Eine SOAP-Nachricht muss nicht zwangsl�ufig direkt vom Sender (initial sender) an den
Empf�nger (ultimate receiver) geschickt werden, sondern kann �ber mehrere Relay-Stationen
(intermediaries) geleitet werden. Den Weg, dem sie dabei folgt, nennt man Message Path.
SOAP macht jedoch keine Angaben dazu, in welcher Reihenfolge die Stationen abgearbeitet
werden. Es gibt aber einige SOAP-Erweiterungen, die diese L�cke zu f�llen versuchen.
Der SOAP-spezifische Teil einer Nachricht wird �ber die Zuordnung zu bestimmten Namensr�umen
erkennbar.
Das Simple Object Access Protocol bietet ein standardisiertes, XML-basiertes
Protokoll zum Verpacken von Nachrichten, die zwischen Applikationen ausgetauscht
werden. Es bietet einen Mechanismus zum Austausch typisierter Daten
in einem dezentralen, verteilten Umfeld und l�a�t sich leicht mit verschiedenen
�Ubertragungsprotokollen nutzen. Die Grundidee ist, da� zwei Anwendungen
Informationen �ubertragen k�onnen, ohne da� auf das Betriebssystem, die Programmiersprache
oder andere technische Implementierungsdetails R�ucksicht genommen
werden mu�. F�ur diesen Informationsaustausch wird dabei nichts weiter
ben�otigt als eine Nachricht, die so codiert ist, da� beide Anwendungen sie
verstehen k�onnen. Eine versendete Nachricht kann dabei jede beliebige Information
enthalten: eine Suchanfrage an eine Suchmaschine oder ein Warenhaus,
die Suche oder Buchung eines Fluges oder einer Zugfahrt, die Abfrage eines
Aktienkurses, eine Warenbestellung etc.
Urspr�unglich begann Microsoft 1998 mit der Entwicklung von SOAP. Sp�ater
beteiligten sich andere Firmen wie IBM und SAP an der Standardisierung. Seit
September 2000 wird die Weiterentwicklung des Standards durch eine Arbeitsgruppe
des W3C koordiniert, wodurch Offenheit und Herstellerunabh�angigkeit
gew�ahrleistet ist. Die aktuelle Version ist SOAP 1.2 ([SOAP Pt. 1] und
[SOAP Pt. 2]).Als standardisiertes Protokoll zum Verpacken von Nachrichten 

WSDL 

Web Service Technologie Stack
basieren Web Services auf einer Reihe zusammen-
h�angender Technologien.


XML Schema: Der Standard beschreibt die Struktur
oder das Format von XML-Daten. Auf diese Weise
kann ein Programm, welches die Daten zur
Bearbeitung erh�lt, erkennen, ob es sich dabei
beispielsweise um Zahlenwerte oder
Aufz�hlungstypen handelt.
SOAP (Simple Access Protocol): Dieser Standard
sorgt f�r die zuverl�ssige �bermittlung elektronischer
Nachrichten zwischen Gesch�ftsanwendungen �ber
das Internet. SOAP wurde von IBM, Lotus und
Microsoft entwickelt.
WSDL (Web Services Describtion Language):
Dieser Standard beschreibt, wie Programme �ber das
Internet oder andere Netze miteinander
kommunizieren. WSDL dient der eindeutigen
Definition von Nachrichtenformaten und Protokollen.
WSDL wurde von IBM und Microsoft entwickelt.
UDDI (Universal Description Discovery and
Integration): Mit dem Standard k�nnen
Unternehmen eigene technische Spezifikationen �ber
E-Business-Regeln untereinander austauschen.
UDDI wurde von IBM, Microsoft und Ariba entwickelt.


  Die SOAP-Spezifikation definiert ein Rahmenwerk f�r die �bertragung von XML-Nachrichten �ber ein standardisiertes und anerkanntes Transportprotokoll.
  
  WSDL ist ein XML-Dialekt, der fehlende, aber oft notwendigen Metadaten f�r Web Services beschreibt. WSDL ist jedoch nicht allein an SOAP gebunden. Es gibt eine Vielzahl von vordefinierten Bindungen, zu denen auch SOAP 1.1 z�hlt. WSDL h�lt sich an eine abstrakte Beschreibung von Web Services. WSDL verfolgt somit unterschiedliche Ziele:

	Erweiterbarkeit: Neue Codierungsvarianten und Transportmechanismen k�nnen definiert werden.
 	Abstrakte Definitionen: Nachrichten und Dienste werden abstrakt beschrieben, k�nnen aber auf eine oder mehrere konkrete Implementierungen abgebildet werden.
 	Wiederverwendung von Definitionen: Vorhandene Definitionen von Endpunkten k�nnen f�r neue Definitionen wiederverwendet werden.


Grundlegendes
   Eine WSDL-Datei besteht aus mehreren XML-Elementen und entsprechenden Kindelementen. Abbildung 2 enth�lt die grafische Darstellung der einzelnen Elemente von WSDL. Diese werden in den folgenden Abschnitten n�her erl�utert. Eine WSDL-Beschreibung beginnt immer mit dem Element definitions. Direkt in diesem Element werden die im Dokument verwendeten Namensr�ume als Attribute definiert. S�mtliche anderen in der WSDL-Beschreibung vorkommenden Elemente sind Kinder dieses Wurzelelements.
   





aaaaaaaaaaaaaaaaa












Wichtig:
Web Service realisierung von SOA

basiert auf mehreren Standards

Unterst�tzt von gro�en Insustrie .. IBM Microsoft Sun



Definition von W3C


Daraus: Web Service Software anwendung, beliebiger Gr��e

Selbstbeschreibend, XML

Web Services interagieren mit mit anderen Software agenten, nicht mit Menschen

Web intenet basierte Protokolle


Die wcihtigsten Standards:


SOAP WSDl UDDI

Stack

Diese Arbeit hat mit Web Service Composition (speziell BPEL) zu tun ganz oben auf dem Stack 
BPEL setzt auf WSDl und SOAP.

Web Services sind nicht zur menschlichen Informationsverarbeitung geschaffen,
sondern f�r die Kommunikation zwischen Anwendungen und Computern.

Web Services sind durch begleitende Metadaten (Beschreibungsmerkmale:
Name, Version, Beschreibung, etc.) selbstbeschreibend und bieten somit
zur Laufzeit Auswertungsm�glichkeiten f�r weitere Services. Web Services
sind in sich gekapselte und abgeschlossene, unabh�ngige Anwendungen, die
f�r eine genau definierte Aufgabe geschaffen wurden.
Web Services sind lose gekoppelt. Die Kommunikation zwischen den Anwendungen
geschieht ausschlie�lich �ber den Nachrichtenaustausch, wobei
dem Nutzer Implementierungsdetails verborgen bleiben. Es erfolgt keine Festlegung
bez�glich konkreter Implementierung und verwendetem Kommunikationsprotokoll.
Web Service sind ortstransparent und k�nnen somit jederzeit von jedem
Ort aus angesprochen und verwendet werden.
Web Services sind protokolltransparent und unterst�tzen mehrere Protokolle
f�r die �bertragung und Verarbeitung von Nachrichten.



Eine Defnition zu Web Services findet man bei Pera und Rintelmann.
Defnition: Der Begrif Service oder Dienst meint eine Software-Einheit, die in sich abgeschlossen
ist und, auf eine Kommunikationsplattform und einen Namensraum bezogen, frei
verteilbar und auch wieder lokalisierbar ist. Ein solcher Service ist nur lose mit anderen Services
gekoppelt deren Schnittstellen �ber XML-Artefakte definiert sind. Web Services sind
Services mit der Besonderheit, dass sie sich der Kommunikationsmechanismen des Web bedienen. [PR05]
Web Services sind zurzeit die beste L�sung f�r die Integration von verschiedenen, autonomen EBusiness-
Systemen. Diese werden in Unternehmen eingesetzt, um eine flexible Software-Architektur
zu gew�hrleisten. Die immer wiederkehrenden Funktionen im Unternehmen werden nur einmal implementiert
und anderen Programmen als Services zur Verf�gung gestellt. Diese Vorgehensweise erlaubt
es den Unternehmen   fexibel den Prozessablauf an die ge�nderten Anforderungen anzupassen, um auf
die Marktver�nderungen schnell reagieren zu k�nnen. Dadurch k�nnen Redundanzen vermieden und
Kosten in der Entwicklung gesenkt werden. Die Implementierungsdetails bleiben vor dem Benutzer
verborgen, denn die beiden Teilnehmer m�ssen nur die Schnittstellendefinition kennen. Die Trennung
des Designs und der Entwicklung erlaubt eine fachliche Aufteilung der zu erstellenden Gesch�ftsprozesse.
Unabh�ngig voneinander k�nnen die Fachbereiche auf ihre Aufgabe abgestimmte Prozesse
defnieren und anderen Fachbereichen diese als Service zur Verf�gung stellen. Bei �nderungen muss
anschlie�end nur der Service ge�ndert werden, wenn es keine so gravierenden �nderungen gab, die
sich bis auf die Schnittstelle auswirken konnten. Nach der durchgef�hrten �nderung steht der Service
wieder allen anderen Gesch�ftsprozessen zur Verf�gung. Dadurch wird allen Gesch�ftspartnern
eine hohe Qualit�t der Software bereitgestellt, was sich auch in den niedrigeren Gesamtkosten eines
Projektes widerspiegelt.


Web Services Defnition Language (WSDL)1 ist eine funktionale, XML-basierte Beschreibungssprache
f�r die Schnittstellen eines Web Services. Ein Web Service repr�sentiert sich nach au�en
�ber seine WSDL-Dokumentstruktur, die zun�chst jedem Web Service vorgegeben werden muss. Eine
WSDL-Datei beschreibt nichts anderes als die abstrakten Eigenschaften eines Web Services. Sie
stellt die Grundlage f�r die sp�tere Definition des Gesch�ftsprozesses dar, da diese die Informationen
sowohl zu Messagetypen als auch zu Schnittstellendefnitionen der beteiligten Prozesse enth�lt. Es
beschreibt eine Sicht auf ein Web Service und aller darin beteiligten Rollen. In den Ports werden alle
aufrufbaren Operationen defniert.
Die WSDL-Dokumentstruktur ist eine Menge von Defnitionen, die folgende Elemente beinhaltet.
Das Wurzelelement eines WSDL-Dokumentes bildet das <definitions>-Element. Es enth�lt den
Namen des Services, Namespaces f�r den Service und verwendete Standards. Jedes WSDL-Dokument
besteht immer aus zwei folgenden Teilen, dem mehrfach verwendeten abstrakten und dem konkreten
Teil. Im abstrakten Teil werden die <types>-, <message>- und <portType>-Elemente und in dem
konkreten Teil die <binding>- und <service>-Elemente defniert. Im konkreten Teil wird also
beschrieben, wie und wo auf einen Web Service zugegrifen werden soll, und im abstrakten Teil,
welche Typen und Nachrichten daran beteiligt sind.
Das n�chst folgende Listing zeigt die Grundstruktur eines WSDL-Dokumentes. Die Web Services
werden haupts�chlich �ber folgende Elemente beschrieben.



Im Folgenden werden die Elemente eines WSDL-Dokumentes einzeln beschrieben und n�her erl
�utert. Die dazu entsprechende WSDL-Instanz kann im Anhang A nachgeschlagen werden.
<types> - Dieses Element kapselt die Defnition von Datentypen eines Web Services ein, die f�r
die auszutauschenden Nachrichten ben�tigt werden. Zur Defnition der Datentypen greift WSDL auf
das XML-Schema zur�ck. Die Defnition der Typen wird in Verbindung mit Ein- und Ausgabedaten
in EPK eine besondere Rolle spielen.
<message> - Dieses Element defniert Nachrichten, die zwischen den Partnern verschickt werden.
Die Parameter werden durch das Kind-Element <part> beschrieben. Jede Nachricht kann aus
einem oder mehreren <part>-Elementen bestehen. Das Element <part> ist vergleichbar mit dem
Parameter eines Funktionsaufrufs und wird durch den Namen und seinen Datentypen eindeutig festgelegt.
Das <message>-Element legt somit eindeutig fest, welche Eingaben ein Web Service ben�tigt
und welche Ausgaben dieser erzeugt.
<portTypes> - ist eines der wichtigsten Elemente in einem WSDL-Dokument. Ein BPEL-Prozess
kommuniziert mit anderen Web Services �ber dieses Element. Es beschreibt ein Web Service mit
seinen Operationen, die an einem Port ausgef�hrt werden. Dieses darf eine beliebige Anzahl die
an diesem Port unterst�tzenden Operationen defnieren. Es b�ndelt alle Operationen mit seinen
beteiligten Nachrichten und stellt somit die abstrakte Schnittstelle eines Web Services dar. Es k�nnen
entweder eingehende oder ausgehende Operationen verwendet werden.
<operation> - Dieses Element wird in <portTypes> und in <binding> verwendet und kapselt
in sich die Funktionalit�t eines Web Services. Eine Operation kann folgende Eigenschaften haben. Die
Namen, die in der Spezifikation verwendet werden, lauten <wsdl:one-way-operation>, <wsdl:requestresponse-
operation>, <wsdl:solicit-response-operation> oder auch
<wsdl:notification-operation>. Die beiden Typen <wsdl:one-way-operation> und <wsdl:requestresponse-
operation> werden als inbound operations definiert, weil diese von Web Services angeboten
werden, in dem diese von den eingehenden Nachrichten getriggert werden. Die letzten beiden
Operationenarten werden outbound operations genannt.
<binding> - ist die Defnition des Kommunikationsprotokolls f�r jeden Port und des Nachrichtenformats.
Die abstrakte <portType>-Schnittstelle wird �ber das Binding-Element an konkrete
Protokolle angebunden.
<port> - Ein konkreter Web-Service-Endpunkt wird �ber das <port>-Element eindeutig identi-
fiziert.
<service> - Dieses Element enth�lt alle <port>-Elemente und beschreibt somit die Menge der
Ports zu jedem einzelnen Service. Es fasst alle Endpunkte eines Web Services zusammen.
<ExtensibilityElements> - werden im Zusammenhang mit BPEL-Prozessen verwendet. Es werden
hier BPEL-Elemente wie <partnerLinkTypes> definiert. Die <partnerLinkTypes> werden
�ber ihre <role> und ihre <portTypes> identifiziert. Das hei�t, dass die Charakterisierung �ber
<partnerLinkType> erfolgt, indem jedem Web Service eine bestimmte Rolle zugeordnet wird. EineRolle repr�sentiert eine Ressource und kapselt eine Berechtigung in sich ein. Die bereitgestellten Ports
von beiden Partnern legen eindeutig die Endpunkte f�r die benutzten <partnerLinks> fest. Diese
sind f�r Basis-Aktivit�ten relevant, die Web-Service-Requests ausl�sen k�nnen und charakterisiert
damit zugleich eine Beziehung zwischen zwei oder mehreren Web Services.











%--Business Process Execution Language for Web services
De�nition 2.1 (Web Service).
Ein Web Service ist eine abgeschlossene, selbsterkl�arende und modulare Software-
Komponente, die �uber das Internet ver�o�entlicht, aufgefunden und benutzt werden kann.
Ein Web Service stellt eine beliebig komplexe Funktionalit�at zur Verf�ugung. Ein ver�o�ent-
lichter Web Service wird mit einer anderen Anwendung (m�oglicherweise ebenfalls ein Web
Service) zu einem neuen System komponiert. Der Nachrichtenaustausch zwischen Web
Service basiert zumeist auf dem XML-Format.

Ein Web Service ist somit eine o�ene Komponente mit einer wohl de�nierten Schnittstelle
zur Au�enwelt. Alle zur Benutzung notwendigen Informationen �uber einen Web Service
sind in der �o�entlich zug�anglichen Beschreibung seiner Schnittstellen enthalten.

Web Service Technology Stack
Nun besch�aftigen wir uns detaillierter mit der Architektur von Web Services. Im Gegensatz
zum monolithischen Stil fr�uherer Ans�atze  In diesem Zusammenhang f�allt h�au�g der Begri� des Technology
Stacks. Abbildung 2.2 veranschaulicht dessen Aufbau in Anlehnung an [Sle01].
Den Kern (genannt Core Layers) bilden bereits etablierte Technologien. Durch den R�uck-
gri� auf weitgehend akzeptierte Standards gelingt es mit dieser Schicht, zu fast allen exis-
tierenden Systemen kompatibel zu sein und eine Hard- und Software-unabh�angige Basis f�ur
die aufbauenden Schichten zu scha�en. Zu den Kern-Schichten z�ahlen:
Transport
DerWeb-Service-Ansatz ist an kein bestimmtes Transport-Protokoll gebunden, um die
Flexibilit�at nicht einzuschr�anken. Doch die Verwendung weit verbreiteter Protokolle
wie TCP/IP und HTTP ist �ublich, um Web Services in jedem Umfeld { vor allem
auch �uber Firewalls hinweg { zug�anglich zu machen. [Kre01]
Extensible Markup Language (XML)
XML ist ein allgemein anerkanntes Datenformat zum Austausch von Informationen

zwischen verschiedenen Anwendungen. Aufgrund seiner Flexibilit�at und Erweiterbar-
keit wird XML als Grundstock der meisten Schichten im Web Service Technology
Stack verwendet. [BPMM00]
Simple Object Access Protocol (SOAP)
SOAP unterst�utzt die Kodierung beliebiger Daten (normalerweise in XML) und deren
Transfer (z. B. �uber HTTP). Im Zusammenhang mit Web Services ist SOAP u. a. eine
Programmiersprachen-unabh�angige Version von RPC (Remote Procedure Call). Pa-
rameter des Funktionsaufrufs und Antworten werden ebenfalls in der SOAP-Nachricht
kodiert. [BEK+00]
Die h�oheren Schichten, hier als Emerging Layers bezeichnet, umfassen die neuen Technologien.
Diese Technologien be�nden sich teilweise noch im Entwicklungsstadium oder werden zur
Zeit standardisiert. Als feste Bestandteile des Technology Stacks gelten:
Web Services Description Language (WSDL)
WSDL ist eine XML-basierte Sprache, mit der spezi�ziert wird, welche Operationen
der Web Service zur Verf�ugung stellt und wie die Verbindung mit dem Web Ser-
vice ablaufen soll. Die Konzeption von WSDL sieht eine grobe Zweiteilung vor: Der
abstrakte Teil de�niert sprach- und plattformunabh�angige Typen, Nachrichten, Ope-
rationen und Port-Typen. Der konkrete Teil { genannt binding { bildet die abstrakten
Elemente auf konkrete Datenstrukturen, Protokolle und Adressen ab. [CCMW01]
Universal Description, Discovery, and Integration (UDDI)
UDDI ist die globale Verzeichnisstruktur f�ur Web Services und gleichzeitig ein Pro-
tokoll, um Web Services zu suchen und zu ver�o�entlichen. Eine UDDI-Registrierung
besteht aus den Wei�en, Gelben und Gr�unen Seiten:




Die Web Services Description Language (WSDL) ist eine Schnittstellenbeschreibungssprache
f�r Web Services und wurde im September 2000 in der
Version 1.0 vorgestellt. Seit M�rz 2006 liegt die Version 2.0 als W3C Candidate
Recommendation vor.
Mit WSDL k�nnen Web Services formal beschrieben werden, damit Applikationen
automatisiert auf die angebotenen Dienste zugreifen und diese
einbinden k�nnen. Innerhalb des WSDL-Dokuments werden die angebotenen
Funktionen, die verwendeten Datentypen, die Struktur der Aufrufnachricht
und des Ergebnisses und das Austauschprotokoll spezifiziert. Im Wesentlichen
werden die von au�en zug�nglichen Operationen definiert, sowie deren
Parameter und R�ckgabewerte. (vgl. Wik06)
In WSDL gibt es sechs Hauptelemente, die sich in zwei Gruppen aufteilen
lassen - die abstrakten und die konkreten Definitionen.

Kritikpunkte von WSDL sind die fehlenden semantischen Spezifikationen
eines Web Services. WSDL spezifiziert lediglich die syntaktischen Elemente eines Web Services, d.h. wie ein Client auf den entsprechenden Web
Service zugreifen kann. Dar�ber hinausgehende semantische Spezifikationen
eines Web Services sind allerdings oft w�nschenswert [...]. (Wik06) Mit
WSDL k�nnen keine Informationen �ber die Antwortzeit, Kosten des Services,
Security-Bestimmungen und genauen Effekte einer Operation definiert
werden. Diese werden allerdings zur automatischen Auffindung und Orchestrierung
(siehe Kapitel 5.2, S.23) von Services ben�tigt.


An WSDL wurden hohe Anforderungen gekn�pft. Es soll die Service-Schnittstellen
Programmiersprachen- und Programmierparadigmen-neutral definieren, dar�ber
hinaus abstrakt und erweiterbar sein.

%ebigo

   
   Web Services sind eine Realisierung einer SOA, mit der sich Informationsdienste im WWW automatisieren lassen. Es handelt sich dabei um funktionale Dienste, die �ber Internetprotokolle und die Internet-Infrastruktur �bertragen werden k�nnen. In den meisten F�llen sind Web Services Software-Komponenten, die mittels Applikationsserver im Internet nutzbar gemacht werden. Um Daten miteinander auszutauschen, werden XML-basierte Standards verwendet.
   
    	

Web Services sind funktionale Dienste, die �ber Internetprotokolle und damit der Internet-Infrastruktur �bertragen werden k�nnen. Die mit Web Service Diensten verbundenen Protokolle und Standards wie SOAP, WSDL und UDDI bilden den technologischen Rahmen f�r die Realisierung anspruchsvoller Szenarien in verteilten Anwendungen.


SOA
Mehr Argumente f�r SOA: SOA-Expertenforum der Computerwoche
Jeder der oben genannten Punkte hat mehrere konkrete Vorteile f�r ein Unternehmen:

Business Innovation

- Schnellere Umsetzung von Fusionen und �bernahmen

- Neue Produkte und Angebote durch Neub�ndelung vorhandener Leistungen

- Verbesserte Kundenzufriedenheit durch innovative Angebote

Anwenderproduktivit�t

- Gesteigerte Produktivit�t durch intuitive Benutzeroberfl�chen auf Basis von AJAX f�hren zu mehr Transaktionen pro Zeiteinheit

- Reduzierter Trainingsaufwand

- Weniger operative Fehler durch Vermeidung manueller Daten�bertragung

Prozesseffizienz

- Gesteigerte Prozessautomatisierung

- Verbesserte interne Prozesse, wie zum Beispiel f�r Compliance und Genehmigungsverfahren

- Verbesserte externe Prozesse, beispielsweise im Vertrieb oder im Einkauf

IT-Projekte beschleunigen

- Wiederverwendung vorhandener Funktionen

- Redundante IT-Anwendungen verringern

- Schnellere Reaktion auf neue Anforderungen aus den Gesch�ftsbereichen

- Geringerer Wartungsaufwand

Entkopplung von Architektur-
Bestandteilen

Reduzierte Komplexit�t der
Architektur insgesamt

Schnellere Reaktionsf�higkeit der
IT auf neue Anforderungen

mehr
Flexibilit�t und Effizienz
(Kosten, Geschwindigkeit)
in der Bereitstellung und im
Betrieb von Funktionalit�t

Technologie neutral:
SoA-Services m�ssen mithilfe minimalen standardisierten Technologien nutzbar sein,
welche fast alle IT-Umgebungen mitbringen. Das hat zur Folge, dass die Aufrufmechanismen,
wie Protokolle, Beschreibungen und die Servicesuche/-entdeckung, weit
verbreitete und akzeptierte Standards nutzen, wie z.B. HTTP, SOAP und WSDL.
Locker verbunden:
Services der SoA d�rfen keine Informationen, interne Strukturen oder Konventionen
erfordern und zwar weder auf der Nutzer- noch auf der Serviceseite.
Lokalit�tstransparent:
SoA-Services sollen ihre Definitions- und Lokalit�tsinformation, der Ort an dem der
Service gespeichert ist, in einem Verzeichnisdienst wie zum Beispiel UDDI gespeichert
haben. Das erm�glicht dem Servicenutzer den Service zu lokalisieren und aufzurufen
unabh�ngig von dessen wirklichem Speicherort.
  unabh�ngig -> autonome Software
(= black box)

aufrufbaren -> intern
 auf anderem
Rechner
 extern
 nachrichtenbasiert
 lose
gekoppelt

Gesch�ftsprozess
steht im Vordergrund

Workflow Steuerung

Der entscheidende Vorteil der serviceorientierten Architektur liegt darin, dass SoA
die Implementierung eines Service von dessen Schnittstelle trennt. Das hei�t es wird
das was vom wie getrennt. Der Servicenutzer braucht nicht zu wissen wie der
Service funktioniert und implementiert ist, f�r ihn ist nur interessant, dass er ihm zur
Verf�gung steht und dass er ihm die Serviceleistung liefert die er ben�tigt. 
 man den Service als eine
isolierte Einheit einer Unternehmensfunktion ansehen kann. Diese sind nach einer
gemeinsamen Vorschrift locker miteinander zu einem Standard Kommunikations-
Framework, einer serviceorientierten Architektur, verbunden. Durch die Unabh�ngigkeit,
welche die Services in diesem Framework haben und die entkapselte Programmierlogik,
die sie mitbringen, sind sie nicht an irgendeine Plattform oder Technologie
gebunden.

Eine serviceorientierte Architektur ist im wesentlichen eine Sammlung von Services.
Diese kommunizieren miteinander indem sie sich Daten schicken oder gemeinsam eine Aktivit�t koordinieren. Dazu wird ein Mechanismus ben�tigt, mit dem Services
zusammengef�gt werden k�nnen.

Grundlegend neu am SOA Ansatz ist die Idee, den Gesch�ftsprozess in den Vordergrund zu stellen. Im Gegensatz zur eher monolithischen IT-Anwendung, die unterschiedliche Funktionalit�ten kapseln, erm�glicht SOA die Verf�gbarkeit und Vernetzung unterschiedlicher Dienste von unterschiedlichen Anbietern im gesamten Unternehmen
  Die bekannteste Implementierung einer serviceorientierten Architektur sind Webservices,
da diese mit einem Minimum an standardisierten Protokollen auskommen.
Der Datenaustausch wird, nach [9], meist mit Hilfe von SOAP, einem XML basierten
Datenaustauschprotokoll, �ber das Transportprotokoll HTTP durchgef�hrt. Dabei ist
es bemerkenswert, wie die verschiedensten Protokolle zusammenarbeiten. SOAP beispielsweise
ist unabh�ngig vom Transportprotokoll definiert und kann daher auch �ber
andere Transportprotokolle wie HTTPS, FTP oder SMTP arbeiten. Alternativ zu
SOAP kann der Datenaustausch auch �ber XML-RPC erfolgen.
Im Folgenden werden die Kernentit�ten des SoA-Paradigmas aufgef�hrt und beschrieben
welcher Teil eines Webservices f�r diese Aufgabe zust�ndig ist:
    
  
  "Web Services" beschreiben sich selbst �ber eine so genannte WSDL Datei. Diese Datei enth�lt eine Dokumentation der verf�gbaren Dienste und Anleitungen und erkl�rt, wie diese Dienste angesprochen werden k�nnen. Der Web Service meldet sich bei einem Service Broker an. Den Broker kann man sich als Verzeichnisdienst, als "Gelbe Seiten" der Dienste vorstellen. Ein Anwender kann in diesem Verzeichnisdienst einen geeigneten Dienst suchen und die WSDL Beschreibung anfordern. �ber ein von W3C definiertes Protokoll (SOAP) k�nnen Anwender und Anbieter miteinander kommunizieren.
  die Web Service Description Language. Sie definiert, was f�r Nachrichten mit einem Web-Service auszutauschen und welche seiner Operationen auszuf�hren sind. Au�erdem beschreibt WSDL, wie die fachlichen Nachrichten als technische Soap-Messages aufgebaut sind und wo im Internet ein Server zur Nutzung des Service bereitgestellt ist.

Definition: Service
.. Ein Service ist eine definierte Leistung, einer im Systemzusammenhang definierten
fachlichen Funktion, die von Anwendungsbausteinen erbracht und genutzt wird.
.. Die Serviceschnittstelle (Signatur und funktionale Spezifikation) ist ein Vertrag
zwischen Nutzer und Anbieter.
.. Die Serviceimplementierung ist nicht Teil dieser Abstimmung, bleibt in der
Verantwortung des Service-Anbieters und ist unter Einhaltung des Schnittstellen-
Vertrages austauschbar.   

Steigende Anforderungen
an Integrationsprojekte
.. Steigende Zahl der
Anwendungen im
Unternehmen
.. Steigende Komplexit�t
der Anwendungen
.. Zunehmende
Schnittstellenkomplexit�t
.. Zunehmende
Technologievielfalt
.. Zunehmende
Integrationstiefe


Allerdings kann mithilfe der Schnittstellentechnologie Webservices eine SOA umgesetzt werden.Ein Dienst ist entsprechend eine logische Einheit von oft hoher Komplexit�t  eine Anwendung, Hardware oder eine Kombination aus beiden, die zus�tzlich noch menschliche Leistungen umfassen kann.
Die SOA koppelt derartige Dienste lose: Sie werden nur bei Bedarf gebunden. Ist die Aufgabe erledigt, wird die Bindung wieder aufgehoben.


%Computerwoche










%Java MagazinSOA ist die Abk�rzung f�r Service-

Die Systemlandschaft von Unternehmen ist oftmals sehr heterogen: SAP-Systeme, Nicht-SAP-Systeme, Eigenentwicklungen und Systeme von externen Partnern und Lieferanten m�ssen untereinander �ber Schnittstellen kommunizieren. Verschiedene Systeme arbeiten jedoch mit unterschiedlichen Programmiersprachen. Schnittstellen m�ssen deshalb erst aufwendig erstellt und sp�ter gewartet und gepflegt werden. An dieser Stelle setzt SOA an. SOA nutzt standardisierte WebServices als Schnittstellentechnik.

Ein Prozess, z.B. die Anlegung eines Kundenauftrages, setzt sich aus vielen einzelnen Services zusammen. Ein Service ist z.B. der Aufruf eines Kundenstammes. Der Auftrag wird z.B. in einem SAP-System eingegeben, der Kundenstamm liegt in einem Nicht-SAP CRM-System, zus�tzlich wird der Kundenstamm f�r die Rechnungsverbuchung ben�tigt.  Da die Systemlandschaften der renommierten Hersteller WebServices unterst�tzen, entf�llt mit SOA die zeit- und kostenaufw�ndige Anpassung von Schnittstellen zwischen den Programmen.

SOA ist zudem flexibler. Ohne SOA werden bei einer �nderung der Gesch�ftsprozesse einzelne Module oftmals um zus�tzliche Funktionen erweitert. Die Komplexit�t steigt. SOA dagegen benutzt die einzelnen Services wie einen Baukasten, einzelnen Services k�nnen beliebig zu genau dem Prozess kombiniert werden, wie es das Gesch�ftsmodell des Unternehmens erfordert, egal in welchem System der Service vorhanden ist.




%---Barry & Associates Inc, Service-oriented architecture (SOA) definition
Definition der SoA [14]:
A service-oriented architecture is essentially a collection of services. These services
communicate with each other. The communication can involve either simple data
passing or it could involve two or more services coordinating some activity. Some
means of connecting services to each other is needed.
Serviceorientierte Architektur ist zwar ein recht neuer Begriff in der
Softwareentwicklung, aber der Begriff Service wird in der Softwareentwicklung
schon l�nger benutzt.
Sun definierte die SoA strenger, um Jini zu beschreiben. Jini ist ein System, um Services,
Ger�te und Anwendungen in einer dynamischen Umgebung zu vernetzen [6].
In letzter Zeit gibt es neue Bem�hungen SoA umzustrukturieren, um den Gedanken
von Webservices, in der serviceorientierten Struktur, aufzugreifen. Die folgende Abbildung
zeigt, wie Technologien benutzt werden k�nnen, um die serviceorientierte
Architektur zu implementieren.


Eine SoA ist ein Softwareentwurfsmodell mit dem elementaren Konzept der Abkapselung
von Anwendungslogik innerhalb der Services. Die Services k�nnen �ber ein
vereinbartes Kommunikationsprotokoll miteinander kommunizieren.
Da die Services serviceorientiert sind, gibt es keine Trennung zwischen Client und
Server. In der SoA gibt es nur noch Services, die unabh�ngig von ihrer Umwelt sind
und wenn, dann nur in einer lockeren Beziehung zueinander stehen. Versucht man die
serviceorientierte Architektur mit der Client-Server Architektur zu verstehen, dann
w�rde die Abbildung (Fig. 3) die Interaktion zweier Services am besten wiedergeben.


Eine SOA ist eine Methode zur Konzeption
und Realisierung von Unternehmensanwendungen, die es verschiedenen
Applikationen unabh�ngig vom zugrunde liegenden Betriebssystem und
der gew�hlten Programmiersprache erlaubt, Daten auszutauschen und zu
verarbeiten. Zur Realisierung werden vollst�ndige Anwendungen, oder Teile
daraus, als Dienste angeboten, die ohne Codierungsaufw�nde genutzt werden
k�nnen. (vgl. Jec03)
Komplexere Gesch�ftsprozesse werden durch die Kombination mehrerer
Services realisiert. Die Programmlogik wird �bere diverse, voneinander unabh�ngige
Dienste verteilt und ist nicht in einem einzigen Programm implementiert.


Zusammenfassung

Business Process Execution Language (BPEL) hat sich als ein Quasi-Standard f�r die Komposition mehrerer Web Services zu einem Gesch�ftsprozess etabliert. Trotz der breiten Akzeptanz dieser Sprache existieren noch sehr wenige Werkzeuge, die das Testen von BPEL-Prozessen unterst�tzen. Mit \textit{BPELUnit} wurde ein Framework entwickelt, der das strukturierte Testen einer BPEL-Komposition in Isolation (unabh�ngig von zusammengestellten Diensten) erm�glicht. Die Testabdeckungsanalyse, als Mittel zur Qualit�tskontrolle von Tests, wird durch das Framework nicht durchgef�hrt. Genau an dieser Stelle kn�pft diese Masterarbeit an.  

Es wird in der Arbeit untersucht, inwiefern die bekannten Testabdeckungsmetriken, die in konventionellen Programmiersprachen zur Bewertung der Tests eingesetzt werden, auf die BPEL-Sprache �bertragen werden k�nnen. Au�erdem werden neue f�r BPEL sinnvolle Metriken ausgearbeitet. Anschlie�end soll ein Konzept f�r die Integration dieser Metriken in das BPELUnit-Framework erarbeitet und implementiert werden.



%ComputerWoche

Die letzten beiden BPEL-Elemente <flow> und <link> werden an dieser Stelle ausf�hrlicher
behandelt, da diese das graph-�hnliche Verhalten des Kontrollflusses beschreiben k�nnen.
Flow - Die strukturierte <flow>-Aktivit�t wird f�r die parallele Ausf�hrung elementarer Aktivit�-
ten benutzt. Des Weiteren kann dieses Element auch f�r die Synchronisation verschiedener Zweige in
einem BPEL-Prozess verwendet werden. Wichtig dabei ist, dass alle die zu konkurrierenden und zu
synchronisierenden Aktivit�ten in einem <flow>-Element definiert werden. Die Reihenfolge der Ausf
�hrung der enthaltenen elementaren und der strukturierten Aktivit�ten wird durch Links eindeutig
festgelegt.

 Das Schl�sselwort activity stellt symbolisch die erlaubte Menge einzelner
Aktivit�ten in einem Gesch�ftsprozess dar. Es wird nachfolgend im Zusammenhang mit elementaren
und strukturierten Aktivit�ten verwendet. Es beschreibt, was ein Prozess macht.
F�r die Transformation spielt das Element <activity/> eine besondere Rolle. Hier werden
elementare Aktivit�ten, die in dem EPK-Diagramm der Funktion entsprechen sollen, und strukturierte
Aktivit�ten, die aus dem Kontrollfluss eines EPK-Diagramms abgeleitet werden sollen,
definiert.
Im Folgenden werden die wichtigsten Elemente des BPEL-Schemas n�her untersucht. Die dazu
entsprechenden Listings, die die Definition von BPEL-Instanzen n�her beschreiben, k�nnen in Anhang
B nachgelesen werden. Zur vollst�ndigen Definition der BPEL-Elemente wird auf die Spezifikation
[ACD+03] verwiesen.
PartnerLink - identifiziert einen Partner-Web Service, der mit dem betrachteten Business-Prozess
Nachrichten austauscht. �ber die Rollen myRole und partnerRole wird der Nachrichtenaustausch
spezifiziert. Es ist ein bilateraler Nachrichtenaustausch zwischen den betrachteten Partnern, wenn
beide Attribute myRole und partnerRole definiert sind. Wird dagegen nur eine Rolle definiert, so
handelt es sich dabei um eine asynchrone Nachricht. Die partnerLinks spielen mit den elementaren
Aktivit�ten Receive, Reply und Invoke eine bedeutende Rolle, die Web Services ausl�sen k�nnen.
Partner - In dem Element <partners> werden die einzelnen Partner mit ihren dazu geh�rigen
<partnerLinks> gekapselt. In diesem Fall bedeutet es, dass einzelne oder mehrere Web Services,
die �ber die <partnerLinks> verbunden werden, hier zusammengefasst werden k�nnen.
Variables - Wenn Web Services ausgef�hrt und zwischen diesen Nachrichten oder Daten ausgetauscht
werden, dann m�ssen diese in Variablen gehalten werden. Eine Variable in einem BPELProzess
ist an ihrem Namen und dem verwendeten Datentyp aus dem XML-Schema eindeutig identi
fizierbar.




 
Invoke - Mit <invoke> wird ein Gesch�ftsprozess des Partners aktiviert, indem eine Nachricht
zum Partner-Web Service gesendet wird. Ein Invoke kann eine Operation in nur eine (request) oder
bidirektionale Richtung (request-response) auf dem <portType> eines Partners ausl�sen. Bei einer
request-Operation handelt es sich um einen asynchronen und bei der request/response-Operation
um einen synchronen Aufruf. Die entsprechende Operation wird in der WSDL-Datei des Partners
definiert. Beim asynchronen Ausl�sen des Prozesses wird nur die inputVariable benutzt. Beim
synchronen Ausl�sen wird zus�tzlich die outputVariable verwendet.
Receive - veranlasst den Prozess zu warten, bis eine passende Nachricht vom Partner-Service
ankommt. In einem solchen Fall kann der Prozess nicht fortfahren oder sich beenden, bis die Nachricht
eintritt. <receive> f�hrt beim <partnerLink> an einem <portType> eine <operation> aus
und bindet die erwartete Nachricht an die entsprechende <variable>.
Reply - veranlasst den Prozess, eine Beantwortung auf die Partner-Nachricht zu senden. Der
Prozess sendet an dem entsprechenden <partnerLink> �ber den dazu geh�rigen <portType> eine
bestimmte <operation>. In dem Fall, dass der Partner nicht empfangen kann, wird ein <fault-
Name> zur Fehlerbehandlung definiert.
Empty - Das <empty>-Element f�hrt keine Aktivit�t durch. Diese Aktivit�t kann aufgrund ihrer
Eigenschaft in beliebigen Kontexten verwendet werden. 
Assign - Beim Nachrichtenaustausch ist es erforderlich, dass zwischen den verschiedenen Aktivit
�ten Manipulationen an Variablen vorgenommen werden. Eine solche Manipulation der Variablen
in BPEL erfolgt �ber die <assign>-Aktivit�t. Diese kopiert Daten von der Quelle (from) zum Ziel
(to). Eine Ausf�hrliche Behandlung dieser Aktivit�t wird in Abschnitt 4.5 und 5.4.2 gegeben.
Link - Die oben vorgestellten Kontrollstrukturen erlauben, viele typische Arbeitsabl�ufe zu modellieren.
Es ist jedoch nicht m�glich, noch weitere Sequenzen zu definieren, wenn die Aktivit�-
ten innerhalb dieser Sequenzen einer bestimmten Reihenfolge unterliegen. Die verwendete XMLDatenstruktur,
die nur hierarchische Modellierung erlaubt, bringt das auf XML-basierte Konzept schnell an seine Grenzen. Um dieses Problem zu umgehen, wurden hierf�r die Links in dem <flow>-
Block eingef�hrt. Die Links erlauben bei der Modellierung Verkn�pfungen zwischen einzelnen Aktivit
�ten herzustellen. Somit stellt ein <flow>-Element mit den dazugeh�rigen <links> graphentheoretisch
nichts anderes als eine Kante zwischen zwei Aktivit�ten dar.
Jeder Link wird �ber seinen Namen in der <links>-Definition eindeutig identifiziert. Um die
Links benutzen zu k�nnen, m�ssen hierf�r in jeder Aktivit�t, die mit den Links verbunden werden
soll, zwei weitere optionale Elemente <source> und <target> definiert werden. Der Kontrollfluss
verl�uft von der Aktivit�t mit dem <source>-Element zu der Aktivit�t mit dem <target>-Element.
Anhand der Definition der linkNames wird die auszuf�hrende Reihenfolge identifiziert. Somit k�nnen
die Aktivit�ten beliebig viele <source>- oder <target>-Eintr�ge2 enthalten, wobei man beachten
sollte, dass die Definitionen eindeutig sind und nicht doppelt vorkommen d�rfen.
Beim Synchronisieren der Links an der Target-Aktivit�t muss das Attribut joinCondition, welches
als optionale Attribut in allen Aktivit�ten belegt werden kann, gesetzt werden. Wenn das Attribut
joinCondition nicht gesetzt ist, so ist der Default-Wert vom Linkstatus aller eingehender Links der
betrachteten Aktivit�t eine logische Disjunktion (logical or ). Der Status des Links kann mit get-
LinkStatus('linkName') abgefragt werden. F�r die richtige Verwendung der Links sorgt die in BPEL
eingef�hrte Dead-Path-Elemination (DPE). Auf die ausf�hrlichere Definition der Link-Semantik und
der DPE wird in diesem Zusammenhang auf die BPEL-Spezifikation [ACD+03] verwiesen.
Die Links erm�glichen, die hierarchische Struktur der XML-Struktur zu sprengen und erlauben
somit zwei Aktivit�ten auszuf�hren, die sich in unterschiedlichen Verschachtelungsebenen der XMLDatenstruktur
befinden. Dieser Zusammenhang ist in der n�chsten Abbildung zu erkennen. Ein solcher
Fall w�re mit anderen strukturierten Aktivit�ten ohne die Links in BPEL nicht modellierbar.

Aktivit�aten
Im Zentrum der Modellierung von Gesch�aftsprozessen stehen die Aktivit�aten und ihre kau-
salen Zusammenh�ange. In BPEL4WS gibt es zwei Sorten von Aktivit�aten: Basic Activities
und Structured Activities.
Die einfachen Aktivit�aten repr�asentieren atomare Einheiten und lassen sich erneut in
zwei Gruppen teilen: F�ur die Kommunikation mit einem anderen Web Service sind die
Aktivit�aten invoke, receive und reply zust�andig. Mit invoke kann ein anderer Web Service
aufgerufen werden, d. h. es wird eine Nachricht an diesen gesendet und gleich oder sp�ater
eine Antwort erwartet. Mit receive wird ein Aufruf (eine Nachricht) von einem anderen Web
Service entgegen genommen und mit reply dieser beantwortet.
Die zweite Gruppe der einfachen Aktivit�aten repr�asentieren interne Schritte: assign ist
eine Wertzuweisung, wait ist ein Timer und empty ist eine leere Aktivit�at. Teilweise steuern
sie auch den weiteren Prozessverlauf: terminate bricht den Prozess ab, throw wirft einen
Fehler und compensate veranlasst die R�ucksetzung eines Teils des Prozesses.
Neben den einfachen Aktivit�aten gibt es in BPEL4WS auch f�unf Klassen strukturierter
Aktivit�aten. Mit diesen Aktivit�aten wird der Kontroll�uss abgebildet: Die einfachste ist se-
quence, diese Aktivit�at de�niert die sequentielle Ordnung einer Menge anderer Aktivit�aten.
Die alternative Auswahl zwischen Aktivit�aten ist mit Hilfe von pick und switch m�oglich, bei
pick entscheidet eine Nachricht von au�en, bei switch wird durch die Auswertung von Daten
eine Entscheidung getro�en. Mit der Aktivit�at while ist es m�oglich, zyklisches Verhalten
zu de�nieren. Letztlich dient die Aktivit�at �ow dazu, eine unabh�angige Menge von Akti-
vit�aten zu spezi�zieren. Innerhalb von �ow k�onnen die Aktivit�aten durch zus�atzliche links
untereinander synchronisiert werden. Jede der strukturierten Aktivit�aten enth�alt ihrerseits
mindestens eine Aktivit�at. Auf diese Weise k�onnen durch Schachtelung beliebig komplexe
Kontroll�uss-Beziehungen gebildet werden.
Ein Sonderrolle spielt die Aktivit�at scope.

Das Kernkonzept von BPEL ist die Aktivit�t. Ein BPEL-Prozess ist genau eine Aktivit�t, in
der weitere Aktivit�ten enthalten sind. BPEL unterscheidet dabei zwei Arten von Aktivit�ten:
Basisaktivit�ten und Strukturierte Aktivit�ten.
Basisaktivit�ten erf�llen die atomaren Aufgaben des Prozesses:
 Kommunikation:
 Empfang von Nachrichten, mit <receive>
 Beantwortung empfangener Nachrichten, mit <reply>
 Aufruf eines Web Service, mit <invoke>
 Datenmanipulation:
 Manipulation von Werten in Variablen, mit <assign>
Sonstige Basisaktivit�ten:
Nichtstun, mit <empty>
 Warten, mit <wait>
 Signalisieren von Fehlern und Ausnahmen, mit <throw>
 Beenden der Prozessinstanz, mit <terminate>
 Kompensation von abgearbeiteten Aktivit�ten, mit <compensate>

Basisaktivit�ten werden mit Hilfe von Strukturierten Aktivit�ten in eine Reihenfolge der Ausf�hrung
gebracht:
 Sequentielle Anordnung von Aktivit�ten, mit <sequence>
 Parallele Anordnung von Aktivit�ten, mit <flow>
 Auswahl von Alternativen (basierend auf Daten), mit <switch>
 Auswahl von Alternativen (basierend auf Nachrichten oder zeitgesteuerten Ereignissen),
mit <pick>
 Definition von Schleifen, mit <while>

ktivit�aten k�onnen ineinander geschachtelt sein und bilden somit eine hierarchische Struk-
tur { einen Aktivit�aten-Baum. DieWurzel dieses Baumes ist der Prozess: Ein Prozess besitzt
genau eine Aktivit�at. Neben dieser Aktivit�at kann ein Prozess event handler, fault handler
und compensation handler besitzen, d. h. ein Prozess ist auch gleichzeitig ein scope.

WS-BPEL 2.0 bietet mit der neuen <forEach>-Aktivit�t ein weiteres Konstrukt zur wiederholten
Ausf�hrung einer Scope-Aktivit�t.

Beim Start der <forEach>-Aktivit�t werden die Ausdr�cke in den <startCounterValue>- und
<finalCounterValue>-Elementen einmalig ausgewertet und bleiben dann �ber die gesamte
Laufzeit der Aktivit�t konstant. Die dort spezifizierten Ausdr�cke m�ssen ein xs:unsignedint
als Ergebnistyp haben, sonst wird der Fehler bpws:forEachCounterError geworfen. Sollte der
Wert von <startCounterValue> gr��er sein, als der von <finalCounterValue>, wird keine
Iteration durchgef�hrt.
�ber das Attribut parallel kann gesteuert werden, ob der Schleifenk�rper sequentiell oder
parallel abgearbeitet wird.

Im Fall von parallel=no wird die Schleife sequentiell abgearbeitet. Die eingebettete Scope-
Aktivit�t wird N+1 Mal (wobei N die Differenz von <finalCounterValue> und <startCounterValue>
ist) nach einander ausgef�hrt. F�r jeden Durchlauf wird eine Variable vom Typ xs:unsignedint
kreiert, die dann mit dem aktuellen Z�hlerwert initialisiert wird. Der Bezeichner dieser Variable
wird �ber das Attribut counterName der <forEach>-Aktivit�t festgelegt. Im ersten Durchlauf
wird sie auf denWert von <startCounterValue> initialisiert und bei jedem weiteren Durchlauf
inkrementiert. Im letzten Durchlauf besitzt die Z�hlervariable denWert von <finalCounterValue>.
Die Z�hlervariable ist lokal f�r den eingebetteten Scope, kann also von diesem gelesen und geschrieben
werden. Nach einer Abarbeitung des Schleifenk�rpers werden jedoch alle �nderungen
an dieser Variable verworfen.
Im Fall von parallel=yes wird die Schleife parallel abgearbeitet. Es wird eine implizite
Flow-Aktivit�t generiert, die N+1 Instanzen der eingebetteten Scope-Aktivit�t enth�lt. F�r jede
dieser Instanzen wird eine Z�hlervariable (wie im sequentiellen Fall) kreiert, die eindeutig mit
einem Wert aus dem Intervall <startCounterValue> bis <finalCounterValue> initialisiert
wird.
Optional kann eine <forEach>-Aktivit�t ein <completionCondition>-Element enthalten.Wird
die in diesem Element spezifizierte Bedingung w�hrend der Abarbeitung erf�llt, so kann die
<forEach>-Aktivit�t vorzeitig beendet werden. Das <completionCondition>-Element enth�lt
das Element <branches>, in dem ein Integer-Ausdruck angegeben wird. Dieser wird beim Start
der <forEach>-Aktivit�t einmalig ausgewertet. Ist der Wert des Ausdrucks gr��er als die Anzahl
der Schleifendurchl�ufe (bzw. parallelen Instanzen), wird der Fehler bpws:invalidBranch-
Condition geworfen. Nach jeder Abarbeitung eines eingebetteten Scope wird die Anzahl der
abgearbeiteten Scopes mit dem Wert dieses Ausdrucks verglichen. Ist die Anzahl der abgearbeiteten
Scopes gr��er oder gleich dem Wert des Ausdrucks, gilt die Bedingung als erf�llt. Es
lassen sich also Bedingungen nach dem Muster mindestens M aus N 6 definieren.

Scope
Die Spezifikation von WS-BPEL 2.0 beschreibt das Verhalten von Scopes f�r den Fall der Initialisierung
und Beendigung.
Die Abarbeitung eines Prozesses bzw. Scopes beginnt mit der Initialisierung. Dabei werden
die Fault und Termination Handler, Partner Links, Correlation Sets und die Variablen instanziiert
und initialisiert. Die Partner Links m�ssen erstellt werden, bevor die im gleichen Scope
definierten Variablen initialisiert werden k�nnen. Die Scope-Initialisierung verl�uft nach dem
Motto: alles oder nichts. Entweder alles wird erfolgreich initialisiert oder es wird der Fehler
bpws:scopeInitializationFailure geworfen, der vom umgebenden Scope behandelt werden muss.
Im Falle eines solchen Fehlers auf Prozessebene gilt der gesamte Prozess als faulted. Nachdem
die Initialisierung abgeschlossen ist, werden die erste innere Aktivit�t und die Event Handler des
Scope parallel instanziiert. Eine Ausnahme dazu stellen Scopes dar, die eine initial start activity
enthalten (d.h. die Instanziierung einer weiteren Prozessinstanz bewirken). In diesem Fall muss
zuerst die initial start activity beendet worden sein, bevor die Event Handler instanziiert werden
k�nnen.
Wird ein Scope vollst�ndig abgearbeitet, m�ssen alle Interaktionen mitWeb Services, die abh�ngig
von innerhalb des Scope definierten Partner Links und messageExchange-Definitionen sind,
beendet sein. Bleiben Empfangsaktivit�ten offen, die sich auf Partner Links oder messageExchange-
Definitionen des Scope beziehen, kann der Fehler bpws:missingReply geworfen werden.8
Ein Scope (bzw. Prozess) kann in WS-BPEL 2.0 das exitOnStandardFault-Attribut besitzen.
Falls der Wert dieses Attributs auf yes gesetzt ist, muss der Prozess sofort beendet werden
(�quivalent zum Erreichen der Aktivit�t <exit>), wenn ein anderer WS-BPEL Standardfehler
als bpws:joinFailure auftritt. IstWert des Attributs hingegen no, so kann der Prozess den Standardfehler
mittels eines Fault Handler behandeln. Der Standardwert des exitOnStandardFault-
Attributs ist no. Wird dieses Attribut von einer Scope-Aktivit�t nicht spezifiziert, so wird der
Wert vom umgebenden Scope bzw. Prozess geerbt.

Isolierte Scopes
Serialisierbare Scopes werden in WS-BPEL 2.0 als isolierte Scopes bezeichnet. Das Attribut
variableAccessSerializable hei�t nun entsprechend isolated. Neu ist auch, dass ein als
isoliert markierter Scope weitere Scopes einbetten kann, die nicht als isoliert markiert sind.9 Der
Zugriff auf gemeinsam genutzte Variablen durch die so eingebetteten Scopes wird dann �ber den
isolierten Scope kontrolliert.
Beachte, dass die Isolation eines Variablenzugriffs nicht zu einem internen Deadlock in einem
BPEL-Prozess f�hren kann. Der Grund daf�r liegt darin, dass ein isolierter Scope nicht eher
gestartet wird, als bis er exklusiven Zugriff auf alle vom ihm ben�tigten nicht-lokalen Variablen
hat.

Event Handler
Event Handler f�hren in WS-BPEL 2.0 grunds�tzlich nur noch Scope-Aktivit�ten aus, um eine
saubere Scope-Snapshot- und Kompensationssemantik zu gew�hrleisten. Dies ist nat�rlich
keine Einschr�nkung, da die Scope-Aktivit�t beliebige weitere Aktivit�ten enthalten kann. Tritt
w�hrend der Abarbeitung eines Event Handler ein Fehler auf, wird dieser also zuerst vom Fault
Handler der eingebetteten Scope-Aktivit�t behandelt bzw. von ihm an den umgebenden Scope
weiter gereicht.
Sowohl von <onEvent>- als auch von <onAlarm>-Event-Handler k�nnen zu einem Zeitpunkt
mehrere aktive Instanzen existieren. Jede Instanz erh�lt daher eine private Kopie der im eingebetteten
Scope deklarierten Daten und Ressourcen (inkl. Links und Partner Links), um Zugriffskonflikte
zu verhindern.

Fault Handler
Die WS-BPEL 2.0 Spezifikation beschreibt die optional zu einem Fehlernamen assoziierten Fehlerdaten
weitaus konkreter, als dies bisher der Fall war.
Fehlerdaten sind WSDL-Nachrichtentypen oder XML-Schema-Elemente. Jeder <catch>-Zweig,
der einen faultName spezifiziert kann nur Fehler eines einziges Typs behandeln. Wenn die
zu behandelnden Fehlerdaten von einem WSDL-Nachrichtentyp sind, dann muss dieser mittels
des (neuen optionalen) faultMessageTyp-Attributs spezifiziert sein.Wenn die Fehlerdaten eine
XML-Element-Definition sind, dann muss diese mittels des (neuen optionalen) faultElement-
Attributs spezifiziert sein.
Da das Attribut faultName optional ist, kann es vorkommen, dass eine spezifizierte faultVariable
keinem konkreten Typ zugeordnet werden kann. Um dies zu verhindern, muss bei der Spezifikation
des faultVariable-Attributs stets auch entweder ein faultMessageType- oder ein
faultElement-Attribut mit angegeben werden. Zudem d�rfen sowohl das faultMessageTypeals
auch das faultElement-Attribut nie ohne ein begleitendes faultVariable-Attribut angegeben
werden.
F�r die Behandlung eines Fehlers k�nnen mehrere Fault Handler zur Auswahl stehen. Die Regeln,
nach denen ein Fault Handler zur Fehlerbehandlung ausgew�hlt wird, wurden in Anbetracht
der erweiterten M�glichkeiten von WS-BPEL 2.0 entsprechend angepasst.1

Termination Handler
Scopes besitzen die M�glichkeit, auf den Ablauf einer erzwungenen Terminierung Einfluss zu
nehmen. Was in BPEL4WS 1.1 als Forced-Termination-Zweig des Fault Handler definiert wurde,
wird in WS-BPEL 2.0 innerhalb des Scope als Termination Handler definiert. Dieser kann
die selben Aktivit�ten verwenden, die auch in einem Fault Handler verwendet werden k�nnen.
Es wird also nicht mehr der Fehler bpws:forcedTermination geworfen, auf den dann im Fault
Handler reagiert werden kann, sondern im Falle einer erzwungenen Terminierung wird nach der
Beendigung aller laufenden Aktivit�ten des Scope der Termination Handler ausgef�hrt. Ist kein
Termination Handler definiert, wird ein Standard-Termination-Handler aktiviert. Dieser kompensiert
alle erfolgreich beendeten eingebetteten Scopes in der standardm��igen Kompensationsreihenfolge
(siehe 3.10.1). Er verh�lt sich also wie der Standard-Fault-Handler.

Compensation Handler
Der aktuelle Zustand des Prozesses setzt sich aus den aktuellen Zust�nden aller gestarteten Scopes
zusammen. Dies beinhaltet Scopes die erfolgreich abgearbeitet wurden, deren Compensation
Handler aber noch nicht aufgerufen wurde. Der aktuelle Zustand eines erfolgreich abgearbeiteten
Scope entspricht dessen Zustand zum Zeitpunkt der Beendigung der Abarbeitung. Ein Scope
kann, m�glicherweise als Teil eines Schleifendurchlaufs, mehrfach ausgef�hrt worden sein. In
diesem Fall enth�lt der aktuelle Zustand des Prozesses den Zustand einer jeder erfolgreichen
(nicht kompensierten) Ausf�hrung eines solchen Scope. Dieser gesicherte Zustand eines erfolgreich
abgearbeiteten unkompensierten Scope wird als Scope-Snapshot bezeichnet.
Compensation Handler nutzen in WS-BPEL 2.0 immer den aktuellen Zustand des Scope, d.h.
den Zustand des Scope, in dem sie definiert wurden und allen darin eingebetteten Scopes. Dies
umfasst die dort deklarierten Variablen, Partner Links und Correlation Sets. Sie sind in der Lage,
dieWerte all dieser Variablen zu lesen und zu setzen. Andere Teile des Prozesses k�nnen die
Ver�nderung an gemeinsam genutzten Variablen sehen, genauso wie die Compensation Handler
Ver�nderungen sehen k�nnen, die durch andere Teile des Prozesses an gemeinsam genutzten
Variablen vorgenommen werden. Dies gilt auch f�r den Fall einer parallelen Abarbeitung. Um
Konflikte im Falle von erwarteter gleichzeitiger Abarbeitung zu verhindern, m�ssen Compensation
Handler isolierte Scopes (siehe 3.6.1) verwenden, wenn sie in den Zustand eines eingebetteten
Scope eingreifen.
Wie auch in BPEL4WS 1.1 steht ein Compensation Handler nur f�r Scopes zur Verf�gung, die
ordnungsgem�� abgearbeitet wurden. Wird ein Compensation Handler wiederholt aufgerufen,
wird dies in WS-BPEL 2.0 ignoriert.11 Dabei spielt es keine Rolle, ob es sich um einen aktivierten
oder deaktivierten Compensation Handler handelt.
Die aus BPEL4WS 1.1 bekannte Aktivit�t <compensate> wurde in zwei Aktivit�ten zur Kompensation
aufgeteilt: <compensate>, zur Kompensation aller eingebetteten Scopes in der standardm��igen
Kompensationsreihenfolge (siehe 3.10.1) und <compensateScope> zur Kompensation
eines speziellen Scope. Ist der zu kompensierende Scope in eine Schleife oder einen Event
Handler eingebettet, m�ssen alle Iterationen der Schleife bzw. alle verarbeiteten Ereignisse des
Event Handler kompensiert werden. Die f�r eine solche Aktivit�t im Laufe der Abarbeitung
aktivierten Compensation Handler werden dabei als eine Einheit betrachtet, die als Compensation
Handler Instance Group bezeichnet wird. Im Falle eines Aufrufs von <compensate>,
umfasst diese Compensation Handler Instance Group alle Compensation-Handler-Instanzen der
erfolgreich abgearbeiteten Scopes, die in der Aktivit�t eingebettet sind. Im Falle eines Aufrufs
von <compensateScope>, jedoch nur die Compensation-Handler-Instanzen des spezifizierten
Ziel-Scope. Sollte w�hrend der Abarbeitung einer dieser Compensation-Handler-Instanzen ein
unbehandelter Fehler auftreten, wird die gesamte Compensation Handler Instance Group beendet
und alle enthaltenen Compensation Handler werden deaktiviert.
Standardm��ige Kompensationsreihenfolge
Die Spezifikation von WS-BPEL 2.0 beschreibt zwei Regeln, die f�r eine standardm��ige Kompensationsreihenfolge
gelten m�ssen. Um diese Regeln zu konkretisieren, werden formal einige
Begriffe definiert.12 Wir wollen uns jedoch hier nur auf eine informelle Wiedergabe dieser Regeln
beschr�nken:
1. Die Kompensation muss sich nach der Reihenfolge der Abarbeitung der zu kompensierenden
Scopes richten, soweit diese durch die Prozessdefinition vorgegeben ist.
Prozesse, in denen es zu Zyklen durch Links kommen kann, sind nicht gestattet, d.h. wenn
von einem Scope A ein Link in einen anderen Scope B f�hrt, darf von Scope B kein Link
in den Scope A zur�ckf�hren.
F�r den Fall von nebenl�ufig abgearbeiteten Scopes, die keine Kontrollflussabh�ngigkeiten durch
Links aufweisen, wird durch diese Regeln keine konkrete Kompensationsreihenfolge festgelegt.
Sie k�nnen also sowohl nebenl�ufig, als auch in strikter umgekehrter Reihenfolge ihrer Abarbeitung
nach einander kompensiert werden.
Kompensation und isolierte Scopes
Compensation Handler eines isolierten Scope werden, anders als entsprechende Fault Handler,
nicht innerhalb dessen isolierter Umgebung ausgef�hrt. Auch k�nnen diese Compensation
Handler nicht selbst isolierte Scopes verwenden, da isolierte Scopes nicht in einander geschachtelt
werden d�rfen. Dies wirft jedoch Fragen in Bezug auf die Isolationssemantik der Compensation
Handler von in isolierten Scopes eingebetteten Scopes auf. Wie sieht deren Umgebung
im Falle eines Aufrufs durch einen Fault Handler (ausgef�hrt innerhalb der isolierten Umgebung
des umgebenden Scope) bzw. im Falle eines Aufrufs durch einen Compensation Handler
(ausgef�hrt au�erhalb der isolierten Umgebung des umgebenden Scope) aus?
Um konsistentes Verhalten sicher zu stellen, verlangt die Spezifikation von WS-BPEL 2.0 das
Compensation Handler innerhalb eines isolierten Scope daher selbst implizit isoliertes Verhalten
zeigen, obwohl dies eine separate isolierte Umgebung erfordert.

Im Zentrum der Modellierung von Gesch�aftsprozessen stehen die Aktivit�aten und ihre kau-
salen Zusammenh�ange. In BPEL4WS gibt es zwei Sorten von Aktivit�aten: Basic Activities
und Structured Activities.
Die einfachen Aktivit�aten repr�asentieren atomare Einheiten und lassen sich erneut in
zwei Gruppen teilen: F�ur die Kommunikation mit einem anderen Web Service sind die
Aktivit�aten invoke, receive und reply zust�andig. Mit invoke kann ein anderer Web Service
aufgerufen werden, d. h. es wird eine Nachricht an diesen gesendet und gleich oder sp�ater
eine Antwort erwartet. Mit receive wird ein Aufruf (eine Nachricht) von einem anderen Web
Service entgegen genommen und mit reply dieser beantwortet.
Die zweite Gruppe der einfachen Aktivit�aten repr�asentieren interne Schritte: assign ist
eine Wertzuweisung, wait ist ein Timer und empty ist eine leere Aktivit�at. Teilweise steuern
sie auch den weiteren Prozessverlauf: terminate bricht den Prozess ab, throw wirft einen
Fehler und compensate veranlasst die R�ucksetzung eines Teils des Prozesses.
Neben den einfachen Aktivit�aten gibt es in BPEL4WS auch f�unf Klassen strukturierter
Aktivit�aten. Mit diesen Aktivit�aten wird der Kontroll�uss abgebildet: Die einfachste ist se-
quence, diese Aktivit�at de�niert die sequentielle Ordnung einer Menge anderer Aktivit�aten.
Die alternative Auswahl zwischen Aktivit�aten ist mit Hilfe von pick und switch m�oglich, bei
pick entscheidet eine Nachricht von au�en, bei switch wird durch die Auswertung von Daten
eine Entscheidung getro�en. Mit der Aktivit�at while ist es m�oglich, zyklisches Verhalten
zu de�nieren. Letztlich dient die Aktivit�at �ow dazu, eine unabh�angige Menge von Akti-
vit�aten zu spezi�zieren. Innerhalb von �ow k�onnen die Aktivit�aten durch zus�atzliche links
untereinander synchronisiert werden. Jede der strukturierten Aktivit�aten enth�alt ihrerseits
mindestens eine Aktivit�at. Auf diese Weise k�onnen durch Schachtelung beliebig komplexe
Kontroll�uss-Beziehungen gebildet werden.
Ein Sonderrolle spielt die Aktivit�at scope.
�U
b
e
r
w
a
c
h
t
e
A
k
t
i
v
i
t
�a
t
e
n
In BPEL4WS ist es m�oglich, eine einzelne Aktivit�at unter besondere Beobachtung zu stel-
len, d. h. auftretende Fehler abzufangen, auf externe Ereignisse zu reagieren und ggf. die
Aktivit�at nach erfolgreicher Ausf�uhrung zu kompensieren, wenn es die �au�eren Umst�an-
de verlangen. Zu diesem Zweck gibt es das Konzept des scopes als Aggregation aus event
handler, fault handler, compensation handler und einer �uberwachten Aktivit�at. Auf der
einen Seite kann die �uberwachte Aktivit�at wiederum strukturiert sein, so dass der scope
eine Menge von Aktivit�aten �uberwachen kann. Auf anderen Seite kann der scope selbst als
eine Aktivit�at aufgefasst werden und damit in einer strukturierten Aktivit�at vorkommen.
Prozesse
Aktivit�aten k�onnen ineinander geschachtelt sein und bilden somit eine hierarchische Struk-
tur { einen Aktivit�aten-Baum. DieWurzel dieses Baumes ist der Prozess: Ein Prozess besitzt
genau eine Aktivit�at. Neben dieser Aktivit�at kann ein Prozess event handler, fault handler
und compensation handler besitzen, d. h. ein Prozess ist auch gleichzeitig ein scope.
In BPEL4WS geh�ort zu einem Web Service genau ein Prozess. Es ist m�oglich, diesen Pro-
zess pr�azise und vollst�andig zu modellieren. Auf diese Weise gelangt man zu einem direkt
ausf�uhrbaren Prozessmodell (= Executable Process). Dar�uber hinaus gestattet BPEL4WS
eine abstrakte Spezi�kation des Verhaltens (= Business Protocol). Die abstrakte Spezi-
�kation ist ein wesentlicher Teil der ver�o�entlichten Beschreibung des Web Service. Ein
potentieller Anwender entscheidet aufgrund dieser Spezi�kation, ob der vorliegende Web
Service mit seiner Komponente kompatibel ist. Im Rahmen dieser Arbeit werden wir die
Kompatibilit�at zweier Web Services de�nieren (siehe Abschnitt 3.1.4).
Die De�nition der Kommunikation innerhalb des Prozesses st�utzt sich auf die De�nition
der Schnittstelle des Web Service mit WSDL ab.
Kommunikation
Ein Web Service ist ein o�enes System, dass mit anderen Komponenten (o. B. d. A. eben-
falls Web Services) �uber Nachrichtenaustausch kommuniziert. Jede Nachricht ist ein XML-
Dokument, wobei die Typ der Dokumente bereits im WSDL-Modell des Web Service de�-
niert werden. Als konsequente Weiterf�uhrung verwendet BPEL4WS f�ur interne Datenstruk-
turen (variables) ebenfalls XML-Dokumente.
Die Schnittstelle f�ur die Kommunikation wird in WSDL de�niert: Eine Nachrichten ist
ein XML-Dokument, das Senden und empfangen einer Nachricht (mit oder ohne Feedback)hei�t Operation. Eine Menge von Operationen wird zu einem Porttype zusammengefasst.
Jeder Web Service bietet eine Menge von Porttypen an.
Die Kommunikation �uber einen Porttypen kann man von zwei Seiten aus betrachten:
Auf der einen Seite gibt es den Web Service, der diesen Porttypen anbietet und auf der
anderen Seite gibt es einen Web Service, der diesen Porttypen von au�en benutzt. Deshalb
wird in BPEL4WS die Verbindung zwischen den Operationen eines Porttypen und den
Aktivit�aten im Prozess durch PartnerLinks abgebildet: Die Aktivit�at receive und reply
implementieren eine Operation an der Schnittstelle dieses Web Service und stellen damit
einem anderem Web Service Funktionalit�at zur Verf�ugung, die Aktivit�at invoke spezi�ziert
eine Operation an der Schnittstelle eines anderen Web Service und benutzt damit dessen
Funktionalit�at. Da ein anderer Web Service (d. h. ein Partner) in einem BPEL4WS-Modell
sowohl Funktionalit�at anbieten als auch benutzen kann, wird ein Partner durch eine Menge
von PartnerLinks spezi�ziert.
PEL4WS (kurz BPEL) unterscheidet zwischen ausf�hrbaren und abstrakten Prozessen, wobei
die Unterscheidung f�r diese Arbeit irrelevant ist. Ein BPEL4WS Prozess selbst kann als
ein Ablaufdiagramm angesehen werden, wobei jeder Knoten eine Aktivit�t darstellt. Hierbei
wird zwischen primitiven und strukturierten Aktivit�ten unterschieden. Die primitiven Aktivit�ten
erm�glichen es: Operationen auf anderen Web Services aufzurufen (invoke), auf
Nachrichten von externen Partnern zu warten (receive), Antworten auf eine Anfrage zu verschicken
(reply), eine bestimmte Zeit zu warten (wait), den Variablen Werte zuzuweisen
(assign), Fehler zu generieren (throw), den gesamten Service explizit zu beenden (terminate)
oder auch nichts zu tun (empty). Die strukturierten Aktivit�ten erm�glichen es die
primitiven zu einer komplexeren Einheit zu kombinieren. Es ist m�glich, folgendes mit strukturierten
Aktivit�ten zu realisieren: Eine geordnete Sequenz der Aktivit�ten zu definieren
(sequence), Verzweigungen innerhalb eines Prozesses zu bilden (switch), Schleifen zu
definieren (while), Entscheidungen nach Ablauf bestimmter Zeit oder anhand externer
Trigger zu treffen (pick) und Aktivit�ten parallel auszuf�hren (flow). Au�erdem k�nnen
mit Hilfe des control links Konstrukts Ausf�hrabh�ngigkeiten zwischen parallelen Aktivit�ten
definiert werden.



 Das Besondere daran: WS-BPEL bietet eine homogene Syntax f�r die Ablaufbeschreibung und den Datenzugriff auf XML-Dokumente. Und es enth�lt Elemente, die speziell auf die Ablaufproblematik langlaufender Gesch�ftsprozesse mit mehreren Partnern zugeschnitten sind. So k�nnen die Aktivit�ten eines Gesch�ftsprozesses in Scopes, das hei�t in kontextorientierten, transaktionalen Einheiten zusammengefasst werden. F�r den Fehlerfall, bei dem bereits abgeschlossene Scopes zur�ckgesetzt werden m�ssen - man spricht von Kompensation - , enth�lt WS-BPEL m�chtige syntaktische Konstrukte. Hier ist WS-BPEL anderen Programmiersprachen klar �berlegen - in der Theorie.


Nun l�sst sich das Konzept der XPath-Variablen nutzen

In WS-BPEL 2.0 wurde das Konzept der XPath-Variablen eingef�hrt. Eine XPath-Variable entsteht syntaktisch einfach dadurch, dass einer BPEL-Variablen ein \$-Zeichen vorangestellt wird. Das Ergebnis einer Suche kann damit direkt als Ausdruck weiterbenutzt werden. Variablenzuweisungen, die in WS-BPEL durch <assign>-Aktivit�ten erfolgen, lassen sich nun �hnlich elegant formulieren wie Zuweisungen in objektorientierten Sprachen.

F�r Prozesse, in denen Suchen und Kopieren zum Aufbau von Nachrichten nicht ausreichen, bietet die neue BPEL-Version standardm��ig einen integrierten Ausdruck f�r die XSLT-Transformation einer XPath-Variablen.




Grundlegende Konzepte von SOA
Hinter Serviceorientierte Architekturen (SOA) steckt die Absicht, Gesch�ftsprozesse zu automatisieren und auf Maschine-zu Maschine-Kommunikation zu verlagern. Komplexe Anwendungen werden hierf�r in standardisierte Services aufgebrochen und verteilt zur Verf�gung gestellt.

BPEL ist blockstrukturiert, d.h. bei der Definition von lokalen Umgebungen (Scopes) k�nnen Variable (Programmierung) eingef�hrt werden. Mit den Scopes k�nnen au�erdem Fehlerbehandlung (Fault Handler), Kompensationsbehandlung (Compensation Handler) und Ereignisbehandlung (Event Handler) assoziiert werden.
 
Man unterscheidet zwischen einer synchronen und asynchronen invoke-Aktivit�t.
Bei der synchronen Variante wird der Inhalt der Eingabevariable (inputVariable) an den entfernten Web Service geschickt und das Ergebnis in der Ausgabevariable (outputVariable) gespeichert. F�r einen asynchronen Aufruf wird keine Ausgabevariable deklariert und der Prozess kann sofort nach dem Aufruf weiter ablaufen.
Aufruf einer Operation eines Dienstes. Kann synchron
oder asynchron sein.  als Folge auf eine <receive>-Aktion. Ein <reply> bezieht sich immer auf einen bestimmten Partnerlink und eine bestimmte Variable.

     invoke  Synchroner (requestresponse) oder asynchroner Aufruf eines Web Service
     receivereply  Anbieten einer synchronen oder asynchronen Web Service Schnittstelle








\nocite{BPEL1.1}
\nocite{BPEL2.0}
\nocite{LuebketestSOA}
\nocite{ZhuDataAdequancy}
\nocite{MayerBPELUnit}
\nocite{Alonso2004}
\nocite{W3C2004}
\nocite{W3C2004a}
\nocite{Dostal2005}
\nocite{Hunter2004}
\bibliographystyle{alpha}
\clearpage
  \addcontentsline{toc}{chapter}{\bibname}
\bibliography{referenzen}
  % Abbildungsverzeichnis ins Inhaltsverzeichnis: 
\clearpage
  \addcontentsline{toc}{chapter}{\listfigurename}
  \listoffigures
\begin{appendix}
\chapter{Verwendete Software und Bibliotheken}
\chapter{}
  \chapter{Beispieldaten}
\begin{tiny}
  \begin{verbatim}
<?xml version="1.0" encoding="UTF-8"?>
<tes:testSuite xmlns:tes="http://www.bpelunit.org/schema/testSuite" 
xmlns:auth="http://auth.service.coverage.example.se.uni.hannover.de" 
xmlns:mes="http://message.service.coverage.example.se.uni.hannover.de" 
xmlns:ord="http://order.service.coverage.example.se.uni.hannover.de">
    <tes:name>suite.bpts</tes:name>
    <tes:baseURL>http://localhost:7777/ws</tes:baseURL>
    <tes:deployment>
        <tes:put name="Beipiel" type="activebpel">
            <tes:property name="BPRFile">build/prozess.bpr</tes:property>
            <tes:wsdl>build/wsdl/OrderService.wsdl</tes:wsdl>
        </tes:put>
        <tes:partner name="AuthenticationService" wsdl="build/wsdl/authentificationService.wsdl"/>
        <tes:partner name="MessageService" wsdl="build/wsdl/MessagingService.wsdl"/>
    </tes:deployment>
    <tes:testCases>
        <tes:testCase name="TestCase" basedOn="" abstract="false" vary="false">
            <tes:clientTrack>
                <tes:sendReceive service="ord:OrderHandler" port="bpelSOAP" operation="receiveOrder">
                    <tes:send fault="false" delaySequence="">
                        <tes:data>
                            <ns1:order xmlns:ns1="http://order.service.coverage.example.se.uni.hannover.de">
                                <shipment>express</shipment>
                                <customerId>ExampleId</customerId>
                                <productId>productId</productId>
                            </ns1:order>
                        </tes:data>
                    </tes:send>
                    <tes:receive fault="false">
                        <tes:condition>
                            <tes:expression>.</tes:expression>
                            <tes:value>'Order is accepted'</tes:value>
                        </tes:condition>
                    </tes:receive>
                </tes:sendReceive>
            </tes:clientTrack>
            <tes:partnerTrack name="AuthenticationService">
                <tes:receiveSend service="auth:AuthenticationService" port="bpelSOAP" operation="authenticate">
                    <tes:send fault="false">
                        <tes:data>
                            <ns2:customerData xmlns:ns2="http://auth.service.coverage.example.se.uni.hannover.de">
                                <name>Alex</name>
                                <customerType>VIP</customerType>
                            </ns2:customerData>
                        </tes:data>
                    </tes:send>
                    <tes:receive fault="false">
                        <tes:condition>
                            <tes:expression>.</tes:expression>
                            <tes:value>'ExampleId'</tes:value>
                        </tes:condition>
                    </tes:receive>
                </tes:receiveSend>
            </tes:partnerTrack>
            <tes:partnerTrack name="MessageService">
                <tes:receiveOnly service="mes:MessageService" port="bpelSOAP" operation="notify" fault="false">
                    <tes:condition>
                        <tes:expression>.</tes:expression>
                        <tes:value>'VIP'</tes:value>
                    </tes:condition>
                </tes:receiveOnly>
            </tes:partnerTrack>
        </tes:testCase>
    </tes:testCases>
</tes:testSuite>
\end{verbatim}
\end{tiny}


\end{appendix}
\newpage
\thispagestyle{empty}
\vspace*{10cm}
\Large{\textbf{Eidesstattliche Erkl�rung}}\vspace{1cm}

\normalsize{Hiermit versichere ich, Alex Salnikow, die vorliegende Masterarbeit ohne fremde Hilfe und
nur unter Verwendung der von mir aufgef�hrten Quellen und Hilfsmittel angefertigt zu haben.\vspace{1cm}

Hannover, \today  \ \ \ \ \ \ \ \ \ \ \ \ \ \ \ \ \ \ \ \ \ \ \ \ \ \ \ \ \ \ \ \ \ \ \ \ \ \ \ \ \ \ \ \ \ \ ............................\\

}


\end{document}