 Testen ist ein in der konventionellen Softwareentwicklung unentbehrliches Mittel zur Qualit�tssicherung.
Selbst wenn das Testen nur die Anwesenheit von Fehlern, nicht aber deren
Abwesenheit zeigen kann (sinngem�� nach Dijkstra (1969)), ist das Auffinden jedes einzelnen
existierenden Fehlers f�r die Entwicklung nutzbarer Systeme ein nicht zu untersch�tzender
Vorteil. Testen ist die in der Anwendung einfachste und wahrscheinlich verbreitetste
Methode der Verifikation und existiert in vielen Auspr�gungen (Riedermann (1997), Seite
56 59 Link und Fr�hlich (2002), Seite 5 Beck (2000), Seite 217 219), die sich im Ansatz
oder in der Art der untersuchten Eigenschaften unterscheiden.
 
 
  Framework: obwohl Simulation eine bessere Kontrolle �ber BPEL Prozess erm�glicht h�tte, hat man sich entschieden ..., weil eine Simulation 
  
  
  Testspezifikation: (Testen PUT mit dem Ziel die Korrektheit des implementierten Gesch�ftsprotokolls)
  Der Prozess wird durch Senden und Empfangen von Nachrichten durchgef�hrt. WSDl Spezifiziert nicht in welcher Reihenfolge die einzelnen Operationen aufgerufen werden m�ssen, was bei den asynchronen Operationen wichtig ist. Framework bietet an dieser Stelle M�glichkeit an.
  
  Incoming data (gesendet von PUT) : muss �berpr�ft werden auf Korrektheit.
  
  Outgoing data : mit dem Ziel bestimmte Zweige auszuf�hren, Fault oder compensation handler auszul�sen.  
  
  Der Format der Daten die ausgetauscht werden, ist literal XML, beschrieben als WSDL Nachrichten Typ, XML Schema Element oder XML Schema simple type. Die gesendeten durch PUT Nachrichten werden auf Korrektheit �berpr�ft. Die relevanten f�r den Test Daten werden durch XPath-Ausdr�cke spezifiziert.
  Interaktionsdetails:
  Die Client und Partner m�ssen simuliert werden, um die Reaktionen des BPEL Prozesses auszul�sen und zu testen.
  Interaktionsequenzen .....Tracks Sequenzen Parallel f�r jeden Partner und Client
  
  Fault handling: das Framework ist f�hig die SOAP Faults zu empfangen und zu senden.
  
  testing flows:Parallel warten auf Nachrichten. Framework bring delay sequenzen mit

Testorganisation:  
  Tests sind in die Test suiten organisiert, die test fixture enth�lt. Die Aufgabe der Fixture ist unter anderem den zu testenden BPEL Prozess zu spezifizieren.
  
  Au�erdem ist es m�glich die Test cases mit Metadaten zu erweitern, um die Anforderungen zu verfolgen. 
  
  Ausf�hrung der Tests:
  Real-life testing der PUT wird normal in ein eengine deployt und als Web Service aufgerufen. Alle Partner Web Services m�ssen durch Mocks ersetzt werden. Mocks m�ssen als Web Services aufgerufen werden und selbst Nachrichten verschicken k�nnen. Beim deployen des PUTs m�ssen alle Partner Web Services URIs durch die URIs der Mocks ersetzt werde. Dadurch muss f�r jede engine ein anbieter spezifisches Deployer implementiert werden.
  
  
  Test Results:
  
  W�hrend BPEL execution: Aufgerufene Mocks k�nnen ihre Interaktionen mit PUT protokollieren. So k�nnen manche ausgef�hrte Aktivit�ten identifiziert werden.
  
  Wie in (Li et al. 2005) die Information �ber Ausf�hrung einer bestimmten BPEL Instance aus den log-Files der BPEL Engine zu verwenden. Solche logs, wenn die �berhaupt vorhanden sind, sind nat�rlich anbieter- spezifisch.
  
